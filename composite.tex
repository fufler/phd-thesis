\documentclass[thesis.tex]{subfiles}

\begin{document}

\FloatBarrier
\subsection{Разрушение в~композиционном материале}

Этот раздел посвящён вопросам моделирования разрушений, возникающих в~композиционных материалах при динамическом
нагружении.

Моделирование волновых процессов в~композиционных материалах является достаточно трудоёмкой задачей в~силу сложности
внутреннего устройства этих материалов. С~одной стороны, композиционный материал, обычно состоящий из~матрицы и
армирующих волокон, не может рассматриваться как однородный изотропный
\cite{беклемышева2013численное,беклемышева2013численное2,беклемышева2013численное3,беклемышева2012численное,беклемышева2014численное}
материал. С~другой стороны, моделирование всех составных частей такого материала является очень сложной с~вычислительной
точки зрения задачей, поэтому на~практике не применяется. Традиционный подход к~расчёту задач о~нагружении
композиционных материалов заключается в~использовании осреднённой модели, предполагающей, что такой материал является
анизотропным \cite{петров2014численный}.

Другая особенность задач этого класса заключается в~том, что существует несколько различных механизмов возникновения
разрушения в~композиционном материале, каждый из~которых по~разному влияет на~изменение прочностных характеристик
материала. Более того, на~данный момент не существует \cite{kaddour2013maturity} единой теории, охватывающей все
процессы, происходящие в~этих материалах, поэтому выбор наиболее существенных механизмов разрушения, и, соответственно,
критериев производится из~эмпирических соображений.

В данной работе используемые композиционные материалы предполагаются анизотропными, а для моделирования процессов
разрушения используются наиболее популярные на~текущий момент критерии разрушения, обзор которых был приведён в~разделе
\ref{раздел:критерии-для-композитов}.

\subsubsection{Волновые процессы в~анизотропных средах}

Найдем аналитическое выражение для начальных условий, необходимых для генерации продольных и поперечных волн в~материале
с орторомбической анизотропией. В~таком случае уравнения движения принимают вид

\begin{equation}
    \label{одномерное-уравнение-движения}
    \PDt{\vec u}+\bm A_z\PDz{\vec u}=0,
\end{equation}
а матрица упругости
\begin{small}
\[
    c_{ij} = \begin{pmatrix}
        c_{11} & c_{12} & c_{13} & 0 & 0 & 0 \\
        c_{12} & c_{22} & c_{23} & 0 & 0 & 0 \\
        c_{13} & c_{23} & c_{33} & 0 & 0 & 0 \\
        0 & 0 & 0 & c_{44} & 0 & 0 \\
        0 & 0 & 0 & 0 & c_{55} & 0 \\
        0 & 0 & 0 & 0 & 0 & c_{66}
    \end{pmatrix}.
\]
\end{small}

Собственные значения для матрицы $\bm A_z$ в~таком случае имеют вид
\begin{small}
\[
    \left\{0 , 0, 0, \sqrt{\frac{c_{33}}{\rho}}, \sqrt{\frac{c_{44}}{\rho}}, \sqrt{\frac{c_{55}}{\rho}},
           -\sqrt{\frac{c_{33}}{\rho}}, -\sqrt{\frac{c_{44}}{\rho}}, -\sqrt{\frac{c_{55}}{\rho}}\right\}.
\]
\end{small}

Соответственно, инварианты Римана $\vec r  = \bm  \Omega _z \vec u$ для уравнения \eqref{одномерное-уравнение-движения}
принимают вид
\begin{small}
\begin{equation}
    \label{инварианты-римана-для-одномерного-случая}
    \vec r = \begin{pmatrix}
        0 & 0 & 0 & 1 & 0 & 0 & 0 & 0 & -\frac{c_{13}}{c_{33}} \\
        0 & 0 & 0 & 0 & 1 & 0 & 0 & 0 & 0 \\
        0 & 0 & 0 & 0 & 0 & 0 & 1 & 0 & -\frac{c_{23}}{c_{33}} \\
        0 & 0 & -\sqrt{c_{33}\rho} & 0 & 0 & 0 & 0 & 0 & 1 \\
        0 & -\sqrt{c_{44}\rho} & 0 & 0 & 0 & 0 & 0 & 1 & 0 \\
        -\sqrt{c_{55}\rho} & 0 & 0 & 0 & 0 & 1 & 0 & 0 & 0 \\
        0 & 0 & \sqrt{c_{33}\rho} & 0 & 0 & 0 & 0 & 0 & 1 \\
        0 & \sqrt{c_{44}\rho} & 0 & 0 & 0 & 0 & 0 & 1 & 0 \\
        \sqrt{c_{55}\rho} & 0 & 0 & 0 & 0 & 1 & 0 & 0 & 0
    \end{pmatrix}
    \begin{pmatrix}
        v_x \\
        v_y \\
        v_z \\
        \Sxx \\
        \Sxy \\
        \Sxz \\
        \Syy \\
        \Syz \\
        \Szz
    \end{pmatrix}.
\end{equation}
\end{small}

Исходя из~физического смысла собственных значений матрицы $\bm A_z$, описанного в~разделе \ref{раздел:орторомбическая-анизотропия},
для задания продольной волны вдоль направления $Z$ следует занулить все инварианты, кроме единственного инварианта,
соответствующего продольной волне, т.е. кроме строчки в~\eqref{инварианты-римана-для-одномерного-случая},
соответствующей собственному значению $-\sqrt{\frac{c_{33}}{\rho}}$ (знак минус выбран для удобства):

\begin{equation}
    \label{продольная-волна-вдоль-z}
    \left\{
        \begin{array}{cccccccccccc}
            c_{33}\sigma_{xx} = c_{13}\sigma_{zz}, \\
            \sigma_{xy} = 0, \\
            c_{33}\sigma_{yy} = c_{23}\sigma_{zz}, \\
            v_z\sqrt{c_{33}\rho} = \sigma_{zz}, \\
            \pm v_y\sqrt{c_{44}\rho} = \sigma_{yz}, \\
            \pm v_x\sqrt{c_{55}\rho} = \sigma_{xz}.
        \end{array}
    \right.
\end{equation}

Задав начальные значения неизвестных в~соответствии с~\eqref{продольная-волна-вдоль-z}, получим продольную волну,
распространяющуюся вдоль оси $Z$.

Таким же образом можно получить соотношения для задания поперечной волны, распространяющейся вдоль оси $Z$  и имеющей
вектор поляризации, направленный вдоль оси $X$. Скорость распространения такой волны равняется $-\sqrt{\frac {c_ {55}}{\rho}}$,
поэтому занулив все инварианты, кроме последней строчки в~\eqref{инварианты-римана-для-одномерного-случая}, получим:
\begin{equation}
    \label{поперечная-волна-x-вдоль-z}
    \left\{
        \begin{array}{cccccccccccc}
            c_{33}\sigma_{xx} = c_{13}\sigma_{zz}, \\
            \sigma_{xy} = 0, \\
            c_{33}\sigma_{yy} = c_{23}\sigma_{zz}, \\
            \pm v_z\sqrt{c_{33}\rho} = \sigma_{zz}, \\
            \pm v_y\sqrt{c_{44}\rho} = \sigma_{yz}, \\
            v_x\sqrt{c_{55}\rho} = \sigma_{xz}.
        \end{array}
    \right.
\end{equation}

Аналогичным образом можно получить аналитические выражения, которыми задаются продольные и поперечные волны,
распространяющиеся вдоль других координатных осей.

На рис.~\ref{рис:анизотропные-волны-p-x}--\ref{рис:анизотропные-волны-s-z} приведены результаты расчёта тестовых задач
о распространении продольных и поперечных волн в~анизотропном материале, которые были решены численно для верификации
используемого программного комплекса. На каждом рисунке изображена одна из~компонент вектора неизвестных
$\vec u = \{v_x, v_y, v_z, \Sxx, \Sxy, \Sxz, \Syy, \Syz, \Szz\}$,  при этом величина $u_i$  на~рисунках~--- результат
численного решения,  $\hat u_i$~--- значение, вычисленное аналитически.


Также в~рамках верификации программного комплекса был проведён расчёт модельной задачи о~точечном взрыве в~анизотропном
теле, результаты которого представлены на~рис.~\ref{рис:точечный-взрыв}.

На рис.~\ref{рис:точечный-взрыв} изображено поле скоростей. Фронты продольных волн в~изотропном теле имеют сферическую
форму.Фронт продольных волн в~анизотропном материале принимает форму эллипса, что вызвано различием скоростей звука в
таком материале в~разных направлениях. Также можно наблюдать распространение сдвиговых волн \cite {огурцов1969анализ},
амплитуда которых максимальна на~направлении, составляющем 45\degree с~осями. Эти волны~--- существенное проявление
анизотропии материала, в~изотропном случае их нет.

\twofigs{рис:точечный-взрыв}{Точечный взрыв в~изотропном теле. Плоский срез.}
        {png/anisotropy/exp_i_00.png}{Изотропное тело}
        {png/anisotropy/exp_ai_00.png}{Анизотропное тело}


На рис.~\ref{рис:точечный-взрыв-в-анизотропном-материале-2} изображено отражение волн от~границы тела в~анизотропном
случае.

\figfull{рис:точечный-взрыв-в-анизотропном-материале-2}{png/anisotropy/exp_ai_01.png}
        {Точечный взрыв в~анизотропном теле. Плоский срез. Отражение от~границ}

\begin{python}
    import gen_anisotropic_waves_plots
\end{python}

\clearpage

\subsubsection[Сравнение критериев разрушения\\композиционного материала]{Сравнение критериев разрушения композиционного материала}

Приведенные в~разделе \ref{раздел:критерии-для-композитов} критерии используются во~многих коммерческих пакетах. При
расчете многослойного материала рассматривается в~отдельности каждый слой, рассчитывается его плосконапряжённое
состояние, и выбранные критерии последовательно применяются к~каждому из~слоев. В~таких расчетах
присутствует количественное и качественное расхождение между теоретическими прогнозами и экспериментальными
наблюдениями, что наглядно демонстрируется результатами международного проекта WWFE \cite{kaddour2013maturity}. При
выборе критерия предполагается использовать ряд рекомендаций, и сложность выбора объясняется отсутствием единой
всеобъемлющей теории.

Критерии Мизеса и основанные на~нем критерии Цая-Хилла, Цая-Ву и Друкера-Прагера сводят все значения напряжений и все
пороги прочности в~единую формулу, в~то время как критерии Хашина и Пака используют несколько уравнений: по~одному на
каждый механизм разрушения. Соответственно, при моделировании на~уровне монослоя (однонаправленное армирование) критерии
Хашина и Пака дают не только область разрушенного материала, но и определяют, по~какому механизму произошло разрушение.
Критерий Пака также дает направления образовавшихся трещин. Это позволяет более корректно учитывать реологию разрушенной
области. Однако критерии Хашина и Пака напрямую опираются на~однонаправленное армирование материала, то есть их
применение при моделировании на~уровне субпакетов не является корректным. При этом применение критериев Цая-Хилла, Цая-Ву
и Друкера-Прагера на~уровне субпакетов вполне корректно: данные критерии будут работать для любого анизотропного
материала, у~которого определены пороги прочности по~необходимым направлениям.

Для расчета модельных задач и сравнения критериев друг с~другом были взяты параметры материала из~публикации Н. Ху и
коллег \cite{hu2007stable} для материалов GFRP (glass fiber reinforced polymer) и CFRP (carbon fiber reinforced
polymer). Эти материалы были выбраны в~качестве модельных, так как для них имеются все необходимые константы.
Рассмотрены задачи соударения шарика с~монослоем, а также с~двумя скрещенными под 90\degree слоями.

\paragraph{Столкновение шарика с~монослоем}

Общий вид геометрии задачи приведен на~рис \ref{рис:столкновение-с-монослоем-постановка}. Рассматривалось столкновение
стального шарика с~монослоем из~материала GFRP, направление армирования~--- ось Х в~первом случае и ось Z во~втором.
Толщина слоя 3~см, радиус шарика~--- 1.5~см, удар по~нормали к~поверхности, скорость шарика 90~м/с.

\fig{рис:столкновение-с-монослоем-постановка}{png/destruction/gen-cg.png}{Общий вид расчетной области}

Результаты расчетов для столкновения с~монослоем, волокна которого направлены по~оси Х, приведены на~рис.
\ref{рис:столкновение-с-x-монослоем-x}--\ref{рис:столкновение-с-x-монослоем-z} в~виде областей разрушения в~различных
проекциях для различных критериев разрушения.

Результаты расчетов для столкновения с~монослоем, волокна которого направлены по~оси Z, приведены на~рис.
\ref{рис:столкновение-с-z-монослоем-x}--\ref{рис:столкновение-с-z-монослоем-z} в~виде областей разрушения в~различных
проекциях для различных критериев разрушения.


\paragraph{Столкновение шарика с~двумя скрещенными монослоями}

Общий вид геометрии задачи приведен на~рис \ref{рис:столкновение-с-двумя-монослоями-постановка}. Рассматривалось
столкновение стального шарика с~двумя скрещенными под 90\degree\ монослоями из~материалов GFRP и CFRP, направления
армирования~--- ось Х для верхнего слоя и ось Y для нижнего. Толщина слоя 1.5~см, радиус шарика~--- 1.5~см, удар по
нормали к~поверхности, скорость шарика 200~м/с.

\fig{рис:столкновение-с-двумя-монослоями-постановка}{png/destruction/gen-2l.png}{Общий вид расчетной области}

Результаты расчетов для столкновения с~монослоями, изготовленными из~GFRP, приведены
на~рис.~\ref{рис:столкновение-с-gfrp-x}--\ref{рис:столкновение-с-gfrp-z} в~виде областей разрушения в~различных
проекциях для различных критериев разрушения. Результаты расчетов для столкновения с~монослоями, изготовленными из~CFRP,
приведены на~рис.~\ref{рис:столкновение-с-cfrp-x}--\ref{рис:столкновение-с-cfrp-z} в~виде областей разрушения
в~различных проекциях для различных критериев разрушения. Области расслоения для обоих материалов приведены на~\ref{рис:расслоение-cfrp-gfrp}.


\paragraph{Анализ результатов, сравнение критериев}
При сравнении приведенных критериев друг с~другом прежде всего можно заметить существенное количественное и качественное
различие результатов. Форма и размеры областей разрушения существенно различаются от~критерия к~критерию при любой из
рассмотренных постановок задачи столкновения.

Критерии Друкера-Прагера, Хашина, Пака, Цая-Хилла и Цая-Ву были выбраны для рассмотрения как наиболее популярные в
зарубежных периодических изданиях и используемые в~коммерческих пакетах по~моделированию задач механики деформируемого
твёрдого тела. Из них наиболее физически обоснованными для композита являются критерии Хашина и Пака --- они учитывают
особенности внутренней структуры композита с~однонаправленным армированием. Остальные же разрабатывались для однородного
анизотропного материала.

Можно видеть, что критерии Друкера-Прагера, Хашина и Цая-Ву дают схожие оценки размера области разрушения. Критерий
Цая-Хилла дает заметно меньший размер, что объясняется тем, что он не различает различные пороги прочности на~сжатие и
растяжение. Области, которые разрушаются по~механизму растяжения с~соответствующим низким порогом прочности, данный
критерий не отображает.

\begin{python}
     import gen_monolayer_images
\end{python}

\begin{python}
     import gen_two_layers_images
\end{python}

\clearpage

Критерий Пака дает заметно больший размер области разрушения. В~рамках данного сравнения это затрудняет анализ
результатов. При этом стоит учитывать, что критерий Пака дает не скалярную меру разрушения, а векторную – направление
микротрещины в~разрушенном узле расчётной сетки. Элементарный объем материала, в~котором возникла микротрещина, меняет
свои упругие характеристики не так сильно, как полностью разрушившийся элементарный объем (множественные трещины,
раскрашивание). Поэтому прочие критерии разрушения, которые могут отображать только скалярное разрушение, могут
игнорировать такие узлы. Соответственно, сравнивать данный критерий с~остальными приходится по~влиянию результатов его
работы на~другие показатели~--- например, размеры области расслоения. При этом критерий Пака может обнаруживать только
микротрещины (в данной реализации ровно одну на~узел сетки, при необходимости можно увеличить до~трех взаимно
перпендикулярных трещин на~узел сетки), но не может обнаруживать полностью разрушенный объем. Критерий Пака
целесообразно комбинировать с~другими критериями для полного покрытия существующих механизмов разрушения. Основным
недостатком данного критерия является наличие четырех внутренних параметров модели, которые могут зависеть от~постановки
задачи, поэтому его верификация для задач низкоскоростного соударения представляет из~себя отдельную проблему.


Можно видеть, что критерий Хашина при ударе вдоль оси армирования обнаруживает несколько областей разрушения различной
формы: центральная область под ударником, <<лучи>> к~углам образца и два кольца краевых разрушений. Из остальных
критериев только критерий Друкера-Прагера обнаруживает <<лучи>> и только критерий Цая-Ву обнаруживает краевые эффекты.
При расчете субпакетов, для которых некорректно применять критерий Хашина непосредственно, технически можно применить
комбинацию критериев Друкера-Прагера и Цая-Ву, но такой интегральный критерий не будет иметь физического обоснования,
результаты таких расчетов также будут иметь сомнительную достоверность.

Можно видеть, что области расслоения также существенно зависят от~примененного критерия. Аналогично областям объемного
разрушения, сходные результаты дают критерии Хашина, Друкера-Прагера и Цая-Ву. Применение критериев Пака и Цая-Хилла
дает существенно отличающиеся области расслоения.

Таким образом, результаты расчетов модельных задач показывают, что даже для простейшей постановки области разрушения,
получаемые в~расчетах с~использованием различных критериев, существенно отличаются друг от~друга как по~форме, так и по
размерам. При этом необходимо заметить, что получить точную форму и размер области разрушения материала в~реальном
эксперименте весьма нетривиально. В~доступных результатах реальных экспериментов измерялись глубина вмятины
после соударения и размер расслоенной области по~данным ультразвукового зондирования. Получение достоверной глубины
вмятины в~численном эксперименте крайне затруднительно, так как требует разработки математической модели реологии
разрушенного материала, что является отдельной задачей (крайне нетривиальной для композитов в~силу их сложной структуры
и большого количества механизмов разрушения). Верификацию критериев разрушения предлагается проводить на~основании
данных о~размере и форме расслоенной области, полученных в~численном и реальном эксперименте. Существенное различие
результатов расчета модельных задач дает основание полагать, что сравнение численного эксперимента с~реальным может
помочь не только выявить наиболее подходящий критерий или совокупность критериев, но и определить внутренние параметры
моделей разрушения композита.

\subsubsection{Динамическая прочность композиционных панелей}

\paragraph{Постановка задачи}

Рассматривается задача о~низкоскоростном ударе по~трёхстрингерной панели. Рассматривается 2 точки нанесения удара: в
стрингер и в~обшивку между двумя соседними стрингерами. Удар наносится цилиндрическим стальным ударником с~диаметром
закругления на~конце 25.4~мм. Геометрия панелей и вид расчетной области приведены ниже.

В начальный момент времени напряжения в~конструкции и ударнике отсутствуют, конструкция покоится, все точки ударника
имеют одинаковую скорость $\vec v$, направленную по~нормали к~элементу обшивки. Скорость ударника пересчитывается из
энергии ударника, использованной в~эксперименте. При этом считается, что масса ударника 10~кг, а полная энергия ударника
равна его кинетической энергии.

Края панели считаются жёстко закреплёнными, все прочие неконтактные границы конструкции и ударника считаются свободными.

Контактные границы между слоями преграды считаются с~условием полного слипания. На контактирующих поверхностях задаётся
равенство компонентов скорости. Контакт между ударником и преградой полагается удовлетворяющим условию свободного
скольжения.

\figfull{рис:схема-панели}{png/3-stringer-panel/panel.png}{Геометрия панели с~подклеенными стрингерами}

Строение конструкции приведено на~рис.~\ref{рис:схема-панели}. Общий вид полной расчетной
области для рассматриваемых постановок показан на~рис.~\ref{рис:удар-в-стрингер}, \ref{рис:удар-в-обшивку}.

\figfull{рис:удар-в-стрингер}{png/3-stringer-panel/panel-3d-stringer.png}{Удар в~стрингер, общий вид расчётной области}
\figfull{рис:удар-в-обшивку}{png/3-stringer-panel/panel-3d-cover.png}{Удар в~обшивку, общий вид расчётной области}

\paragraph{Параметры материала}
\label{раздел:параметры-панели}

Каждый отдельный субпакет в~преграде рассматривается как анизотропное тело. Геометрия конструкции ниже структурного
уровня субпакета, обусловленная укладкой монослоёв в~субпакете, не рассматривается. Упругие характеристики
монослоев приведены в~таблице \ref{таб:параметры-монослоя} (данные для статической нагрузки), а ударника~---
в таблице \ref{таб:параметры-стали}.

\begin{table}[h!]
    \begin{center}
        \begin{tabular}{|l|p{1.5cm}|p{1.5cm}|p{1.5cm}|p{1.5cm}|p{1.5cm}|p{1.0cm}|p{1.3cm}|}
            \hline
            Материал  & $E_{11}+$, МПа & $E_{11}-$, МПа & $E_{22}+$, МПа & $E_{22}-$, МПа & $G_{12}$, МПа  & $\nu$ & $\rho$, $\textnormal{кг/м}^3$ \\
            \hline
            Монослой  & 16483 & 13376 & 805 & 854 & 437 & 0.32 & 1580 \\
            \hline
        \end{tabular}
    \end{center}
    \caption{Упругие характеристики монослоя}
    \label{таб:параметры-монослоя}
\end{table}

\begin{table}[h!]
    \begin{center}
        \begin{tabular}{|l|p{1.5cm}|p{1.5cm}|p{1.5cm}|p{1.5cm}|p{1.5cm}|}
            \hline
            Материал  & $E_{11}+$, ГПа & $\nu$ & $\rho$, $\textnormal{кг/м}^3$ & $\lambda$, ГПа & $\mu$, ГПа \\
            \hline
            Сталь  & 200 & 0.28 & 7500 & 99.43 & 78.13 \\
            \hline
        \end{tabular}
    \end{center}
    \caption{Упругие характеристики ударника}
    \label{таб:параметры-стали}
\end{table}

\begin{table}[h!]
    \begin{center}
        \begin{tabular}{|L{7cm}|C{6cm}|}
            \hline
            Материал  &  Предельно допустимая нагрузка, МПа \\
            \hline
            Сжатие вдоль волокон & 2630 \\
            \hline
            Растяжение вдоль волокон  & 1530 \\
            \hline
            Сжатие поперёк волокон & 86 \\
            \hline
            Растяжение поперёк волокон  & 213 \\
            \hline
            Сдвиговая нагрузка  & 113 \\
            \hline
        \end{tabular}
    \end{center}
    \caption{Прочностные характеристики субпакетов}
    \label{таб:параметры-субпакета}
\end{table}

\newpage
В расчёте используется следующая укладка монослоев в~субпакете:
45\degree/0\degree/-45\degree/0\degree/0\degree/90\degree/0\degree/0\degree/-45\degree/0\degree/45\degree.
Для определения эффективных характеристик субпакета в~данной работе используется простое осреднение, основанное на
процентном содержании монослоев с~каждой ориентацией волокон. На основании данных табл.~\ref{таб:параметры-субпакета}
итоговая матрица упругости для осредненного субпакета принимается в~виде \eqref{матрица-упругости-для-панели} (значения
указаны в~МПа).

\begin{equation}
    C_{ij} = \begin{pmatrix}
    10300 & 6960 & 6960 0 & 0 & 0 \\
    6960 & 23250 & 6960 & 0 & 0 & 0 \\
    6960 & 6960 & 10300 & 0 & 0 & 0 \\
    0 & 0 & 0 & 5010 & 0 & 0 \\
    0 & 0 & 0 & 0 & 1670 & 0 \\
    0 & 0 & 0 & 0 & 0 & 5010
    \end{pmatrix}.
    \label{матрица-упругости-для-панели}
\end{equation}

Здесь следует отметить, что полученные таким образом реологические коэффициенты для осредненного анизотропного субпакета
не обладают высокой достоверностью для данной постановки задачи. Они де-факто пересчитаны из~результатов статических
испытаний, а применять их предполагается для динамической задачи.

С точки зрения структуры композита количественные значения реологических параметров как для статических, так и для
динамических задач определяются свойствами волокон и матрицы, адгезией между ними, а также особенностями укладки.
Однако для статических и для динамических задач проявляются разные свойства структурных элементов. Ярким примером
отличия является воздействие одноосной нагрузки вдоль направления волокон. В~случае статической нагрузки модули Юнга при
сжатии и при растяжении отличаются. В случае динамической нагрузки в~упругом приближении разницы нет, импульсы сжатия и
растяжения распространяются с~одинаковой скоростью, определяемой исключительно упругими свойствами волокон (волна в
матрице <<вытягивается>> за~более быстрой волной в~волокнах, легкими искажениями волнового фронта в~случае большой
удельной доли волокон можно пренебречь).

Для получения коэффициентов с~более высокой достоверностью (и, соответственно, для повышения качества решения) возможен
другой подход к~осреднению~--- задание свойств волокон и матрицы, адгезии и укладки, выполнение прямого численного
моделирования микроструктуры композита в~динамических задачах, после чего решение обратной задачи для получения
эффективных характеристик осредненных структурных элементов. Однако такой подход предполагает использование большой
вычислительной мощности и требует отдельного исследования.

\paragraph{Удар в~обшивку, 90 Дж, критерий Друкера-Прагера}

На рис.~\ref{рис:удар-в-обшивку-друкер-прагер} показана форма разрушенной области при нанесении удара в~обшивку (между
стрингерами панели). Энергия удара составляет 90 Дж (скорость ударника 4.24~м/с). Параметры осреднённых субпакетов
приведены в~разделе~\ref{раздел:параметры-панели}. Для численного моделирования разрушений используется критерий
Друкера-Прагера,
обобщенный на~случай анизотропного материала. Итоговый размер разрушенной области (в проекции на~поверхность панели)
составляет~40х40~мм.

Полученная область симметричная, характерная эллиптическая форма не получена, все разрушения локализованы строго в~зоне
удара и вызваны первичными фронтами волн.

\figfull{рис:удар-в-обшивку-друкер-прагер}{png/3-stringer-panel/cover-drucker-prager.png}
    {Форма разрушенной области. Вид со~стороны ударника}

\newpage
\paragraph{Удар в~обшивку, 90 Дж, критерий Хашина}

На рис.~\ref{рис:удар-в-обшивку-хашин} показана форма разрушенной области при нанесении удара в~обшивку (между
стрингерами панели). Энергия удара составляет 90 Дж (скорость ударника 4.24~м/с). Параметры осреднённых субпакетов
приведены в~разделе \ref{раздел:параметры-панели}. Для моделирования разрушений используется критерий Хашина. Итоговый
размер разрушенной области (в проекции на~поверхность панели) составляет 60х45~мм.

Отдельно стоит отметить, что в~данном рассчёте полученная область не только вытянутая, но и имеет характерную тонкую
перемычку посередине (в англоязычной литературе используется термин <<peanut shape>>~--- <<форма арахиса>>). Данная форма
наблюдается в~экспериментах, однако относится исключительно к~форме расслоения между монослоями. Получение данной формы
для осреднённого субпакета является нетривиальным фактом, требующим дополнительного изучения на~предмет физической
корректности.

\figfull{рис:удар-в-обшивку-хашин}{png/3-stringer-panel/cover-hashin.png}
    {Форма разрушенной области. Вид со~стороны ударника}

\newpage
\paragraph{Удар в~обшивку, 90 Дж, критерий Пака}

На рис.~\ref{рис:удар-в-обшивку-пак} показана форма разрушенной области при нанесении удара в~обшивку (между стрингерами
панели). Энергия удара составляет 90 Дж (скорость ударника 4.24~м/с). Параметры осреднённых субпакетов приведены в
разделе \ref{раздел:параметры-панели}. Для моделирования разрушений используется критерий Пака. Итоговый размер
разрушенной области (в проекции на~поверхность панели) составляет 120х52~мм.

Область вытянутая, общая форма и размер хорошо согласуются с~данными экспериментов. Стоит отметить, что в~данной серии
рассчётов критерий Пака даёт наибольший размер разрушенной области.

\figfull{рис:удар-в-обшивку-пак}{png/3-stringer-panel/cover-puck.png}
    {Форма разрушенной области. Вид со~стороны ударника}

\newpage
\paragraph{Удар в~обшивку, 90 Дж, критерий Цая-Хилла}

На рис.~\ref{рис:удар-в-обшивку-цай-хилл} показана форма разрушенной области при нанесении удара в~обшивку (между
стрингерами панели). Энергия удара составляет 90 Дж (скорость ударника 4.24~м/с). Параметры осреднённых субпакетов
приведены в~разделе \ref{раздел:параметры-панели}. Для моделирования разрушений используется критерий Цая-Хилла.
Итоговый размер разрушенной области (в проекции на~поверхность панели) составляет 40х40~мм.

Полученная область симметричная, характерная эллиптическая форма не получена, все разрушения локализованы строго в~зоне
удара и вызваны первичными фронтами волн.

\figfull{рис:удар-в-обшивку-цай-хилл}{png/3-stringer-panel/cover-tsai-hill.png}
    {Форма разрушенной области. Вид со~стороны ударника}

\newpage
\paragraph{Удар в~обшивку, 90 Дж, критерий Цая-Ву}

На рис.~\ref{рис:удар-в-обшивку-цай-ву} показана форма разрушенной области при нанесении удара в~обшивку (между
стрингерами панели). Энергия удара составляет 90 Дж (скорость ударника 4.24~м/с). Параметры осреднённых субпакетов
приведены в~разделе \ref{раздел:параметры-панели}. Для моделирования разрушений используется критерий Цая-Ву. Итоговый
размер разрушенной области (в проекции на~поверхность панели) составляет 52х44~мм.

Полученная область симметричная, эллиптическая форма с~характерным соотношением осей не получена, все разрушения
локализованы строго в~зоне удара и вызваны первичными фронтами волн.

\figfull{рис:удар-в-обшивку-цай-ву}{png/3-stringer-panel/cover-tsai-wu.png}
    {Форма разрушенной области. Вид со~стороны ударника}

\newpage
\paragraph{Удар в~стрингер, 135 Дж, критерий Друкера-Прагера}

Форма разрушенной области при нанесении удара в~стрингер изображена на рис.~\ref{рис:удар-в-стрингер-друкер-прагер}.
Энергия удара составляет 135 Дж (скорость ударника 5.2~м/с). Параметры осреднённых субпакетов приведены в~разделе
\ref{раздел:параметры-панели}. Чёрным цветом на~рисунке показано положение стрингера. Выделенное направление субпакета
(ось 0\degree\ укладки) направлено перпендикулярно к~стрингеру. Для моделирования разрушений используется критерий
Друкера-Прагера. Итоговый размер разрушенной области (в проекции на~поверхность панели) составляет 40х40~мм.

Полученная область симметричная, характерная эллиптическая форма не получена, все разрушения локализованы строго в~зоне
удара и вызваны первичными фронтами волн.

\figfull{рис:удар-в-стрингер-друкер-прагер}{png/3-stringer-panel/stringer-drucker-prager.png}
    {Форма разрушенной области. Вид со~стороны ударника}

\newpage
\paragraph{Удар в~стрингер, 135 Дж, критерий Хашина}

На рис.~\ref{рис:удар-в-стрингер-хашин} показана форма разрушенной области при нанесении удара в~стрингер. Энергия удара
составляет 135 Дж (скорость ударника 5.2~м/с). Параметры осреднённых субпакетов приведены в~разделе
\ref{раздел:параметры-панели}. Чёрным цветом на~рисунке показано положение стрингера. Выделенное направление субпакета (ось
0\degree\ укладки) направлено перпендикулярно к~стрингеру. Для моделирования разрушений используется критерий Хашина.
Итоговый размер разрушенной области (в проекции на~поверхность панели) составляет 52х35~мм.

Получена эллиптическая форма разрушенной области. Из особенностей стоит отметить, что наблюдаются незначительные
разрушения в~субпакетах стрингера.

\figfull{рис:удар-в-стрингер-хашин}{png/3-stringer-panel/stringer-hashin.png}
    {Форма разрушенной области. Вид со~стороны ударника}


\newpage
\paragraph{Удар в~стрингер, 135 Дж, критерий Пака}

На рис.~\ref{рис:удар-в-стрингер-пак} показана форма разрушенной области при нанесении удара в~стрингер. Энергия удара
составляет 135 Дж (скорость ударника 5.2~м/с). Параметры осреднённых субпакетов приведены в~разделе
\ref{раздел:параметры-панели}. Чёрным цветом на~рисунке показано положение стрингера. Выделенное направление субпакета (ось
0\degree\ укладки) направлено перпендикулярно к~стрингеру. Для моделирования разрушений используется критерий Пака.
Итоговый размер разрушенной области (в проекции на~поверхность панели) составляет 65х32~мм.

Получена эллиптическая форма разрушенной области. Размер разрушенной
области наибольший среди серии расчётов.

\figfull{рис:удар-в-стрингер-пак}{png/3-stringer-panel/stringer-puck.png}
    {Форма разрушенной области. Вид со~стороны ударника}

\newpage
\paragraph{Удар в~стрингер, 135 Дж, критерий Цая-Хилла}

На рис.~\ref{рис:удар-в-стрингер-цай-хилл} показана форма разрушенной области при нанесении удара в~стрингер. Энергия
удара составляет 135 Дж (скорость ударника 5.2~м/с). Параметры осреднённых субпакетов приведены в~разделе
\ref{раздел:параметры-панели}. Чёрным цветом на~рисунке показано положение стрингера. Выделенное направление субпакета (ось
0\degree\ укладки) направлено перпендикулярно к~стрингеру. Для моделирования разрушений используется критерий Цая-Хилла.
Итоговый размер разрушенной области (в проекции на~поверхность панели) составляет 40х40~мм.

Полученная область симметричная, характерная эллиптическая форма не получена, все разрушения локализованы строго в~зоне
удара и вызваны первичными фронтами волн.

\figfull{рис:удар-в-стрингер-цай-хилл}{png/3-stringer-panel/stringer-tsai-hill.png}
    {Форма разрушенной области. Вид со~стороны ударника}

\newpage
\paragraph{Удар в~стрингер, 135 Дж, критерий Цая-Ву}

На рис.~\ref{рис:удар-в-стрингер-цай-ву} показана форма разрушенной области при нанесении удара в~стрингер. Энергия
удара составляет 135 Дж (скорость ударника 5.2~м/с). Параметры осреднённых субпакетов приведены в~разделе
\ref{раздел:параметры-панели}. Чёрным цветом на~рисунке показано положение стрингера. Выделенное направление субпакета (ось
0\degree\ укладки) направлено перпендикулярно к~стрингеру. Для моделирования разрушений используется критерий Цая-Ву.
Итоговый размер разрушенной области (в проекции на~поверхность панели) составляет 45х30~мм.

Полученная область симметричная, эллиптическая форма с~характерным соотношением осей не получена, все разрушения
локализованы строго в~зоне удара и вызваны первичными фронтами волн.

\figfull{рис:удар-в-стрингер-цай-ву}{png/3-stringer-panel/stringer-tsai-wu.png}
    {Форма разрушенной области. Вид со~стороны ударника}

\paragraph{Выводы}

Основные результаты по~сравнению рассмотренных критериев разрушения композитов при решении динамических задач
представлены в~таблицах \ref{таб:сравнение-критериев} и \ref{таб:сравнение-расчётов-с-экспериментом}.

\begin{table}
    \centering
    \small
    \begin{tabular}{|L{4.2cm}|C{1.9cm}|C{2.0cm}|C{2.0cm}|C{1.9cm}|C{1.9cm}|} \hline
    & Друкер-Прагер & Хашин & Пак & Цай-Хилл & Цай-Ву \\ \hline
    Учёт наличия матрицы и волокон & Нет & Да & Да & Нет & Нет \\ \hline
    Различие пределов на~сжатие и растяжение & Да & Да & Да & Нет & Да \\ \hline
    Различие механизмов разрушения & Нет & Да & Да & Нет & Нет \\ \hline
    Используемая мера разрушения & Скаляр & Скаляр & Вектор & Скаляр & Скаляр \\ \hline
    Отсутствие внутренних параметров модели & Да & Да & Нет & Да & Да \\ \hline
    Целевой материал & \scriptsize{Однородный анизотропный материал} & \scriptsize{Армированный монослой} &
                       \scriptsize{Армированный монослой} & \scriptsize{Однородный анизотропный материал} &
                       \scriptsize{Однородный анизотропный материал} \\ \hline
    \end{tabular}
    \caption{Физический смысл критерия, допущения и область применения}
    \label{таб:сравнение-критериев}
\end{table}

\begin{table}[h!]
    \centering
    \small
    \begin{tabular}{| L{4.1cm} | C{1.6cm} | C{1.6cm} | C{1.6cm} | C{1.6cm} | C{1.6cm} | C{1.6cm} |}
        \hline
        \multirow{2}{2.0cm}{\textbf{Постановка задачи}} & \multicolumn{6}{ c| } {\textbf{Размер разрушенной области, мм}} \\
        \cline{2-7}
        & Друкер-Прагер & Хашин & Пак & Цай-Хилл & Цай-Ву & Эксперимент \\
        \hline
        Удар в~обшивку, 90 Дж & 40x40 & 60x45 & 100x52 & 40x40 & 52x44 & 120x65 \\
        \hline
        Удар в~стрингер, 135 Дж & 40x40 & 52x35 & 65x32 & 40x40 & 45x30 & 80x30 \\
        \hline
    \end{tabular}
    \caption{Сравнение результатов расчёта с~экспериментом}
    \label{таб:сравнение-расчётов-с-экспериментом}
\end{table}

Таким образом, строго говоря, ни один из~распространенных критериев разрушения не адаптирован для моделирования
армированного композита в~приближении осреднённых субпакетов. Критерии Друкера-Прагера, Цая-Хилла и Цая-Ву ориентированы
на однородный анизотропный материал и не учитывают того, что для армированного композита есть различные механизмы
внутреннего разрушения, обусловленные наличием матрицы и волокон. Критерии Хашина и Пака учитывают особенности строения
композита, но выведены для монослоя с~однонаправленным армированием, их использование для субпакета с~некоторой укладкой
монослоёв не является вполне корректным.

Фактическое применение в~данной работе указанных критериев для численного решения динамической задачи о~низкоскоростном
ударе показывает, что критерий Пака даёт удовлетворительные результаты. Однако критерий Пака содержит четыре внутренних
параметра, которые не измеримы напрямую в~натурном эксперименте. Таким образом, результат, полученный с~использованием
критерия Пака, является функцией параметров монослоя, геометрии укладки монослоёв в~субпакет и, возможно, каких-либо еще
скрытых параметров композита. Использованные в~данной работе значения внутренних параметров критерия Пака позволили
получить решение приемлемого качества, но обобщение этого результата на~другие материалы и конструкции будет
затруднительно.

Таким образом, на~данный момент для расчётов динамических задач можно рекомендовать к~применению критерий Пака после
выполнения калибровки для каждого конкретного материала и определения внутренних параметров.

Можно ожидать, что проблема выбора критерия разрушения и его параметров будет снята при моделировании структуры
композита до~уровня монослоя. В~этом случае будет физически корректно применение критерия Хашина, учитывающего
особенности структуры композита и при этом не содержащего внутренних параметров. Однако моделирование интересных с
практической точки зрения конструкций до~уровня монослоя за~приемлемое время невозможно на~современном уровне развития
вычислительной техники.

Возможным вариантом решения проблемы видится разработка метода численного осреднения характеристик композита под
действием динамической нагрузки. При этом предлагается выполнение прямого численного моделирования структуры композита
на примере небольшого образца (до уровня монослоя или до~уровня матрицы и волокон). В~дальнейшем выполняется решение
обратной задачи для получения эффективных характеристик осредненных субпакетов, как упругих (матрица реологических
параметров), так и прочностных (пределы прочности и параметры критериев разрушения). Однако данный подход требует
большой вычислительной мощности и является задачей отдельного исследования.

\end{document}
