\documentclass[thesis.tex]{subfiles}

\begin{document}

    \subsection{Метод маркеров}

    Классический метод маркеров и ячеек \cite{harlow1965numerical} разработан для расчёта задач гидродинамики и предполагает
    использование специальных точек (маркеров) для определения тех ячеек расчётной сетки, что заполнены жидкостью.
    Попытка применить тот же самый метод для расчёта задач механики деформируемого твёрдого тела сталкивается с
    определёнными сложностями, связанными как с~различием решаемых уравнений, так и особенностями используемых численных
    методов.

    Одним из~основных преимуществ сеточно-характеристического метода, используемого в~данной работе для решения задач
    механики деформируемого твёрдого тела, является корректный учёт граничных и контактных условий разного рода. Для
    постановки большинства из~этих условий требуется иметь информацию о~локальной топологии расчётной сетки.
    Классический метод маркеров не предусматривает механизмов восстановления границы в~процессе расчёта, поэтому в
    данной работе отдельное внимание уделено этому весьма важному процессу. Для решения этой задачи можно предложить два
    варианта: восстановление границы на~каждом шаге по~набору маркеров или использование отдельной подвижной сетки для
    отслеживания перемещения границы. Задача восстановления границы в~общем случае для трёхмерной расчётной области
    является очень сложной и ресурсоёмкой, поэтому в~данной работе используется второй подход: численный метод
    предполагает одновременное использование как лагранжевых сеток, так и эйлеровых. При этом вместо классических
    маркерных точек используется полноценная поверхностная неструктурированная сетка из~треугольников, которая в
    дальнейшем тексте называется маркерной сеткой (см.~рис.~\ref{fig:markered-mesh}).

    \figfull{fig:markered-mesh}{png/markers/mesh.png}
        {Расчётная (белые линии) кубическая маркерная сетка и поверхностная сетка из~треугольников (синие линии)}

    Тем не менее, наличие специальной сетки, отслеживающей положение границы, не избавляет от~необходимости выполнять
    восстановление границы на~неподвижной сетке, так как для корректной работы сеточно-характеристического метода
    требуется однозначно определять, лежит ли расчётный узел на~границе тела или нет. Граница на~расчётной сетке
    восстанавливается в~несколько этапов (см рис.~\ref{fig:markers_borders_reconstruction}):

    \begin{enumerate}
        \item Выделение граничных ячеек на~сетке из~параллелепипедов (ячейка считается граничной, если она пересекается
              хотя бы с~одним из~треугольников поверхностной маркерной сетки).
        \item Выделение внутренних ячеек (ячейка считается внутренней, если она целиком лежит внутри тела, ограниченного
              поверхностной маркерной сеткой).
        \item Нахождение граничных узлов в~расчётной области (расчётный узел считается граничным, если он входит только
              в~граничную ячейку и не входит во~внутреннюю ячейку).
    \end{enumerate}

    \begin{figure}[ht!]
        \centering
        \subcaptionbox{Выделение граничных ячеек}
            {\tikzset{every picture/.style={scale=0.45}}\subfile{tikz/markers_border_reconstruction}}
        \subcaptionbox{Выделение внутренних ячеек}
            {\tikzset{every picture/.style={scale=0.45}}\subfile{tikz/markers_border_reconstruction_fill}}
        \subcaptionbox{Выделение граничных узлов}
            {\tikzset{every picture/.style={scale=0.45}}\subfile{tikz/markers_border_reconstruction_done}}
        \caption{Нахождение граничных узлов в~расчётной области}
        \label{fig:markers_borders_reconstruction}
    \end{figure}

    \begin{figure}[ht!]
        \centering
        \subcaptionbox{Нахождение первой внутренней ячейки}
            {\tikzset{every picture/.style={scale=0.45}}\subfile{tikz/markers_border_reconstruction_first_cell}}
        \subcaptionbox{Первая итерация поиска внутренних ячеек}
            {\tikzset{every picture/.style={scale=0.45}}\subfile{tikz/markers_border_reconstruction_first_wave}}
        \subcaptionbox{Вторая итерация поиска внутренних ячеек}
            {\tikzset{every picture/.style={scale=0.45}}\subfile{tikz/markers_border_reconstruction_second_wave}}
        \caption{Нахождение внутренних ячеек в~расчётной области}
        \label{fig:markers_borders_reconstruction_waves}
    \end{figure}


    Из этих трёх этапов второй является наиболее сложным, так как даже при правильном выделении граничных ячеек остаётся
    открытым вопрос, каким образом найти внутренние ячейки. В~данной работе рассматривается вопрос выделения внутренних
    ячеек при условии односвязности исходной области. Для нахождения внутренних ячеек используется волновой алгоритм
    (см. рис.~\ref{fig:markers_borders_reconstruction_waves}), для работы которого необходимо найти хотя бы одну
    внутреннюю ячейку. Нахождение первой ячейки, с~которой в~дальнейшем будет расходиться <<волна>> поиска, выполняется
    следующим образом. В~произвольной точке маркерной поверхности вычисляется нормаль, затем вычисляются координаты
    новой точки, отстоящей на~определённое расстояние $l$ от~исходной в~направлении, обратном нормали. После этого
    ячейка, содержащая точку с~найденными координатами, помечается как внутренняя. Этот метод работает достаточно хорошо
    при условии, что исходная геометрия является достаточно гладкой, т.е. не имеет очень острых углов. Выбор расстояния,
    на~которое требуется отойти вглубь тела, зависит от~особенностей геометрии, но, как показывает практика, в~качестве
    такого расстояния в~большинстве случаев можно использовать две длины диагонали ячейки.

    В~случаях сложной геометрии (например, когда тело имеет тонкую перемычку) или при неправильном подборе расстояния $l$
    предложенный метод будет неверно выделять внутренние ячейки: вместо них будут найдены ячейки, полностью лежащие вне
    расчётной области. Для решения этой проблемы предлагается следующих подход: если при выделении внутренних ячеек была
    достигнута граница расчётной сетки, то результаты поиска нужно инвертировать (см. рис.~\ref{fig:markers_border_inverse}):
    все внутренние ячейки пометить как внешние, а внешние~--- как внутренние, при этом граничные ячейки модифицировать
    не нужно.

    \begin{figure}[ht!]
        \centering
        \subcaptionbox{Ошибочное заполнение внешней области}
            {\tikzset{every picture/.style={scale=0.5}}\subfile{tikz/markers_border_reconstruction_wrong_fill}}
        \subcaptionbox{Результат инвертирования}
            {\tikzset{every picture/.style={scale=0.5}}\subfile{tikz/markers_border_reconstruction_fill}}
        \caption{Нахождение внутренних ячеек в~расчётной области}
        \label{fig:markers_border_inverse}
    \end{figure}

    Восстановленная таким способом граница расчётной сетки не всегда может быть использована для вычислений. Вследствие
    вычислительных неточностей на~восстановленной границе могут быть лишние или, наоборот, недостающие граничные ячейки.
    Расчёт на~такой сетке будет заметно менее точным: неровная граница сетки будет порождать множество <<паразитных>>
    отражённых волн, которые будут <<смазывать>> общую волновую картину. Для предотвращения нежелательных эффектов
    такого рода проводится процедура уточнения границы: в~зависимости от~конкретной ситуации добавляются или
    удаляются граничные ячейки (см. рис.~\ref{fig:markers-refine-scheme}). Процесс уточнения сетки состоит из~двух
    проходов:

    \begin{enumerate}
        \item Для всех внутренних (граничных в~том числе) ячеек расчётной сетки вычисляется количество $n$ соседних
              (т.е. имеющих с~текущей общую грань) ячеек, которые помечены как внешние. Если $n\geq 3$, то
              текущая ячейка помечается как внешняя. Процесс продолжается до~тех пор, пока на~каждой итерации происходит
              изменение статуса хотя бы одной ячейки.
        \item Для всех внешних ячеек расчётной сетки вычисляется количество $n$ соседних ячеек, которые помечены как
              внутренние. Если $n\geq 3$, то текущая ячейка помечается как внутренняя. Процесс продолжается до~тех пор,
              пока на~каждой итерации происходит изменение статуса хотя бы одной ячейки.
    \end{enumerate}

    Очевидно, что такой алгоритм завершит своё выполнение за~конечное число итераций.

    Полученная после уточнения сетка  (см. рис.~\ref{fig:markers-refine-compare}) вполне может быть использована для
    расчётов, так как более не содержит неровностей на~границе.

    \begin{figure}[ht!]
        \centering
        \subcaptionbox{До уточнения}
            {\tikzset{every picture/.style={scale=0.5}}\subfile{tikz/markers_border_reconstruction_refine_before}}
        \subcaptionbox{После уточнения}
            {\tikzset{every picture/.style={scale=0.5}}\subfile{tikz/markers_border_reconstruction_refine_after}}
        \caption{Процесс уточнения границы расчётной сетки в~двумерном случае (схема)}
        \label{fig:markers-refine-scheme}
    \end{figure}

    \twofigsH
        {fig:markers-refine-compare}{Процесс уточнения границы расчётной сетки (реальная геометрия)}
        {png/markers/before-refinement.png}{До уточнения}
        {png/markers/after-refinement.png}{После уточнения}

    Примеры границ, восстановленных описанным выше методом, приведены на~рис.
    \ref{fig:markered-mesh-torus}--\ref{fig:markered-mesh-body}.

    \figfullH{fig:markered-mesh-torus}{png/markers/torus.png}
            {Пример выделения граничных и внутренних узлов на~расчётной сетке (тор)}

    \figfullH{fig:markered-mesh-body}{png/markers/body.png}
            {Пример выделения граничных и внутренних узлов на~расчётной сетке (торс человека)}

    Как было сказано выше, для корректной работы сеточно-характерис\-ти\-чес\-ко\-го метода при расчёте граничных узлов требуется
    иметь информацию о~локальной структуре границы, а именно необходима возможность получить нормаль в~произвольной точке
    границы. Можно выделить два способа получения нормали:

    \begin{itemize}
        \item вычисление нормали по~поверхностной маркерной сетке;
        \item вычисление нормали по~расчётной структурированной сетке.
    \end{itemize}

    \begin{figure}[t!]
        \centering
        \subcaptionbox{\label{fig:markers-cube-normals-problem}Куб}{
            \tikzset{every picture/.style={scale=0.5}}\subfile{tikz/markers_cube_normals}}
        \subcaptionbox{\label{fig:markers-plane-normals-problem}Поверхность раздела двух сред}{
            \tikzset{every picture/.style={scale=0.5}}\subfile{tikz/markers_plane_normals}}
        \caption{Способы вычисления нормали. Красным цветом изображены нормали, вычисленные по~маркерной
                 поверхности, чёрным~--- по~расчётной сетке}
        \label{fig:markers-normals-problem}
    \end{figure}

    \newpage
    Ни один из~этих способов не может быть применён для вычисления нормали в~произвольной ситуации. Так, первый метод не
    всегда будет давать корректные результаты при вычислении нормали в~угловых (и близлежащих узлах) тела в~форме куба
    (см. рис.~\ref{fig:markers-cube-normals-problem}). Из рисунка видно, что вычисление нормали по~маркерной поверхности
    неоднозначно, поэтому в~данном случае удобнее было бы применить метод вычисления нормали по~расчётной сетке. С
    другой стороны, вычисление нормали по~расчётной сетке будет некорректно работать в~случае, рассмотренном на~рис.
    \ref{fig:markers-plane-normals-problem}. Как видно из~рисунка, в~данном случае, наоборот, именно нормаль,
    вычисленная по~маркерной сетке, даёт правильный результат. В~этом случае даже небольшое отклонение от~истинной
    нормали приведёт к~тому, что волна, фронт которой движется в~направлении, перпендикулярном плоскости раздела двух
    сред, отразится в~неправильном направлении. Поэтому в~данной работе используется алгоритм адаптивного вычисления
    нормали: для граничных точек вычисляются обе нормали (по поверхностной сетке и по~граничной), а затем, если угол
    между ними составляет более $30\degree$, то выбирается нормаль, вычисленная по~расчётной сетке, в~противном случае~---
    нормаль, вычисленная по~поверхностной сетке. Угол в~$30\degree$ был выбран эмпирически: при нём нормали
    вычислялись корректно для всех тестовых задач, использовавшихся для верификации алгоритмов. Очевидно, что при
    использовании описанного метода для решения других задач, возможно, возникнет потребность изменить подобранное
    значение.


    Другой немаловажный вопрос, возникающий во~время реализации метода маркеров, касается экстраполяции значений в
    новых точках, возникающих как следствие движения границы (см. рис.~\ref{fig:markers-interpolation}). Так, при
    движении границы в~какой-то момент в~качестве граничных будут отмечены те узлы, что на~прошлом шаге вообще не
    принимали участия в~расчёте. Чтобы расчёт был корректным, требуется каким-то образом вычислить значения искомых
    параметров в~новых узлах.

    \begin{figure}[h!]
        \centering
        \subcaptionbox{Положение границы и вектор скорости в~точке}{
            \tikzset{every picture/.style={scale=0.5}}\subfile{tikz/markers_border_interpolation_before}}
        \subcaptionbox{Новое положение границы и новые расчётные узлы}{
            \tikzset{every picture/.style={scale=0.5}}\subfile{tikz/markers_border_interpolation_after}}
        \caption{Появление новых узлов на~расчётной сетке}
        \label{fig:markers-interpolation}
    \end{figure}

    Следует отметить, что в~силу достаточно сложных полей скоростей, возникающих во~время расчёта, весьма затруднительно
    определить, движение какого именно узла расчётной сетки привело к~появлению новых граничных узлов. Поэтому в~данной
    работе предлагается следующий механизм экстраполяции значений в~новых узлах:

    \begin{enumerate}
        \item Последовательно перебираются расчётные узлы внутри сферы определённого радиуса (1--5 диагоналей ячейки
              расчётной сетки в~зависимости от~величины потенциальных деформаций) с~центром в~новом узле. Для каждой
              пары (старый узел, новый узел) вычисляется расстояние между точками $d$, а также угол $\phi$ между
              вектором, соединяющим старый и новый узел, и вектором скорости в~старом узле.
        \item Из всех найденных пар выбираются пары с~наименьшим расстоянием $d$, а затем из~найденных пар выбирается
              единственная~--- с~наименьшим углом $\phi$.
        \item Значения из~найденного старого узла переносятся в~новый.
    \end{enumerate}

    Следующий важный момент касается вопроса движения маркерной границы. Самый простой и очевидный способ изменения
    положения границы заключается в~следующем: для каждого узла маркерной поверхности интерполируется значение вектора
    скорости в~этой точке $\bar v$, затем этот узел смещается на~расстояние $\abs{\bar v} \cdot dt$ в~направлении
    вектора $\bar v$. Такой подход имеет весьма неприятную особенность: при колебательных движениях границы постоянно
    меняются граничные ячейки сетки, что приводит к~необходимости интерполяции значений в~новых узлах и ведёт к~потере
    точности. Поэтому предлагается следующий алгоритм движения маркерной сетки:

    \begin{enumerate}
        \item Вводится параметр $d_{min}$~--- минимальное смещение, по~достижении которого имеет смысл сдвигать границу.
        \item После очередного расчётного шага для каждого маркерного узла высчитывается смещение:
              $\bar d^{n+1}_i=\bar d^{n}_{i}+\bar v \cdot dt$. Если $\abs{\bar d^{n+1}_i} \geq d_{min}$, то узел
              маркерной сетки сдвигается на~расстояние $\abs{\bar d^{n+1}_i}$ в~направлении вектора $\bar d^{n+1}_i$, а
              смещение полагается равным нулю: $\bar d^{n+1}_i = \bar 0$.
    \end{enumerate}

    Такой подход в~сочетании с~алгоритмом уточнения границы позволяет существенно снизить количество <<паразитных>> волн и,
    соответственно, повысить качество численного расчёта.

    Как было сказано ранее, описываемый метод предполагает использование неподвижной эйлеровой сетки для проведения
    вычислений. При расчёте задач высокоскоростного соударения это может существенно затруднить вычисления, так как
    потребует очень большой сетки, которая сможет покрыть всю область, в~которой будут находиться взаимодействующие
    тела. Например, на~рис.~\ref{fig:markers-ricochet} изображена примерная траектория движения шарика при рикошете от
    преграды. Как видно из~рисунка, в~каждый момент времени используется лишь небольшая часть сетки, остальная же часть~---
    по~сути, впустую расходует оперативную память.

    \begin{figure}[h!]
        \centering
        \tikzset{every picture/.style={scale=0.4}}\subfile{tikz/markers_ricochet}
        \caption{Схематическая траектория движения шарика при рикошете. Красным отмечена части сетки, действительно
            необходимые для выполнения расчёта}
        \label{fig:markers-ricochet}
    \end{figure}

    Для того, чтобы решить эту проблему, предлагается следующий подход (см. рис.~\ref{fig:markers-adaptive-grid}):

    \begin{enumerate}
        \item После каждого расчётного шага вычисляется размер сетки, действительно необходимой для выполнения расчёта.
        \item Выбирается опорный элемент на~старой сетке, к~которому будет <<привязана>> новая сетка.
        \item Строится новая сетка, на~которую переносятся значения со~старой. Размеры ячеек новой сетки совпадают с
              таковыми для старой.
    \end{enumerate}

    \begin{figure}[h!]
        \centering
        \tikzset{every picture/.style={scale=0.5}}\subfile{tikz/markers_adaptive_grid}
        \caption{Перестроение сетки между шагами. Пунктиром показано положение сетки на~прошлом временном слое, сплошной
                 линией~--- на~текущем}
        \label{fig:markers-adaptive-grid}
    \end{figure}

    Благодаря тому, что две сетки имеют одинаковые размеры ячеек и содержат общую опорную точку, перенос значений со
    старой сетки на~новую является по~сути копированием данных из~одних ячеек памяти в~другие. Важно отметить, что при
    этом не происходит интерполяции, поэтому предложенный метод совершенно никак не сказывается на~точности, но
    позволяет сэкономить существенное количество памяти при расчёте задач с~большими перемещениями взаимодействующих
    тел. Сэкономленная таким образом память может быть использована для построения более мелких сеток.


    Предложенный метод маркеров позволяет численно решать задачи, в~которых взаимодействующие тела деформируются.
    Примеры расчётов приведены далее на~рис.~\ref{fig:markers_splinter_1}--\ref{fig:markers_sphere_1}.

    \twofigs{fig:markers_splinter_1}{Столкновение шарика с~пластиной}
        {png/markers/splinter/image1.png}{Начало расчёта}
        {png/markers/splinter/image21.png}{20-й шаг}

    \twofigs{fig:markers_splinter_2}{Столкновение шарика с~пластиной}
        {png/markers/splinter/image41.png}{40-й шаг}
        {png/markers/splinter/image61.png}{60-й шаг}

    \twofigs{fig:markers_splinter_3}{Столкновение шарика с~пластиной}
        {png/markers/splinter/image81.png}{80-й шаг}
        {png/markers/splinter/image101.png}{100-й шаг}

    \twofigs{fig:markers_splinter_4}{Столкновение шарика с~пластиной}
        {png/markers/splinter/image121.png}{120-й шаг}
        {png/markers/splinter/image141.png}{140-й шаг}

    \twofigs{fig:markers_sphere_1}{Деформация шарика при  ударе об~пластину}
        {png/markers/sphere/image-0000.png}{Начало расчёта}
        {png/markers/sphere/image-0020.png}{20-й шаг}

    \twofigs{fig:markers_sphere_2}{Деформация шарика при  ударе об~пластину}
        {png/markers/sphere/image-0040.png}{40-й шаг}
        {png/markers/sphere/image-0060.png}{60-й шаг}

    \twofigs{fig:markers_sphere_3}{Деформация шарика при  ударе об~пластину}
        {png/markers/sphere/image-0080.png}{80-й шаг}
        {png/markers/sphere/image-0100.png}{100-й шаг}

\end{document}
