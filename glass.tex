\documentclass[thesis.tex]{subfiles}

\begin{document}

\FloatBarrier

\subsection{Разрушение в~стекле под действием лазерного излучения}

\subsubsection{Постановка}

Рассматривается задача о~воздействии на~стеклянную мишень пучка лазерного излучения. Характеристики стекла приведены
в~табл.~\ref{таб:свойства-стекла}. Для моделирования воздействия используется задание эквивалентного (осреднённого)
импульса при помощи начального и граничного условий. Как показывает эксперимент \cite{demidov2015new}, при коротком
воздействии излучения деформации наблюдаются только в~непосредственной близости от~точки контакта пучка и преграды.
В~такой ситуации тепловые эффекты не успевают проявиться в~слое конечной толщины, поэтому моделирование исходного
воздействия при помощи осреднённого импульса является физически обоснованным.

\figfull{fig:statement}{png/glass/glass-laser-pressure.png}{Постановка задачи}

Образец, на~который осуществляется воздействие, представляет собой прямоугольный параллелепипед размером 70x70x20~мм.
В качестве эквивалентного импульса в~центре мишени в~круге радиусом 5~мм задан импульс давления в~8 ГПа длительностью
5~мкс (см.~рис.~\ref{fig:statement}).

\begin{table}[ht]
    \centering
    \begin{tabular}{| l | l |}
        \hline
        \textbf{Тип материала} & Изотропный \\
        \hline
        \textbf{Коэффициент Ламэ $\lambda$} & 22.5 ГПа \\
        \hline
        \textbf{Коэффициент Ламэ $\mu$} & 28.7 ГПа \\
        \hline
        \textbf{Плотность} & 2500 $\textnormal{кг/см}^3$ \\
        \hline
        \textbf{Предел прочности} & 1.5 ГПа \\
        \hline
    \end{tabular}
    \caption{Параметры стекла}
    \label{таб:свойства-стекла}
\end{table}

Для моделирования хрупкого разрушения в~стекле используется критерий наибольшего главного напряжения и однобереговая
модель трещин Майнчена-Сака.

Задача решается численно при помощи сеточно-характеристического метода на~неструктурированной
тетраэдральной сетке (см. рис.~\ref{fig:mesh}). Количество узлов в~сетке -- $\sim{12}$ миллионов.

\figfull{fig:mesh}{png/glass/glass-laser-mesh.png}{Расчётная сетка}

\subsubsection{Результаты расчёта}

На рис.~\ref{fig:res_0}--\ref{fig:res_150} представлены результаты расчёта. На верхней половине каждого рисунка
изображены области разрушения в~стекле, на~нижней~--- компонента $S_{zz}$ тензора напряжений в~тот же
момент времени.

На~рис.~\ref{fig:destr_1_1}--\ref{fig:destr_1_5} изображён процесс формирования разрушения в~объёме материала,
а на~рис.~\ref{fig:destr_2_1}--\ref{fig:destr_2_5}~--- процесс тыльного откола. Помимо области разрушения материала на
верхней части каждого рисунка изображено поле скоростей.

Шаг по~времени в~данном расчёте равен 28 нс.

\figfull{fig:res_0}{png/glass/glass-laser/image1.png}{Распространение возмущения, 0-й расчётный шаг}
\figfull{fig:res_30}{png/glass/glass-laser/image31.png}{Распространение возмущения, 30-й расчётный шаг}
\figfull{fig:res_60}{png/glass/glass-laser/image61.png}{Распространение возмущения, 60-й расчётный шаг}
\figfull{fig:res_90}{png/glass/glass-laser/image91.png}{Распространение возмущения, 90-й расчётный шаг}
\figfull{fig:res_120}{png/glass/glass-laser/image121.png}{Распространение возмущения, 120-й расчётный шаг}
\figfull{fig:res_150}{png/glass/glass-laser/image151.png}{Распространение возмущения, 150-й расчётный шаг}

\figfull{fig:destr_1_1}{png/glass/glass-laser-velocity-field-1/image1.png}{Формирование разрушения в~объёме материала}
\figfull{fig:destr_1_2}{png/glass/glass-laser-velocity-field-1/image11.png}{Формирование разрушения в~объёме материала}
\figfull{fig:destr_1_3}{png/glass/glass-laser-velocity-field-1/image21.png}{Формирование разрушения в~объёме материала}
\figfull{fig:destr_1_4}{png/glass/glass-laser-velocity-field-1/image31.png}{Формирование разрушения в~объёме материала}
\figfull{fig:destr_1_5}{png/glass/glass-laser-velocity-field-1/image41.png}{Формирование разрушения в~объёме материала}

\figfull{fig:destr_2_1}{png/glass/glass-laser-velocity-field-2/image1.png}{Формирование откольной тарелочки}
\figfull{fig:destr_2_2}{png/glass/glass-laser-velocity-field-2/image11.png}{Формирование откольной тарелочки}
\figfull{fig:destr_2_3}{png/glass/glass-laser-velocity-field-2/image21.png}{Формирование откольной тарелочки}
\figfull{fig:destr_2_4}{png/glass/glass-laser-velocity-field-2/image31.png}{Формирование откольной тарелочки}
\figfull{fig:destr_2_5}{png/glass/glass-laser-velocity-field-2/image41.png}{Формирование откольной тарелочки}

\FloatBarrier

\subsubsection{Сравнение с~экспериментом}

Результаты расчёта на~качественном уровне хорошо согласуются с~экспериментом \cite{demidov2015new}, особенно с~учётом
того, что начальное воздействие в~расчёте было задано при помощи эквивалентного импульса. На
рис.~\ref{рис:стекло-сравнение} представлены изображения разрушенной области, полученной по~результатам численного
моделирования, а также фотография образца, полученного после эксперимента.

\twofigs{рис:стекло-сравнение}
        {Сравнение итоговой разрушенной области, полученной при расчёте, с~результатами эксперимента}
        {png/glass/glass-laser-crack-3d.png}{Результаты расчёта}
        {png/glass/experiment.png}{Фотография образца}

Видно, что в~результате численного моделирования получены две характерные области разрушения в~образце: разрушенная
область в~форме <<цветка>> в~объёме и область тыльного откола. Характерные размеры разрушенных областей также совпадают в~пределах~20\%.
Область разрушения, вызванная непосредственным высокотемпературным воздействием, в~ходе численного
эксперимента не получена, так как моделировалось только хрупкое разрушение, вызванное ударной нагрузкой в~виде
осреднённого импульса.

Отдельно стоит отметить, что механизм возникновения разрушения в~объёме, полученного в~ходе численного решения задачи,
является принципиально трёхмерным. Явление тыльного откола возможно смоделировать даже в~одномерной постановке, но
получить разрушение в~объёме, вызванное взаимодействием сходящихся сферических волн, в~одномерной постановке невозможно.

\subsubsection[Зависимость размеров области разрушения\\от~прочности материала]{Зависимость размеров области разрушения от~прочности материала}

С целью выяснить, каким образом изменение параметров стекла влияет на~качественную картину разрушения, было проведено
несколько расчётов, в~которых прочность стекла завышалась. Результаты расчётов приведены на~рис. \ref{рис:стекло-прочность-1}
---\ref{рис:стекло-прочность-2}. Как видно из~результатов расчёта, увеличение прочности стекла ожидаемо
ведёт к~уменьшению итоговой разрушенной зоны. Отсюда можно сделать вывод, что для большего совпадения с~экспериментом
требуется задать более точные параметры материала, так как варьирование прочностных характеристик ведёт к~заметному
изменению размеров разрушенной области.

Заметим, что после того, верификации модели рядом экспериментов, с ее помощью можно будет определять прочность материалов.


\twofigs{рис:стекло-прочность-1}{Область разрушения при разных пределах прочности}
       {png/glass/laser-crack-stress-1-5.png}{1.5 ГПа}
       {png/glass/laser-crack-stress-2.png}{2.0 ГПа}
\twofigs{рис:стекло-прочность-2}{Область разрушения при разных пределах прочности}
       {png/glass/laser-crack-stress-2-5.png}{2.5 ГПа}
       {png/glass/laser-crack-stress-3.png}{3.0 ГПа}

\end{document}
