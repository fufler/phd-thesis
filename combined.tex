\documentclass[thesis.tex]{subfiles}

\begin{document}

\subsection{Комбинированный метод}

Рассмотренные ранее сеточно-характеристический метод и метод сглаженных частиц хорошо подходят для численного решения
определённых классов задач механики деформируемого твёрдого тела. При этом каждый из~методов, очевидно, не является
универсальным. Так, например, для решения задач с~большими деформациями области интегрирования при помощи сеточно-характеристического
метода требуется проделать весьма существенный объём работы для реализации адаптивного перестроения
расчётной сетки. Метод сглаженных частиц в~свою очередь обладает заметными недостатками в~части, касающейся расчёта
контактных и граничных условий. В~связи с~этим возникает естественная потребность реализовать комбинированный метод,
сочетающий достоинства двух описанных ранее подходов.

\subsubsection{Описание метода}

Основная идея комбинированного метода заключается в~том, что исходная область интегрирования разбивается на~несколько
частей (как минимум, на~две), вычисления в~каждой из~которых производятся наиболее подходящим численным методом. На
границе двух областей, расчёты в~которых проводятся разными численными методами, производится <<сшивка>> решений.
Математические основы комбинированного численного метода подробно рассмотрены в~статье \cite{петров2014схм}. В~этой же
статье доказано сохранение первого или второго порядка аппроксимации комбинированным методом в~зависимости от~того,
какой вид интерполяции используется в~области, рассчитываемой сеточно-характеристическим методом.

Как было сказано выше, в~простейшем случае для расчёта при помощи комбинированного метода область интегрирования делится
на две части (см.рис \ref{рис:spgcm}). Одна из~этих частей (на рис. слева) рассчитывается при помощи
сеточно-характеристического метода, другая~--- при помощи метода сглаженных частиц. Так как оба метода являются
локальными, то вычисления вдали от~границы раздела областей производятся стандартным для соответствующего метода
способом.

При вычислении граничных узлов сеточно-характеристическим методом интерполяция значений инвариантов Римана на~выводящих
характеристиках осуществляется при помощи того же ядра сглаживания, которое используется для расчёта в~области частиц.
Такой подход обеспечивает математически корректное прохождение возмущений из~области частиц в~область сеток.

Для учёта возмущений, движущихся в~обратном направлении, используется слой фиктивных частиц: эти частицы находятся в
области сеток, а значения в~них восстанавливаются при помощи интерполяционных функций, используемых
сеточно-характеристическим методом. После того, как значения в~этих частицах восстановлены, они могут быть использованы
для выполнения следующего расчётного шага в~области частиц.

\begin{figure}[th!]
    \begin{center}
        \tikzset{every picture/.style={scale=1.25}}
        \subfile{tikz/spgcm}
    \end{center}
    \caption{Разбиение области интегрирования для расчёта комбинированным методом. Область сеток~--- слева, область
             частиц~--- справа. Зелёным отмечены фиктивные частицы}
    \label{рис:spgcm}
\end{figure}

\subsubsection{Программная реализация}

С точкие зрения программной реализации комбинированный метод представляет из~себя набор взаимосвязанных компонентов,
призванных объединить две соответствующих реализации численных методов (см. рис.~\ref{рис:комбинированный-бинарник}).

\begin{figure}[th!]
    \begin{center}
        \tikzset{every picture/.style={scale=1.25}}
        \subfile{tikz/spgcm_binary}
    \end{center}
    \caption{Принципиальная схема устройства программы с~комбинированным численным методом}
    \label{рис:комбинированный-бинарник}
\end{figure}

В программном комплексе, в~котором реализован комбинированный численный метод, можно выделить две основные части:
непосредственная реализация метода (SPGCM) и набор тестов (Tests). SPGCM представляет собой набор модулей по~интеграции
двух численных методов, а именно:
\begin{itemize}
    \item GCM~--- реализация сеточно-характеристического метода на~тетраэдральных сетках.
    \item SPH~--- реализация метода сглаженных частиц.
    \item Dispatcher~--- модуль, отвечающий за~распределение участков области интегрирования между двумя численными
    методами, а также между вычислительными узлами при многопроцессорном расчёте на~кластере.
    \item CombinedModel~--- модуль, непосредственно обеспечивающий сшивку численных методов и управление расчётом. Именно
    в~нём реализована основная логика, позволяющая проводить комбинированный расчёт.
    \item Snapshotter~--- модуль, предоставляющий возможность сохранять результаты расчётов в~едином формате, что
    необходимо для возможности визуализации полученных в~ходе расчёта данных.
\end{itemize}

Модуль тестирования результатов (Tests) служит для верификации получаемых в~результате расчёта данных, а также
обеспечивает возможность использования механизмов непрерывной интеграции в~ходе процесса разработки. Модуль состоит из
следующих компонентов:
\begin{itemize}
    \item Analytics~--- набор постановок задач и соответствующих им аналитических решений.
    \item Unit tests~---  набор модульных тестов, позволяющих проводить проверку корректности реализации
    отдельных частей программы в~изолированном окружении.
    \item Func tests~--- набор функциональных тестов, сверяющих результаты численного расчёта тестовых постановок
    и соответствующих им аналитических решений.
    \item paint.py~--- утилита для визуализации результатов моделирования, а также результатов выполнения
    функциональных тестов.
\end{itemize}

Модуль CombinedModel отвечает за~выполнение расчёта комбинированным методом. Строго говоря, его главной задачей является
последовательное вычисление значений в~областях интегрирования соответствующими численными методами и выполнение
необходимых операций для корректной сшивки методов. Последовательность действий, выполняемых эти методом, такова:
\begin{enumerate}
    \item Вычисление необходимых значений в~фиктивных частицах с~использованием значений, полученных на~предыдущем
    расчётном шаге.
    \item Определение максимального допустимого шага по~времени $\tau_{max}=$ \\
          $=\max (\tau_{gcm}, \tau_{sph})$, где
    $\tau_{gcm}$ и $\tau_{sph}$~--- максимально допустимые шаги по~времени для области сеток и частиц соответственно.
    \item Решение системы уравнений \eqref{уравнения_движения_в_частицах} в~области частиц.
    \item Вычисление значений на~новом временном слое в~области сеток.
    \item Интегрирование системы уравнений \eqref{уравнения_мдтт} с~использованием значений, полученных на~шаге 3. Этот
    шаг необходим, так в~отличии от~области сеток, где фактически хранятся значения на~прошлом и текущем временных слоях,
    в~области частиц хранятся значения только для текущего временного слоя.
    \item Интерполяция на~визуализационную сетку и сохранение соответствующих результатов расчёта на~файловой системе.
\end{enumerate}

\subsubsection{Верификация метода}
\label{раздел:верификация-комбинированного-метода}

Для верификации реализованного комбинированного численного метода был решён ряд модельных задач, имеющих
точное аналитическое решение.  Во всех постановках задаётся возмущение в~некоторой области изначально неподвижного
ненапряжённого тела. Для каждой постановки на~соответствующих графиках приведены несколько кривых: точное решение
(analytic), результат расчёта только сеточно-характеристическим методом (gcm), результат расчёта только методом
сглаженных частиц (sph) и результат расчёта комбинированным методом (gcm-sph и sph-gcm).

Как видно из~рис.~\ref{рис:spgcm-onestep-normal-x}--\ref{рис:spgcm-sine-tangent-z}, результаты расчётов
согласуются с~аналитическим решением при учёте погрешностей, неизбежно возникающих при численном моделировании.

\paragraph{Задача о~распаде разрыва}

Рассматривается железное тело в~форме куба с~гранью 4~м, первоначально находящееся в~первом координатном углу трёхмерной
декартовой системы координат, причем три ребра куба лежат на~координатных осях. Для этого тела рассматриваются две
постановки: распад разрыва по~скорости в~случае продольной и поперечной волн.

В случае продольной (поперечной) волны части тела, находящейся условно слева от~плоскости, проходящей через центр куба и имеющей
нормаль (0.306, -0.153, 0.94), придается начальная скорость 1~м/с (0.75~м/с) вдоль нормали (перпендикулярно к~нормали).

Границей раздела областей расчёта разными численными методами является плоскость $z=2.5$.

Результаты расчёта представлены на~рис.~\ref {рис:spgcm-onestep-normal-x}--\ref {рис:spgcm-onestep-tangent-z}.

\figcommon{рис:spgcm-onestep-normal-x}{png/spgcm/OneStep-normal/pics/1D-V.X-all.png}
          {Эволюция со~временем скорости движения среды по~оси X для случая продольной волны}{0.75\linewidth}{ht}
\figcommon{рис:spgcm-onestep-normal-y}{png/spgcm/OneStep-normal/pics/1D-V.Y-all.png}
          {Эволюция со~временем скорости движения среды по~оси Y для случая продольной волны}{0.75\linewidth}{ht}
\figcommon{рис:spgcm-onestep-normal-z}{png/spgcm/OneStep-normal/pics/1D-V.Z-all.png}
          {Эволюция со~временем скорости движения среды по~оси Z для случая продольной волны}{0.75\linewidth}{ht}

\figcommon{рис:spgcm-onestep-tangent-x}{png/spgcm/OneStep-tangent/pics/1D-V.X-all.png}
          {Эволюция со~временем скорости движения среды по~оси X для случая поперечной волны}{0.75\linewidth}{ht}
\figcommon{рис:spgcm-onestep-tangent-y}{png/spgcm/OneStep-tangent/pics/1D-V.Y-all.png}
          {Эволюция со~временем скорости движения среды по~оси Y для случая поперечной волны}{0.75\linewidth}{ht}
\figcommon{рис:spgcm-onestep-tangent-z}{png/spgcm/OneStep-tangent/pics/1D-V.Z-all.png}
          {Эволюция со~временем скорости движения среды по~оси Z для случая поперечной волны}{0.75\linewidth}{ht}

\paragraph{Распад двух разрывов}
Постановка совпадает с~описанной в~предыдущем параграфе, но имеет два отличия:
\begin{enumerate}
    \item Тело имеет форму параллелепипеда с~длиной ребра 6~м воль оси $Z$, длины остальных ребер составляют 4~м.
    \item Начальная скорость в~данном случае придается области тела между двумя плоскостями, характеризующимися той же
    нормалью и находящимися по~разные стороны от~центра тела на~расстоянии 1~м от~него.
\end{enumerate}

Границей раздела областей расчёта разными численными методами является плоскость $z=4.5$.

Результаты расчёта представлены на~рис.~\ref {рис:spgcm-meander-normal-x}--\ref {рис:spgcm-meander-tangent-z}.

\figcommon{рис:spgcm-meander-normal-x}{png/spgcm/Meander-normal/pics/1D-V.X-all.png}
          {Эволюция со~временем скорости движения среды по~оси X для случая продольной волны}{0.75\linewidth}{ht}
\figcommon{рис:spgcm-meander-normal-y}{png/spgcm/Meander-normal/pics/1D-V.Y-all.png}
          {Эволюция со~временем скорости движения среды по~оси Y для случая продольной волны}{0.75\linewidth}{ht}
\figcommon{рис:spgcm-meander-normal-z}{png/spgcm/Meander-normal/pics/1D-V.Z-all.png}
          {Эволюция со~временем скорости движения среды по~оси Z для случая продольной волны}{0.75\linewidth}{ht}

\figcommon{рис:spgcm-meander-tangent-x}{png/spgcm/Meander-tangent/pics/1D-V.X-all.png}
          {Эволюция со~временем скорости движения среды по~оси X для случая поперечной волны}{0.75\linewidth}{ht}
\figcommon{рис:spgcm-meander-tangent-y}{png/spgcm/Meander-tangent/pics/1D-V.Y-all.png}
          {Эволюция со~временем скорости движения среды по~оси Y для случая поперечной волны}{0.75\linewidth}{ht}
\figcommon{рис:spgcm-meander-tangent-z}{png/spgcm/Meander-tangent/pics/1D-V.Z-all.png}
          {Эволюция со~временем скорости движения среды по~оси Z для случая поперечной волны}{0.75\linewidth}{ht}

\paragraph{Распространение синусоидальных и треугольных возмущений}

Постановка задачи полностью совпадает с~постановкой предыдущей задачи, отличается лишь форма первоначального возмущения.
Рассматривается развитие первоначальных возмущений двух форм: треугольной и синусоидальной. Разделение
на подобласти совпадает с~использованным в~предыдущей задаче.

\figcommon{рис:spgcm-sine-normal-x}{png/spgcm/Sine-normal/pics/1D-V.X-all.png}
          {Эволюция со~временем скорости движения среды по~оси X для случая продольной волны}{0.75\linewidth}{ht}
\figcommon{рис:spgcm-sine-normal-y}{png/spgcm/Sine-normal/pics/1D-V.Y-all.png}
          {Эволюция со~временем скорости движения среды по~оси Y для случая продольной волны}{0.75\linewidth}{ht}
\figcommon{рис:spgcm-sine-normal-z}{png/spgcm/Sine-normal/pics/1D-V.Z-all.png}
          {Эволюция со~временем скорости движения среды по~оси Z для случая продольной волны}{0.75\linewidth}{ht}

\figcommon{рис:spgcm-sine-tangent-x}{png/spgcm/Sine-tangent/pics/1D-V.X-all.png}
          {Эволюция со~временем скорости движения среды по~оси X для случая поперечной волны}{0.75\linewidth}{ht}
\figcommon{рис:spgcm-sine-tangent-y}{png/spgcm/Sine-tangent/pics/1D-V.Y-all.png}
          {Эволюция со~временем скорости движения среды по~оси Y для случая поперечной волны}{0.75\linewidth}{ht}
\figcommon{рис:spgcm-sine-tangent-z}{png/spgcm/Sine-tangent/pics/1D-V.Z-all.png}
          {Эволюция со~временем скорости движения среды по~оси Z для случая поперечной волны}{0.75\linewidth}{ht}

\FloatBarrier

После верификации комбинированного метода при помощи набора тестовых постановок, для которых существует точное
аналитическое решение, было рассмотрено несколько постановок модельных задач о~пробивании. В~качестве цели ставилось
сравнение на~качественном уровне с~экспериментом и количественное сравнение с~результатами, полученными при
моделировании коммерческим продуктом LS-DYNA.

\paragraph{Пробивание пластины ударником в~форме параллелепипеда}
Решается численно задача о взаимодействии элемента с~преградой при ударе гранью с~площадью миделя равной $\sim 0.57
\textnormal{см}^2$. Удар производится по~нормали со~скоростью 1200~м/с, толщина пластины составляет 6.2~мм. Ударник
изготовлен из~материала <<сталь 20>>, преграда~--- <<сталь 09Г2С>>. Параметры материалов  приведены в~табл.~ \ref{таб:параметры-материалов-в-задаче-пробивания}.

Согласно эксперименту при таких начальных данных сквозного пробивания преграды не происходит.

Результаты представлены на~рис.~\ref{рис:пробивание-параллелепипедом}.

\begin{figure}[h!]
    \begin{center}
        \begin{gnuplot}
            set datafile separator ","
            set terminal epslatex color size 12cm, 8cm 14
            set grid ytics xtics
            set grid
            set xlabel "t, мкс"
            set ylabel "$v_x$, м/с"
            set xrange [0:40]
            set yrange [0:1200]
            plot "csv/normal-cube-spgcm.csv" title "SPGCM", \
                "csv/normal-cube-dyna.csv" title "LS-DYNA" with points pointtype 7
        \end{gnuplot}
    \end{center}
    \caption{Сравнение результатов численного решения задачи о~пробивании пластины параллелепипедом при
             нормальном ударе с~помощью комбинированного метода и LS-DYNA}
    \label{рис:пробивание-параллелепипедом}
\end{figure}

\paragraph{Пробивание пластины ударником в~форме шара}

Опыт в~основном аналогичен предыдущему, с~отличием в~толщине преграды (6.0мм вместо 6.2мм), начальной скорости ударника
(925м/с вместо 1200м/с) и форме ударника(шар вместо параллелепипеда).

Согласно эксперименту при данной скорости удара происходит сквозное пробитие преграды ударником.

Результаты представлены на~рис.~\ref{рис:пробивание-шаром}.

\begin{figure}[h!]
    \begin{center}
        \begin{gnuplot}
            set datafile separator ","
            set terminal epslatex color size 12cm, 8cm 14
            set grid ytics xtics
            set grid
            set xlabel "t, мкс"
            set ylabel "$v_x$, м/с"
            set xrange [0:40]
            set yrange [0:1200]
            plot "csv/normal-sphere-spgcm.csv" title "SPGCM", \
                "csv/normal-sphere-dyna.csv" title "LS-DYNA" with points pointtype 7
        \end{gnuplot}
    \end{center}
    \caption{Сравнение результатов численного решения задачи о~пробивании пластины шаром при
             нормальном ударе с~помощью комбинированного метода и LS-DYNA}
    \label{рис:пробивание-шаром}
\end{figure}

\paragraph{Пробивание пластины ударником в~форме шара при ударе под углом 30\degree}

Так же, как и в~предыдущем опыте, ударник шаровидной формы врезается в~преграду толщиной 6.0мм. Угол между вектором
скорости и нормалью к~поверхности пластины составляет 30\degree. Начальная скорость по~модулю равна 1200м/с.

Согласно эксперименту при данной начальной скорости обеспечивается сквозное пробивание преграды.

Результаты представлены на~рис.~\ref{рис:пробивание-шаром-под-углом}.

\begin{figure}[h!]
    \begin{center}
        \begin{gnuplot}
            set datafile separator ","
            set terminal epslatex color size 12cm, 8cm 14
            set grid ytics xtics
            set grid
            set xlabel "t, мкс"
            set ylabel "$v_x$, м/с"
            set xrange [0:40]
            set yrange [0:1200]
            plot "csv/angled-sphere-spgcm.csv" title "SPGCM", \
                "csv/angled-sphere-dyna.csv" title "LS-DYNA" with points pointtype 7
        \end{gnuplot}
    \end{center}
    \caption{Сравнение результатов численного решения задачи о~пробивании пластины шаром при
             косом ударе с~помощью комбинированного метода и LS-DYNA}
    \label{рис:пробивание-шаром-под-углом}
\end{figure}

\newpage
Исходя из~результатов решения модельных задач (рис.~\ref{рис:пробивание-параллелепипедом}--\ref{рис:пробивание-шаром-под-углом})
можно говорить о~том, что имеется совпадение с~экспериментом на~качественном
уровне, а также имеется количественное совпадение с~результатами, полученными при помощи LS-DYNA.

\end{document}
