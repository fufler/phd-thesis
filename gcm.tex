\documentclass[thesis.tex]{subfiles}

\begin{document}

\subsection{Сеточно-характеристический метод}

Сеточно-характеристический \cite{магомедов1988схм} метод используется для решения системы уравнений механики деформируемого
твёрдого тела, записанных в~матричной форме \eqref{матричные_уравнения_мдтт}:
\[
    \PDt{\vec u} + \mathbf A_x \PD{\vec u}{x} +
                   \mathbf A_y \PD{\vec u}{y} +
                   \mathbf A_z \PD{\vec u}{z} = \vec f.
\]

Для решения этого уравнения используется метод расщепления по~направлениям \cite{федоренко528введение}, суть которого
заключается в~замене исходного уравнения следующей системой:
\begin{align}
    \frac{\partial}{\partial t}\vec u+\mathbf{A}_x \frac{\partial}{\partial x}\vec u = 0, \label{расщепление_1} \\
    \frac{\partial}{\partial t}\vec u+\mathbf{A}_y \frac{\partial}{\partial y}\vec u = 0, \label{расщепление_2} \\
    \frac{\partial}{\partial t}\vec u+\mathbf{A}_z \frac{\partial}{\partial z}\vec u = 0, \label{расщепление_3} \\
    \frac{\partial}{\partial t}\vec u = \vec f. \label{расщепление_4}
\end{align}

Полученная система решается последовательно, т.е. уравнения решаются в~порядке \eqref{расщепление_4}, \eqref{расщепление_3},
\eqref{расщепление_2}, \eqref{расщепление_1}, при этом на~каждом шаге учитываются решения, полученные ранее.

Строго говоря, этот метод решения не является единственным возможным \cite{челноков2005диссертация}. Существуют другие
способы сведения исходного матричного уравнения к~системе одномерных уравнений, но ввиду более высокой вычислительной
сложности в~данной работе они не используются. Для симметризации схемы в~данной работе используется метод выбора
произвольного ортонормированного базиса в~каждой точке расчётной сетки, также предложенный в~\cite{челноков2005диссертация}.


\subsubsection{Решение одномерной задачи}
После того, как было выполнено расщепление по~направлениям, исходная задача по~сути сводится к~задаче решения одномерного
уравнения переноса \eqref{уравнение_переноса}: \newpage
\begin{equation}
    \PDt{\vec u} + \mathbf A \PDx{\vec u} = 0,
    \label{уравнение_переноса}
\end{equation}
где $\vec u = (v_x,v_y,v_z,\Sxx,\Sxy,\Sxz,\Syy,\Syz,\Szz)^T$~--- вектор неизвестных.

Отличительной особенностью уравнений механики деформируемого твёрдого тела является их гиперболичность~--- это означает,
что матрицы $\mathbf A_i$ в~\eqref{матричные_уравнения_мдтт} имеют полный набор собственных векторов.

Если $\vec \omega_i$~--- собственные строки матрицы $\mathbf A$, отвечающие собственным значениям $\lambda_i$, то
\[
    \vec\omega_i \mathbf A = \lambda_i\vec \omega_i.
\]
Или, что то же самое,
\[
    \begin{pmatrix}
        \omega_1^1 & \omega_2^1 & \dots & \omega_n^1 \\
        \omega_1^2 & \omega_2^2 & \dots & \omega_n^2 \\
        \dots      & \dots      & \dots & \dots      \\
        \omega_1^n & \omega_2^n & \dots & \omega_n^n
    \end{pmatrix} \mathbf A  = \begin{pmatrix}
        \lambda_1 & 0         & \dots & 0       \\
        0         & \lambda_2 & \dots & 0      \\
        \dots     & \dots     & \dots & \dots  \\
        0         & 0         & \dots & \lambda_n
    \end{pmatrix}\begin{pmatrix}
        \omega_1^1 & \omega_2^1 & \dots & \omega_n^1 \\
        \omega_1^2 & \omega_2^2 & \dots & \omega_n^2 \\
        \dots      & \dots      & \dots & \dots      \\
        \omega_1^n & \omega_2^n & \dots & \omega_n^n
    \end{pmatrix},
\]
где $\vec \omega_i=(\omega_1^i,\omega_2^i,\dots,\omega_n^i)$. В~силу того, что все собственные векторы матрицы $\mathbf A$
линейно независимы, из~предыдущего равенства следует
\begin{equation}
    \mathbf A =  \bm \Omega^{-1} \bm \Lambda \bm \Omega,
    \label{разложение_матрицы}
\end{equation}
где $\bm \Lambda=diag(\lambda_i)$, матрица $\bm \Omega$ составлена из~собственных строк матрицы $\mathbf A$.

Подставляя \ref{разложение_матрицы} в~\ref{уравнение_переноса} получаем
\[
    \PDt{\vec u} + \bm \Omega^{-1} \bm \Lambda \bm \Omega\PDx{\vec u} = 0.
\]

Домножая обе части равенства слева на~невырожденную матрицу $\bm \Omega$, получаем уравнение
\[
    \PDt{\bm \Omega\vec u} + \bm \Lambda \PDx{\bm \Omega \vec u} = 0,
\]
которое после замены переменных $\vec w=\bm \Omega\vec u$ сводится к~набору независимых одномерных уравнений переноса,
в которых неизвестная величина является скаляром:
\begin{equation}
    \PDt{w_i} + \lambda_i\PDx{w_i} = 0.
    \label{уравнение_переноса_для_инвариантов_римана}
\end{equation}

Введённые переменные $w_i$~--- не что иное, как инварианты Римана, которые сохраняют своё значение вдоль характеристик
уравнения. Из набора уравнений \ref{уравнение_переноса_для_инвариантов_римана} можно сделать вывод, что решение
уравнения \ref{уравнение_переноса} представляется в~виде суммы плоских волн, каждая из~которых распространяется со
скоростью, равной соответствующему собственному числу матрицы $\mathbf A$.

\subsubsection{Решение уравнения переноса}
Как известно, решением уравнения переноса
\[
    \PDt{u}+\lambda\PDx{u}=0
\]
при $\lambda=const$ является любая функция $u(x,t)=f(x-\lambda t)$, удовлетворяющая начальным и граничным условиям.
Значения этой функции постоянны вдоль характеристики $x-\lambda t=const$, т.е.
\[
    u(x,t+\Delta t) = f(x-\lambda(t+\Delta t)) = f((x-\lambda\Delta t) - \lambda t) = u(x-\lambda\Delta t, t).
\]

Таким образом, этот способ достаточно наглядно можно изобразить графически (см. рис.~\ref{рис:схм}).
\begin{figure}[h]
    \begin{center}
        \tikzset{every picture/.style={scale=1}}
        \subfile{tikz/gcm}
    \end{center}
    \caption{Сеточно-характеристический метод}
    \label{рис:схм}
\end{figure}

Как видно из~рис.~\ref{рис:схм}, сеточно-характеристический метод является локальным как по~времени, так и по
пространственным координатам, т.е. для вычисления значения на~текущем временном слое требуется знать значения на~прошлом
временном слое в~<<соседних>> узлах расчётной сетки. В~принципе, в~силу наличия характеристик у~решаемого
уравнения вовсе не обязательно \cite{васюков2012диссертация} использовать ближайшие к~текущей точке узлы. Однако для
уменьшения вычислительной сложности алгоритма в~данной работе используется ограничение сверху на~величину шага по
времени, подобранное так, что характеристики всегда попадают в~соседний объёмный элемент расчётной сетки:
\[
    \tau_{max} = \frac{h_{min}}{\max_j \abs{\lambda_j}},
\]

где $h_{min}$~--- минимальный характерный размер объёмного элемента сетки (высота тетраэдра или ребро куба), а
$\lambda_j$~--- собственные значения матриц $A_i$ из~уравнения \ref{матричные_уравнения_мдтт}.

\subsubsection{Интерполяция}
Как было сказано ранее, решение на~новом временном слое находится при помощи переноса значения с~предыдущего временного
слоя вдоль соответствующей характеристики уравнения. Однако точка, из~которой требуется перенести значение, в~большинстве
случаев не совпадает ни с~одним из~узлов расчётной сетки. Для восстановления значения в~такой точке используются
механизмы интерполяции: квадратичная для треугольника и тетраэдра \cite{петров2011библиотека}, линейная для отрезка,
билинейная для прямоугольника и трилинейная для прямоугольного параллелепипеда. Соответственно, численное решение,
построенное при помощи се\-точ\-но-ха\-рак\-те\-рис\-ти\-чес\-ко\-го метода, имеет аппроксимацию первого или второго
порядка по~координатам (в зависимости от~выбранного способа интерполяции) и первый порядок по~времени.

\subsubsection{Расчёт граничных узлов}
Описанный ранее способ вычисления значения на~текущем временном слое предполагает, что характеристика попадает во
внутренность тела на~предыдущем расчётном слое. Такое предположение абсолютно неверно для граничных узлов тела: часть
характеристик (а именно не более трёх \cite{челноков2005диссертация}), выпущенных из~таких узлов, попадут за~пределы
тела. В~таком случае требуется использование дополнительных условий. В~данной работе применяются три типа граничных
условий:
\begin{enumerate}
    \item Условие свободной границы предполагает, что на~границу тела не действуют внешние силы, т.е. отсутствуют как
    нормальные, так и тангенциальные напряжения:
    \[
        \sigma_{\tau_1}=\sigma_{\tau_2}=\sigma_n=0.
    \]
    \item Условие полного слипания предполагает, что границы двух контактирующих тел соединены так, что поверхность
    их соприкосновения яляется общей. В~таком случае ставится условие равенства скоростей соответствующих точек двух тел,
    а также условие равенства нормальных напряжений:
    \begin{align}
        \vec v=\tilde{\vec v}, \nonumber \\
        \sigma\vec n=\tilde \sigma \vec n. \nonumber
    \end{align}
    \item Условие скольжения без трения предполагает, что поверхности контактирующих тел движутся друг относительно
    друга без трения, поэтому у~соответствующих точек двух тел равны нормальные напряжения, а тангенциальные напряжения
    равны нулю:
    \begin{align}
        \sigma_n=\tilde{\sigma}_n, \nonumber \\
        \sigma_{\tau_1}=\tilde{\sigma}_{\tau_1}=0, \nonumber  \\
        \sigma_{\tau_2}=\tilde{\sigma}_{\tau_2}=0. \nonumber
    \end{align}
\end{enumerate}

Использование этих граничных условий позволяет получить дополнительные уравнения (3 для случая свободной границы и 6
для контакта), из~которых можно определить значения инвариантов Римана, характеристики для которых являются выводящими.

\subsubsection{Расчёт разрушаемого контакта}

Узел расчётной сетки, находящийся на~границе раздела двух материалов может иметь два различных состояния:
неразрушенное и разрушенное, причём возможен только переход из~первого состояния во~второе, но не наоборот. Для расчёта
разрушаемого контакта используется следующий алгоритм:

\begin{enumerate}
    \item Выполняется расчёт очередного шага в~предположении, что узел не разрушен (т.е. для него установлено
          граничное условие полного слипания).
    \item Вычисляется значение нормальной составляющей силы, действующей на~границе двух тел:
          \[
              F = \sigma_{ij}\cdot \vec n_i \cdot \vec n_i.
          \]
          Если сила оказывает растягивающее воздействие и её модуль превышает значение адгезионной прочности
          ($-F > \sigma^*$), то узел помечается как разрушенный.
    \item Если узел помечен как разрушенный, то на~всех последующих шагах в~нём используется граничное условие свободной
          границы вместо условия полного слипания.
\end{enumerate}

\end{document}
