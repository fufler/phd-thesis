\documentclass[thesis.tex]{subfiles}

\begin{document}

\section*{Введение}
\addcontentsline{toc}{section}{Введение}

\subsection*{Актуальность темы}

Высокие темпы развития современной науки и техники приводят к~появлению новых классов задач, от~решения которых зависит
возможность использования передовых разработок. Так, например, всё более широкое применение находят композиционные
материалы, которые по~многим параметрам заметно превосходят \cite{кербер2011полимерные} <<классические>> материалы,
такие как металлы. Композиты обладают \cite{миллс2011конструкционные,баженов2010полимерные} высокой удельной прочностью,
жёсткостью и износостойкостью, что позволяет использовать их для создания новых классов инженерных конструкций. Весьма
активно в~последнее время ведутся работы по~использованию этих материалов в~аэрокосмической отрасли при проектировании и
создании летательных аппаратов \cite{linde2006influence,буланов1998технология,flower2003materials}. Однако несмотря на
все свои преимущества, композиционные материалы обладают и рядом недостатков, которые сдерживают широкое распространение
этих материалов при создании конструкций, испытывающих нагрузки.

В силу своего устройства композиционные материалы обладают принципиально анизотропными свойствами. Так, например, предел
прочности материала на~сжатие в~различных направлениях может отличаться \cite{васильев1988механика} на~несколько
порядков, что накладывает существенные ограничения на~способы использования конструкций, изготовленных из~таких
материалов. Как показывает практика, наибольшее применение композиционные материалы находят при создании конструкций, не
подверженных серьёзным нагрузкам. Поэтому наиболее актуальными являются вопросы о~поведении композиционных материалов
при низкоскоростных ударах. Но даже при таких режимах нагружения поведение композиционных материалов принципиально
отличается от~поведения металлов в~аналогичных условиях. Из-за наличия армирования в~композиционном материале существует
несколько механизмов его разрушения \cite{richardson1996review}, причём в~некоторых случаях факт разрушения совершенно
не заметен снаружи. Так, расслоение между матрицей и армирующими волокнами композиционного материала резко снижает
несущую способность материала, но при этом может быть обнаружено только специальными средствами для проведения
неразрушающего контроля,  но эти средства главным образом предназначены для однородных материалов, в~связи с~чем
исследование композитов представляет собой трудоемкую и дорогостоящую задачу. В~связи с~этим достаточно остро стоит
вопрос создания средств компьютерного моделирования, позволяющих предсказывать поведение конструкций, выполненных из
композиционного материала, при динамическом нагружении.

Волновые процессы \cite{лурье1955пространственные,бреховских1973волны}, протекающие в~композиционном материале при
ударном нагружении, могут быть описаны при помощи уравнений механики деформируемого твёрдого тела
\cite{лехницкий1977теория}. Аналитические выражения \cite{Zhupanska2010indentation,shahid1993progressive}, описывающие
волновые процессы, могут быть получены только для весьма ограниченного класса задача. Поэтому для подавляющего
большинства задач ввиду сложности рассматриваемой в~постановке геометрии, а также различного вида граничных условий
использование аналитических решений невозможно.

Тем не менее, задача точного количественного и качественного описания волновых процессов, протекающих в~средах со
сложной внутренней структурой (например, слоистых средах) является весьма актуальной. Так, процессы хрупкого разрушения
имеют волновую природу и вызваны взаимодействием в~веществе множества прямых и отражённых волн. В~случае слоистой
структуры или анизотропной природы материала характер волновой картины сильно усложняется из-за наличия в~теле
внутренних границ, на~которых возможно отражение и преломление распространяющихся волн.

Традиционным подходом к~решению задач, для которых найти аналитическое решение не представляется возможным, является
численное моделирование. Численное моделирование как отдельный раздел науки сформировалось относительно недавно~--- в
середине XX века были опубликованы одни из~первых работ по~данной тематике \cite{crank1947practical,vonneumann1950method,courant1952solution,lax1954weak,лебедев1955уравнения,рябенький1956устойчивость,годунов1957разностный,дородницын1958один,владимиров1958численное,белоцерковский1958расчёт,федоренко1962применение,ландау1958чтсленные,lax1960system,тихонов1961однородные},
заложившие основу для дальнейшего развития этого направления. Изначально численные методы предлагались для решения
наиболее актуальных задач того времени --- задач аэро- и гидродинамики. В~дальнейшем предложенные подходы были
успешно применены и для решения задач других классов.

Наиболее часто для численного решения задач механики деформируемого тела используется \cite{bermudez1999finite}
метод конечных элементов \cite{морозов2008метод,партон2007механика}. Наиболее широкое распространение этот метод нашёл
при решении задач о~статическом нагружении. Тем не менее, этот метод применяется \cite{johnson1977high} и для расчёта
задач о~столкновении тел, но в~силу своей природы метод не обладает свойствами, необходимым для корректного
моделирования волновых процессов.

В данной работе для расчёта задач, связанных с~распространением волн в~материалах, используется
сеточно-характеристический метод \cite{магомедов1988схм,белоцерковский1994численное,петров1989численное,петров1986волнове,петров1984численное,холодов1984регуляризация,ермаков2013построение}.
Этот метод очень хорошо подходит для численного моделирования волновых процессов благодаря тому, что учитывает
характеристические свойства системы уравнений, используемой для их описания. Обладая высоким пространственным и
временным разрешением, метод позволяет моделировать процессы взаимодействия прямых и отражённых волн, а также корректно
учитывать различного рода граничные и контактные условия. При использовании соответствующих моделей
\cite{майнчен1967тензор} и критериев разрушения \cite{puck1969failure,hashin1973fatigue,tsai1971general,hill1979theoretical,cazacu2001generalization}
метод позволяет моделировать разрушение материала, а также волновой отклик от~разрушенных областей. Для учёта реологии
материала, в~котором протекают волновые процессы, используется наиболее распространённый на~данный момент подход~---
осреднение параметров \cite{победря1984механика,бахвалов1980осреднение,бахвалов1974осреднение}.

Численное моделирование задачи низкоскоростного соударения имеет практический интерес, так как на~данный момент не
существует всеобъемлющей теории по~предсказанию поведения композиционных материалов при таких режимах нагружения, а
установление факта повреждения конструкции является весьма нетривиальной задачей. При высокоскоростных соударениях в
большинстве случаев повреждения могут быть обнаружены даже без использования специальных средств. Однако постановка и
проведение экспериментов для задач высокоскоростного или высокоэнергетического взаимодействия тел на~практике
представляют большую сложность ввиду высокой стоимости подобных опытов или же их опасности.

Для численного моделирования задач о~высокоскоростных столкновениях тел в~данной работе используется метод сглаженных
частиц \cite{monaghan1988introduction}, получивших наибольшее распространение в~области моделирования задач
гидродинамики. Этот метод при условии использовании некоторых методик для повышения устойчивости и монотонизации
\cite{monaghan1989problem,monaghan1997sph,потапов2009диссертация,паршиков2013диссертация} позволяет численно решать
задачи, в~которых имеются существенная деформации области интегрирования. К~недостаткам метода сглаженных частиц можно
отнести тот факт, что он достаточно сильно размывает фронты ударных волн, в~связи с~чем он обычно не применяется для
получения информации о~волновых процессах в~материале. В~данной работе предложены два метода, позволяющие одновременно
учитывать волновые процессы и решать задачи с~большими деформациями области интегрирования: комбинированный метод
\cite{петров2015комбинированный,васюков2014комбинирование,petrov2014combined,петров2014схм} на~основе метода сглаженных
частиц и сеточно-характеристического метода и адаптация классического метода маркеров и ячеек \cite{harlow1965numerical}
для работы совместно с~сеточно-характеристическим методом.

\subsection*{Цели работы}

\begin{enumerate}
    \item Разработка и программная реализация вычислительного алгоритма для численного метода, позволяющего
          моделировать сложные волновые процессы, происходящие в анизотропных композитных конструкциях.
    \item Анализ и сопоставление существующих критериев объёмного разрушения материала с~целью формулирования рекомендаций по их
          применению при моделировании композиционных материалов.
    \item Реализация и верификация программного комплекса, использующего сеточно-характеристический метод для
          численного моделирования процессов в~анизотропных средах.
    \item Реализация и верификация программного комплекса, использующего комбинацию сеточно-характеристического
          метода и метода сглаженных частиц для решения задач с~конечными деформациями.
    \item Адаптация трёхмерного метода маркеров для решения задачи динамического деформирования твёрдого
          тела с~конечными деформациями при помощи сеточно-характеристического метода.
    \item Разработка и верификация программного комплекса, использующего предложенную комбинацию метода маркеров
          и сеточно-ха\-рак\-те\-риc\-ти\-чес\-ко\-го метода, для моделирования волновых процессов в~деформируемом
          твёрдом теле.
    \item Применение описанных выше методов для решения ряда практически значимых задач.
\end{enumerate}

\subsection*{Научная новизна}

\begin{enumerate}
    \item Адаптирован для использования совместно с~сеточно-характерис\-ти\-чес\-ким методом и проверен на~модельных
          задачах трёхмерный метод маркеров, позволяющий проводить численное моделирование процессов в~задачах механики
          деформируемого твёрдого тела в~условиях конечных деформаций.
    \item Проведен анализ наиболее используемых на~данный момент критериев разрушения композиционного
          материала, по~результатам которого сделаны выводы о~целесообразности применения этих критериев для решения практически
          значимых задач.
    \item Программно реализован численный метод, позволяющий моделировать сложные волновые процессы, происходящие в конструкциях,
          изготовленных из~анизотропных композиционных материалов.
    \item Реализован и проверен на~модельных задачах программный комплекс, использующий численный метод,
          являющийся комбинацией сеточно-характеристического метода и метода сглаженных частиц.
    \item С~использованием описанных методов решён ряд практически значимых задач, среди которых:
        \begin{itemize}
            \item задача о~низкоскоростном ударе по~трёхстрингерной композиционной панели;
            \item задача о~низкоскоростном ударе по~композиционным панелям, выполненным из~различных материалов
                  (GFRP и CFRP);
            \item задача об~объёмном разрушении стекла под действием лазерного излучения;
            \item задача о~высокоскоростном ударе по~прозрачной слоистой конструкции;
            \item задача о~падении самолёта на~оболочку ядерного реактора;
            \item задача о~пробивании оболочки спутника микрометеоритом;
            \item задача о~столкновении стального осколка со~стальной преградой при различных соотношениях
                  характерных размеров  тел, участвующих в~столкновении.
        \end{itemize}
\end{enumerate}

\subsection*{Теоретическая и практическая значимость работы}
Реализованный метод моделирования волновых процессов, протекающих в~анизотропных композиционных материалах, может быть
в~дальнейшем использован  для численного моделирования поведения композиционных авиационных конструкций при действии
различных динамических  нагрузок. Также этот метод может быть использован для решения задач неразрушающего контроля и
дефектоскопии.

Результаты моделирования процесса разрушения стеклянного образца под действием лазерного излучения могут быть
использованы для исследования поведения подобных материалов при энергетических воздействиях и верификации моделей
объёмного разрушения хрупких материалов.

Моделирование волновых процессов в~многослойных прозрачных конструкциях проводится для улучшения
их защищённости, что может быть использовано, например, в~военной и гражданской авиации.

Численное моделирование процессов высокоскоростного взаимодейст\-вия используется для исследования несущей
способности различных конструкций, подверженных ударному нагружению.

Предложенная комбинация метод маркеров и сеточно-ха\-рак\-те\-рис\-ти\-чес\-ко\-го метода обеспечивает
возможность выполнения расчётов, требующих одновременного изменения геометрии расчётной области и моделирования
сложных волновых процессов, порождённых интенсивными динамическими нагрузками.

Работа поддержана рядом государственных и коммерческих грантов:
\begin{enumerate}
    \item Грант РФФИ 11-01-00723 А. Разработка численных методов моделирования динамических задач биомеханики на
          современных высокопроизводительных вычислительных системах, 2011-2013 гг.
    \item Грант РФФИ 13-07-00072 А. Разработка параллельных алгоритмов для решения систем уравнений гиперболического
          типа на~многопроцессорных вычислительных системах, 2013-2015 гг.
\end{enumerate}

\subsection*{Методология и методы исследования}

Исследуются задачи механики деофрмируемого твёрдого тела методами численного моделирования. В работе используется
сеточно-ха\-рак\-те\-рис\-ти\-чес\-кий численный метод на неструктурированных тетраэдральных сетках, комбинация метода маркеров
и сеточно-характеристического метода  на регулярных гексаэдральных эйлеровых сетках, а также комбинация метода
сглаженных частиц и сеточно-характеристического метода. Для верификации методов используется сравнение с
аналитическими решениями ряда задач, а также сравнение с экспериментальными данными. При исследовании прикладных
задач методами моделирования рассчитывается полная пространственно-временная волновая картина, позволяющая получить
точную информация о полях скоростей и напряжений внутри тела, а также выявить области деформации и разрушения.

\subsection*{Положения, выносимые на защиту}
Положения, выносимые на защиту, соответствуют основным результатам, приведённым в заключении диссертации.

\subsection*{Степень достоверности и апробации результатов}

Результаты диссертации опубликованы в~одиннадцати \cite{беклемышева2013численное,васюков2014комбинирование,петров2014схм,petrov2014combined,петров2014численный,петров2015комбинированный,беклемышева2014численное,беклемышева2012численное,ермаков2013построение,беклемышева2013численное2,беклемышева2013численное3}
работах, из~которых семь \cite{беклемышева2013численное,васюков2014комбинирование,петров2014схм,petrov2014combined,петров2014численный,петров2015комбинированный,беклемышева2014численное}~---
в~изданиях, рекомендованных ВАК для публикации основных результатов диссертации.

Результаты работы были доложены, обсуждены и получили одобрение специалистов на~следующих научных конференциях:

\begin{enumerate}
    \item 55-я научная конференция МФТИ <<Проблемы фундаментальных и прикладных естественных и технических наук в
          современном информационном обществе>> (МФТИ, Долгопрудный, 2012).
    \item 56-я научная конференция МФТИ <<Проблемы фундаментальных и прикладных естественных и технических наук в
          современном информационном обществе>> (МФТИ, Долгопрудный, 2013).
    \item Семинары Центра компьютерного моделирования (ЦКМ) в рамках программы совместных фундаментальных исследований
          ЦАГИ и РАН (ЦАГИ, Жуковский, 2014--2015).
\end{enumerate}

Результаты работы были доложены, обсуждены и получили одобрение специалистов на~научных семинарах в~следующих
организациях:

\begin{enumerate}
    \item Национальный исследовательский центр <<Курчатовский институт>> (Москва, 2014).
    \item Центральный аэрогидродинамический институт имени профессора
          Н. Е. Жуковского (Жуковский, 2014--2015).
    \item Институт прикладной математики им. М.В. Келдыша Российской академии наук (Москва, 2015).
    \item Институт автоматизации проектирования Российской академии наук (Москва, 2015).
\end{enumerate}

\subsection*{Личный вклад соискателя в~работах с~соавторами}

Сеточно-характеристический метод реализован для численного решения задач о~динамическом поведении анизотропных
композитных конструкций.

В~части численных методов соискателем реализован метод, представляющий из~себя комбинацию метода сглаженных частиц и
сеточно-ха\-рак\-те\-рис\-ти\-чес\-ко\-го метода. Проведена верификация реализованного метода на~ряде модельных задач
распада разрыва. Также реализован метод маркеров, предполагающий использование структурированных эйлеровых сеток и
сеточно-характеристического метода для моделирования процессов с~существенными изменениями геометрии расчётной
области.

В~части программной реализации соискателем проделана работа по~реализации описанных ранее методов, а также
интеграции полученного программного комплекса с~библиотеками, использующимися для визуализации (paraview, pvbatch)
в~автоматическом режиме.

В~части математического моделирования соискателем выполнено исследование волновых процессов, возникающих в
композиционных материалах при ударном нагружении и приводящих к~последующему разрушению материала и деламинации.
Проведено численное моделирование последствий воздействия лазерного излучения на~стеклянный образец, а также
проведено сравнение полученных данных с~экспериментом. Выполнено моделирование поведения многослойной прозрачной
конструкции при ударном нагружении. Получена зависимость размеров области расслоения от~адгезионной прочности. Также
выполнены расчёты нескольких высокоскоростных столкновений, в~частности, падения быстро движущегося объекта на~сферическую
оболочку, столкновения микрометеорита с космическим спутником, а также пробивания стальным осколком стальных преград различной толщины.

\end{document}
