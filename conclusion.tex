\documentclass[thesis.tex]{subfiles}

\begin{document}

    \section*{Заключение}
    \addcontentsline{toc}{section}{Заключение}

    \textbf{Основные результаты и выводы диссертации:}

    \begin{enumerate}
        \item Реализован сеточно-характеристический численный метод с~использованием неcтруктурированных расчётных
              сеток для моделирования сложных волновых процессов, происходящих в~анизотропных композитных конструкциях.
              Метод обладает возможностью определения области разрушения в~результате воздействия различных динамических
              нагрузок, а также расчёта откликов разрушенных областей для решения задач неразрушающего контроля.
        \item Проведено исследование наиболее используемых на~данный момент критериев разрушения композиционного
              материала, по~результатам которого сделаны выводы о~применимости этих критериев для решения практически
              значимых задач динамической прочности композитных конструкций.
        \item Реализован и проверен на~модельных задачах программный комплекс, использующий комбинированный численный
              метод, основанный на~сеточно-характеристическом методе и методе сглаженных частиц. Получено качественное
              совпадение результатов решения модельных задач с~результатами расчёта при помощи коммерческого
              программного комплекса.
        \item Адаптирован для использования совместно с~сеточно-характеристи\-чес\-ким методом и проверен на~модельных
              задачах трёхмерный метод маркеров, позволяющий проводить численное моделирование в~задачах механики
              деформируемого твёрдого тела в~условиях конечных деформаций. Этот подход обеспечивает возможность решать
              задачи с~большими деформациями, а также позволяет в~несколько раз ускорить расчёт по~сравнению с
              использованием неструктурированных сеток для всего тела благодаря фиксированному шагу по~времени.
        \item Выполнено моделирование волновых процессов в~стекле, возникающих под действием лазерного излучения.
              На~качественном уровне получено совпадение с~результатами эксперимента.
        \item Проведено моделирование волновых процессов, сопровождающихся хрупким разрушением, в~многослойной
              стеклянной конструкции.
        \item Выполнено моделирование низкоскоростного удара по~композиционным панелям из~разных материалов. Получены
              результаты, свидетельствующие о~зависимости размеров и формы области расслоения от~используемого критерия
              объёмного разрушения.
        \item Проведено моделирование низкоскоростного удара по~трёхстрингерной композиционной панели с~использованием
              различных критериев разрушения. Получено совпадение с~результатами эксперимента на~качественном уровне.
        \item Выполнено моделирование процесса пробивания осколком преград различной толщины.
        \item Выполнены расчёты ряда задач о~высокоскоростном соударении тел: падение самолёта на
              здание и столкновение микрометеорита со~спутником.

    \end{enumerate}


\end{document}
