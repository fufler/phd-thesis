\documentclass[thesis.tex]{subfiles}

\begin{document}

\subsection{Пробивание преград различной толщины}

Рассматривается задача о~пробивании стальным осколком стальной преграды при различном соотношении диаметра
осколка и толщины преграды. Аналогичная постановка была рассмотрена в~разделе \ref{раздел:верификация-комбинированного-метода}
для верификации комбинированного численного метода. Стоит отметить, что сравнение с~экспериментом решений, полученных
численным путём, для большинства подобных постановок является весьма нетривиальной задачей, как ввиду отсутствия самих
экспериментальных данных, так и из-за сложности измерения необходимых параметров в~ходе эксперимента. В~подавляющем
большинстве случаев верификация проводится по~некоторым интегральным критериям (например, по~факту пробивания или
скорости отскока).

Во всех постановках диаметр осколка берётся равным 3~мм, а толщина преграды составляет 1~мм (тонкая преграда), 3~мм
(преграда средней толщины) и 9~мм (толстая преграда) Угол между нормалью к~поверхности преграды и вектором скорости
осколка составляет 0\degree\ и 30\degree\ для разных постановок. Модуль скорости осколка равен 1200~м/с для всех
постановок. Преграда изготовлена из~материала <<сталь 09Г2С>>,  а осколок~--- из~материала <<сталь 20>>.  Параметры
материалов (в соответствии с~используемой моделью вещества \eqref{уравнения_движения_в_частицах}) приведены в~табл.~
\ref{таб:параметры-материалов-в-задаче-пробивания}.

\begin{table}[h!]
    \centering
    \begin{tabular}{|l|l|l|l|l|l|} \hline
    Материал & $\rho_0$, $\textnormal{г/см}^3$ & $\mu$, Мбар & $K$, Мбар & $Y_0$, Кбар & $\sigma$, Кбар \\ \hline
    Сталь 20  & 7.86 & 2.0 & 2.52 & 1.75 & 3.5 \\ \hline
    Сталь 09Г2С & 7.86 & 2.0 & 2.52 & 3.45 & 4.9 \\ \hline
    \end{tabular}
    \caption{Параметры материалов}
    \label{таб:параметры-материалов-в-задаче-пробивания}
\end{table}

Результаты расчётов приведены на~рис.  \ref{рис:осколок-тонкая-преграда-0}--\ref{рис:осколок-толстая-преграда-7}.

По представленным результатам можно сделать вывод, что численный метод воспроизводит качественные особенности процесса
высокоскоростного соударения осколка и преграды. Так, при столкновении с~преградой малой толщины наблюдается сквозное
пробивание при обоих углах подхода осколка, при этом для столкновения под углом 30\degree\ область повреждения преграды
имеет характерные загнутые края, которые можно наблюдать в~эксперименте \cite{губеладзе2010исследование}. При
столкновении осколка с~преградой средней толщины также наблюдается сквозное пробивание при обоих углах подхода, но, что
характерно, в~случае столкновения с~преградой под углом 30\degree\ область разрушения имеет значительно меньшие размеры.
Как и ожидалось, при столкновении с~толстой преградой сквозного пробивания не происходит ни при одном из~углов
столкновения.

\begin{python}
from gen_obstacle_images import gen
gen(
    'thin',
    {
        0:  (0, 0),
        5:  (5.04e-6, 5.05e-6),
        10: (1.02e-5, 1.02e-5),
        15: (1.53e-5, 1.53e-5),
        20: (2.02e-5, 2.04e-5),
        25: (2.52e-5, 2.55e-5),
        30: (3.01e-5, 3.05e-5),
        35: (3.51e-5, 3.55e-5)
    },
    'тонкой преградой',
    'тонкая-преграда'
)
gen(
    'eq',
    {
        0:  (0, 0),
        5:  (5.02e-6, 5.06e-6),
        10: (1.02e-5, 1.02e-5),
        15: (1.53e-5, 1.53e-5),
        20: (2.02e-5, 2.04e-5),
        25: (2.52e-5, 2.54e-5),
        30: (3.01e-5, 3.04e-5),
        35: (3.51e-5, 3.53e-5)
    },
    'преградой средней толщины',
    'средняя-толщина'
)
gen(
    'thick',
    {
        0:  (0, 0),
        5:  (4.99e-6, 5.06e-6),
        10: (1.00e-5, 1.02e-5),
        15: (1.50e-5, 1.52e-5),
        20: (1.99e-5, 2.01e-5),
        25: (2.49e-5, 2.51e-5),
        30: (2.98e-5, 3.01e-5),
        35: (3.48e-5, 3.50e-5)
    },
    'толстой преградой',
    'толстая-преграда'
)
\end{python}


\end{document}
