\documentclass[thesis.tex]{subfiles}

\begin{document}

\section{Уравнения механики деформируемого твёрдого тела}
\subsection{Уравнения движения и реологические соотношения}

Процессы, происходящие в~деформируемом твёрдом теле, могут быть описаны \cite{новацкий1975теория,седов1973механика}
при помощи следующей системы уравнений:
\begin{eqnarray}
    \label{уравнения_мдтт}
    \rho\dot{v}_i=\nabla_j\sigma_{ij}+f_i & \textnormal{(уравнения движения),}\nonumber\\
    \dot\sigma_{ij}=C_{ijkl}\dot{\varepsilon}_{kl}+F_{ij} & \textnormal{(реологические соотношения).}
\end{eqnarray}
Здесь $\rho$~--- плотность среды, $v_i$~--- компоненты скорости смещения, $\sigma_{ij}$, $\varepsilon_{ij}$ ---
компоненты тензоров напряжений и деформаций, $\nabla_j$~--- ковариантная производная по~$j$-й координате,
$f_i$~--- массовые силы, действующие на~единицу объёма, $F_{ij}$~--- добавочная правая часть, конкретный вид которой
определяется свойствами рассматриваемого материала.

В приближении малых деформаций тензор скоростей деформаций $\dot{\varepsilon}_{ij}$ может быть выражен через
компоненты скоростей смещения следующим образом:
\[
    e_{ij}=\dot{\varepsilon}_{ij}=\frac{1}{2}(\nabla_j v_i+\nabla_i v_j).
\]
Конкретный вид компонент тензора $C_{ijkl}$ определяется исключительно реологией среды и более подробно будет рассмотрен
далее.

Обычно для удобства и краткости исходная система \eqref{уравнения_мдтт} записывается в~матричном виде
\begin{equation}
    \label{матричные_уравнения_мдтт}
    \PDt{\vec u} + \sum_{i=1}^{3} \mathbf A_i \PD{\vec u}{x_i} = \vec f,
\end{equation}
где $\vec u = \{v_x,v_y,v_z,\Sxx,\Sxy,\Sxz,\Syy,\Syz,\Szz\}^T$~--- вектор
неизвестных, содержащий три компоненты скорости смещения и 6~компонент тензора напряжений (в силу закона парности
касательных напряжений \cite{кондауров1973основы}), $\mathbf A_i$~--- матрицы, зависящие от~реологии среды,
а $\vec f$~--- правая часть уравнения, характеризующая воздействие внешних сил на~систему. Очевидно, вид матрицы $A_i$
напрямую зависит от~вида тензора $C_{ijkl}$ и определяется реологией среды, а также используемой моделью вещества.

\subsection{Изотропная линейно-упругая модель}

В случае линейно-упругого изотропного тела реологические соотношения из~системы уравнений \eqref{уравнения_мдтт}  принимают
\cite{ландау1987теория} вид
\[
\begin{array}{lcl}
    C_{ijkl} &=& \lambda\delta_{ij}\delta_{kl} + \mu(\delta_{ik}\delta_{jl}+\delta_{il}\delta_{jk}), \\
    F_{ij} &=& 0,
\end{array}
\]
а сами исходные уравнения записываются так:
\[
\begin{array}{lcl}
    \PDt{v_x} &=&  \frac{1}{\rho}\left(\PDx{\Sxx} + \PDy{\Sxy} + \PDz{\Sxz} \right), \\
    \PDt{v_y} &=&  \frac{1}{\rho}\left(\PDx{\Sxy} + \PDy{\Syy} + \PDz{\Syz} \right), \\
    \PDt{v_z} &=&  \frac{1}{\rho}\left(\PDx{\Sxz} + \PDy{\Syz} + \PDz{\Szz} \right), \\

    \PDt{\Sxx} &=& (\lambda+2\mu)\PDx{v_x} + \lambda\PDy{v_y} + \lambda\PDz{v_z}, \\
    \PDt{\Syy} &=& \lambda\PDx{v_x} + (\lambda+2\mu)\PDy{v_y} + \lambda\PDz{v_z}, \\
    \PDt{\Szz} &=& \lambda\PDx{v_x} + \lambda\PDy{v_y} + (\lambda+2\mu)\PDz{v_z}, \\

    \PDt{\Sxy} &=&  \mu\left(\PDy{v_x} + \PDx{v_y}\right), \\
    \PDt{\Sxz} &=&  \mu\left(\PDz{v_x} + \PDx{v_z}\right), \\
    \PDt{\Syz} &=&  \mu\left(\PDz{v_y} + \PDy{v_z}\right). \\
\end{array}
\]
В этой системе уравнений $\lambda$ и $\mu$~--- упругие постоянные Ламэ, которые могут быть выражены через модуль Юнга
$E$ и коэффициент Пуассона $\nu$ следующим образом:
\[
\begin{array}{lcl}
    \lambda &=& \frac{E\nu}{(1+\nu)(1-2\nu)}, \\
    \mu &=& \frac{E}{2(1+\nu)}.
\end{array}
\]
С учётом этого исходные уравнения \eqref{уравнения_мдтт} могут быть записаны в~матричной форме
\eqref{матричные_уравнения_мдтт}, где матрицы $\mathbf A_i$ имеют следующий вид:
\begin{small}
\begin{equation}
    \label{изотропная_матрица_x}
    \mathbf A_x = \begin{pmatrix}
    0 & 0 & 0 & -\frac 1 \rho & 0 & 0 & 0 & 0 & 0 \\
    0 & 0 & 0 & 0 & -\frac 1 \rho & 0 & 0 & 0 & 0 \\
    0 & 0 & 0 & 0 & 0 & -\frac 1 \rho & 0 & 0 & 0 \\
    -\lambda-2\mu & 0 & 0 & 0 & 0 & 0 & 0 & 0 & 0 \\
    0 & -\mu & 0 & 0 & 0 & 0 & 0 & 0 & 0 \\
    0 & 0 & -\mu & 0 & 0 & 0 & 0 & 0 & 0 \\
    -\lambda & 0 & 0 & 0 & 0 & 0 & 0 & 0 & 0 \\
    0 & 0 & 0 & 0 & 0 & 0 & 0 & 0 & 0 \\
    -\lambda & 0 & 0 & 0 & 0 & 0 & 0 & 0 & 0
    \end{pmatrix},
\end{equation}
\begin{equation}
    \label{изотропная_матрица_y}
    \mathbf A_y = \begin{pmatrix}
    0 & 0 & 0 & 0 & -\frac 1 \rho & 0 & 0 & 0 & 0 \\
    0 & 0 & 0 & 0 & 0 & 0 & -\frac 1 \rho & 0 & 0 \\
    0 & 0 & 0 & 0 & 0 & 0 & 0 & -\frac 1 \rho & 0 \\
    0 & -\lambda & 0 & 0 & 0 & 0 & 0 & 0 & 0 \\
    -\mu & 0 & 0 & 0 & 0 & 0 & 0 & 0 & 0 \\
    0 & 0 & 0 & 0 & 0 & 0 & 0 & 0 & 0 \\
    0 & -\lambda-2\mu & 0 & 0 & 0 & 0 & 0 & 0 & 0 \\
    0 & 0 & -\mu & 0 & 0 & 0 & 0 & 0 & 0 \\
    0 & -\lambda & 0 & 0 & 0 & 0 & 0 & 0 & 0
    \end{pmatrix},
\end{equation}
\begin{equation}
    \label{изотропная_матрица_z}
    \mathbf A_z = \begin{pmatrix}
    0 & 0 & 0 & 0 & 0 & -\frac 1 \rho & 0 & 0 & 0 \\
    0 & 0 & 0 & 0 & 0 & 0 & 0 & -\frac 1 \rho & 0 \\
    0 & 0 & 0 & 0 & 0 & 0 & 0 & 0  & -\frac 1 \rho \\
    0 & 0 & -\lambda & 0 & 0 & 0 & 0 & 0 & 0 \\
    0 & 0 & 0 & 0 & 0 & 0 & 0 & 0 & 0 \\
    -\mu & 0  & 0 & 0 & 0 & 0 & 0 & 0 & 0 \\
    0 & 0 & -\lambda & 0 & 0 & 0 & 0 & 0 & 0 \\
    0 & -\mu & 0 & 0 & 0 & 0 & 0 & 0 & 0 \\
    0 & 0 & -\lambda-2\mu & 0 & 0 & 0 & 0 & 0 & 0
    \end{pmatrix}. \nonumber
\end{equation}
\end{small}

\subsection{Тензор упругих постоянных}

Как было сказано ранее, напряжения, возникающие в~материале при малых деформациях, линейным образом зависят от~величины
деформации. В~трёхмерном случае эта связь описывается законом Гука (по сути реологические соотношения из~уравнений
\eqref{уравнения_мдтт}):
\[
    \sigma_{ij} = C_{ijkl}\varepsilon_{kl}.
\]

Очевидно, что в~самом общем случае тензор $C_{ijkl}$ содержит  $3^4=81$ компоненту. В~предположении малых деформаций
тензор напряжений симметричен, одновременно с~этим по~определению симметричен и тензор деформаций. В~таком случае
несложно доказать \cite{реслер2011механическое}, что $C_{ijkl}=C_{kikl}=$\newline $=C_{ijlk}$, откуда следует, что тензор
упругих постоянных для анизотропного тела в~приближении малых деформаций имеет только $6^2=36$ независимых компонент.
Далее, ввиду потенциальности упругого деформирования тензор упругих постоянных обладает дополнительной симметрией
$C_{ijkl}=C_{klij}$, а, соответственно, имеет только 21 независимую компоненту, которые удобнее всего представить в~виде
симметричной матрицы упругости:
\begin{equation}
    \label{анизотропная_матрица}
    c_{ik} =\begin{pmatrix}
    c_{11} & c_{12} & c_{13} & c_{14} & c_{15} & c_{16} \\
    c_{12} & c_{22} & c_{23} & c_{24} & c_{25} & c_{26} \\
    c_{13} & c_{23} & c_{33} & c_{34} & c_{35} & c_{36} \\
    c_{14} & c_{24} & c_{34} & c_{44} & c_{45} & c_{46} \\
    c_{15} & c_{25} & c_{35} & c_{45} & c_{55} & c_{56} \\
    c_{16} & c_{26} & c_{36} & c_{46} & c_{56} & c_{66}
    \end{pmatrix}.
\end{equation}

После введения этой матрицы закон Гука может быть записан в более простой матричной форме:
\[
    \begin{pmatrix}
    \sigma_{11} \\
    \sigma_{22} \\
    \sigma_{33} \\
    \sigma_{23} \\
    \sigma_{13} \\
    \sigma_{12}
    \end{pmatrix}
    = \begin{pmatrix}
    c_{11} & c_{12} & c_{13} & c_{14} & c_{15} & c_{16} \\
    c_{12} & c_{22} & c_{23} & c_{24} & c_{25} & c_{26} \\
    c_{13} & c_{23} & c_{33} & c_{34} & c_{35} & c_{36} \\
    c_{14} & c_{24} & c_{34} & c_{44} & c_{45} & c_{46} \\
    c_{15} & c_{25} & c_{35} & c_{45} & c_{55} & c_{56} \\
    c_{16} & c_{26} & c_{36} & c_{46} & c_{56} & c_{66}
    \end{pmatrix}
    \begin{pmatrix}
    \varepsilon_{11} \\
    \varepsilon_{22} \\
    \varepsilon_{33} \\
    \varepsilon_{23} \\
    \varepsilon_{13} \\
    \varepsilon_{12}
    \end{pmatrix}.
\]

\subsection{Анизотропная линейно-упругая модель}

Для описания линейно-упругого приближения для анизотропной среды используется \cite{фаворская2011постановка} следующая
система уравнений:
\begin{small}
\begin{equation}
\label{анизотропные_уравнения_мдтт}
\begin{array}{lcl}
    \PDt{v_x} &=& \frac{1}{\rho}(\PDx \Sxx + \PDy \Sxy + \PDz \Sxz), \\
    \PDt{v_y} &=& \frac{1}{\rho}(\PDx \Sxy + \PDy \Syy + \PDz \Syz), \\
    \PDt{v_z} &=& \frac{1}{\rho}(\PDx \Sxz + \PDy \Syz + \PDz \Szz), \\

    \PDt \Sxx &=& c_{11}\PDx{v_x} + c_{12}\PDy{v_y} + c_{13}\PDz{v_z} +
                  c_{14}(\PDy{v_z} + \PDz{v_y}) +
                  c_{15}(\PDx{v_z} + \PDz{v_x}) +
                  c_{16}(\PDx{v_y} + \PDy{v_x}), \\
    \PDt \Syy &=& c_{12}\PDx{v_x} + c_{22}\PDy{v_y}+c_{23}\PDz{v_z} +
                  c_{24}(\PDy{v_z}+\PDz{v_y}) +
                  c_{25}(\PDx{v_z}+\PDz{v_x}) +
                  c_{26}(\PDx{v_y}+\PDy{v_x}), \\
    \PDt \Szz &=& c_{13}\PDx{v_x} + c_{23}\PDy{v_y} + c_{33}\PDz{v_z} +
                  c_{34}(\PDy{v_z} + \PDz{v_y}) +
                  c_{35}(\PDx{v_z} + \PDz{v_x}) +
                  c_{36}(\PDx{v_y} + \PDy{v_x}), \\
    \PDt \Syz &=& c_{14}\PDx{v_x} + c_{24}\PDy{v_y} + c_{34}\PDz{v_z} +
                  c_{44}(\PDy{v_z} + \PDz{v_y}) +
                  c_{45}(\PDx{v_z} + \PDz{v_x}) +
                  c_{46}(\PDx{v_y }+ \PDy{v_x}), \\
    \PDt \Sxz &=& c_{15}\PDx{v_x} + c_{25}\PDy{v_y} + c_{35}\PDz{v_z} +
                  c_{45}(\PDy{v_z} + \PDz{v_y}) +
                  c_{55}(\PDx{v_z} + \PDz{v_x}) +
                  c_{56}(\PDx{v_y} + \PDy{v_x}), \\
    \PDt \Sxy &=& c_{16}\PDx{v_x} + c_{26}\PDy{v_y} + c_{36}\PDz{v_z} +
                  c_{46}(\PDy{v_z} + \PDz{v_y}) +
                  c_{56}(\PDx{v_z} + \PDz{v_x}) +
                  c_{66}(\PDx{v_y} + \PDy{v_x}).
\end{array}
\end{equation}
\end{small}
Коэффициенты $c_{ij}$ в~этих уравнениях являются компонентами матрицы \eqref{анизотропная_матрица}. Соответственно,
систему уравнений \eqref{анизотропные_уравнения_мдтт} можно записать в~матричной форме \eqref{матричные_уравнения_мдтт}, при
этом матрицы $\mathbf{A_i}$ будут иметь следующий вид:
\begin{small}
\begin{equation}
    \label{анизотропная_матрица_x}
    \mathbf{A_x} = -\begin{pmatrix}
    0 & 0 & 0 & \frac 1 \rho & 0 & 0 & 0 & 0 & 0 \\
    0 & 0 & 0 & 0 & \frac 1 \rho & 0 & 0 & 0 & 0 \\
    0 & 0 & 0 & 0 & 0 & \frac 1 \rho & 0 & 0 & 0 \\
    c_{11} & c_{16} & c_{15} & 0 & 0 & 0 & 0 & 0 & 0 \\
    c_{16} & c_{66} & c_{56} & 0 & 0 & 0 & 0 & 0 & 0 \\
    c_{15} & c_{56} & c_{55} & 0 & 0 & 0 & 0 & 0 & 0 \\
    c_{12} & c_{26} & c_{25} & 0 & 0 & 0 & 0 & 0 & 0 \\
    c_{14} & c_{46} & c_{45} & 0 & 0 & 0 & 0 & 0 & 0 \\
    c_{13} & c_{36} & c_{35} & 0 & 0 & 0 & 0 & 0 & 0
    \end{pmatrix},
\end{equation}
\begin{equation}
    \label{анизотропная_матрица_y}
    \mathbf{A}_y = -\begin{pmatrix}
    0 & 0 & 0 & 0 & \frac 1 \rho & 0 & 0 & 0 & 0 \\
    0 & 0 & 0 & 0 & 0 & 0 & \frac 1 \rho & 0 & 0 \\
    0 & 0 & 0 & 0 & 0 & 0 & 0 & \frac 1 \rho & 0 \\
    c_{16} & c_{12} & c_{14} & 0 & 0 & 0 & 0 & 0 & 0 \\
    c_{66} & c_{26} & c_{46} & 0 & 0 & 0 & 0 & 0 & 0 \\
    c_{56} & c_{25} & c_{45} & 0 & 0 & 0 & 0 & 0 & 0 \\
    c_{26} & c_{22} & c_{24} & 0 & 0 & 0 & 0 & 0 & 0 \\
    c_{46} & c_{24} & c_{44} & 0 & 0 & 0 & 0 & 0 & 0 \\
    c_{36} & c_{23} & c_{34} & 0 & 0 & 0 & 0 & 0 & 0
    \end{pmatrix},
\end{equation}
\begin{equation}
    \label{анизотропная_матрица_z}
    \mathbf{A}_z = -\begin{pmatrix}
    0 & 0 & 0 & 0 & 0 & \frac 1 \rho & 0 & 0 & 0 \\
    0 & 0 & 0 & 0 & 0 & 0 & 0 & \frac 1 \rho & 0 \\
    0 & 0 & 0 & 0 & 0 & 0 & 0 & 0 & \frac 1 \rho \\
    c_{15} & c_{14} & c_{13} & 0 & 0 & 0 & 0 & 0 & 0 \\
    c_{56} & c_{46} & c_{36} & 0 & 0 & 0 & 0 & 0 & 0 \\
    c_{55} & c_{45} & c_{35} & 0 & 0 & 0 & 0 & 0 & 0 \\
    c_{25} & c_{24} & c_{23} & 0 & 0 & 0 & 0 & 0 & 0 \\
    c_{45} & c_{44} & c_{34} & 0 & 0 & 0 & 0 & 0 & 0 \\
    c_{35} & c_{34} & c_{33} & 0 & 0 & 0 & 0 & 0 & 0
    \end{pmatrix}.
\end{equation}
\end{small}


\subsection{Виды анизотропии}

\subsubsection{Орторомбическая анизотропия}
\label{раздел:орторомбическая-анизотропия}

Если материал обладает различными свойствами в~каждом из~трёх направлений, то говорят, что материал обладает
орторомбической анизотропией. В~этом приближении количество независимых компонент матрицы упругости $c_{ij}$ сокращается
до девяти:
\[
    c_{ij} = \begin{pmatrix}
    c_{11} & c_{12} & c_{13} & 0 & 0 & 0 \\
    c_{12} & c_{22} & c_{23} & 0 & 0 & 0 \\
    c_{13} & c_{23} & c_{33} & 0 & 0 & 0 \\
    0 & 0 & 0 & c_{44} & 0 & 0 \\
    0 & 0 & 0 & 0 & c_{55} & 0 \\
    0 & 0 & 0 & 0 & 0 & c_{66}
    \end{pmatrix}.
\]
В таком случае компоненты матрицы имеют наглядную интерпретацию \cite{фаворская2011постановка} (см.~рис~\ref{рис:орторомбическая-анизотропия}).

\fig{рис:орторомбическая-анизотропия}{png/anisotropy/orthorhombic.png}
    {Физическая интерпретация компонент матрицы упругости}
Оси координат, изображённые на~\ref{рис:орторомбическая-анизотропия}, соответствуют осям декартовой системы координат, в
которой задан тензор $C_{ijkl}$.

Скорость продольной волны, распространяющейся вдоль направления 1, равна $\sqrt{c_{11}/\rho}$. Скорости поперечных волн,
распространяющихся вдоль оси 1, равны $\sqrt{c_{66}/\rho}$ и $\sqrt{c_{55}/\rho}$ соответственно для случаев, когда
движение среды происходит вдоль направлений 2 и 3. Скорость продольной волны, распространяющейся вдоль направления 2,
равна $\sqrt{c_{22}/\rho}$. Скорости поперечных волн, распространяющихся вдоль оси 2, равны $\sqrt{c_{66}/\rho}$ и
$\sqrt{c_{44}/\rho}$ соответственно для случаев, когда движение среды происходит вдоль направлений 1 и 3. Скорость
продольной волны, распространяющейся вдоль направления 3, равна $\sqrt{c_{33}/\rho}$. Скорости поперечных волн,
распространяющихся вдоль оси 1, равны $\sqrt{c_{55}/\rho}$ и $\sqrt{c_{44}/\rho}$ соответственно для случаев, когда
движение среды происходит вдоль направлений 1 и 2.

При орторомбической анизотропии матрицы $\mathbf{A_i}$ в~уравнениях \eqref{матричные_уравнения_мдтт} имеют вид:
\begin{small}
\[
    \mathbf{A}_x = -\begin{pmatrix}
    0 & 0 & 0 & \frac 1 \rho & 0 & 0 & 0 & 0 & 0 \\
    0 & 0 & 0 & 0 & \frac 1 \rho & 0 & 0 & 0 & 0 \\
    0 & 0 & 0 & 0 & 0 & \frac 1 \rho & 0 & 0 & 0 \\
    c_{11} & 0 & 0 & 0 & 0 & 0 & 0 & 0 & 0 \\
    0 & c_{66} & 0 & 0 & 0 & 0 & 0 & 0 & 0 \\
    0 & 0 & c_{55} & 0 & 0 & 0 & 0 & 0 & 0 \\
    c_{12} & 0 & 0 & 0 & 0 & 0 & 0 & 0 & 0 \\
    0 & 0 & 0 & 0 & 0 & 0 & 0 & 0 & 0 \\
    c_{13} & 0 & 0 & 0 & 0 & 0 & 0 & 0 & 0
    \end{pmatrix},
\]
\[
    \mathbf{A}_y = -\begin{pmatrix}
    0 & 0 & 0 & 0 & \frac 1 \rho & 0 & 0 & 0 & 0 \\
    0 & 0 & 0 & 0 & 0 & 0 & \frac 1 \rho & 0 & 0 \\
    0 & 0 & 0 & 0 & 0 & 0 & 0 & \frac 1 \rho & 0 \\
    0 & c_{12} & 0 & 0 & 0 & 0 & 0 & 0 & 0 \\
    c_{66} & 0 & 0 & 0 & 0 & 0 & 0 & 0 & 0 \\
    0 & 0 & 0 & 0 & 0 & 0 & 0 & 0 & 0 \\
    0 & c_{22} & 0 & 0 & 0 & 0 & 0 & 0 & 0 \\
    0 & 0 & c_{44} & 0 & 0 & 0 & 0 & 0 & 0 \\
    0 & c_{23} & 0 & 0 & 0 & 0 & 0 & 0 & 0
    \end{pmatrix},
\]
\[
    \mathbf{A}_z = -\begin{pmatrix}
    0 & 0 & 0 & 0 & 0 & \frac 1 \rho & 0 & 0 & 0 \\
    0 & 0 & 0 & 0 & 0 & 0 & 0 & \frac 1 \rho & 0 \\
    0 & 0 & 0 & 0 & 0 & 0 & 0 & 0 & \frac 1 \rho \\
    0 & 0 & c_{13} & 0 & 0 & 0 & 0 & 0 & 0 \\
    0 & 0 & 0 & 0 & 0 & 0 & 0 & 0 & 0 \\
    c_{55} & 0 & 0 & 0 & 0 & 0 & 0 & 0 & 0 \\
    0 & 0 & c_{23} & 0 & 0 & 0 & 0 & 0 & 0 \\
    0 & c_{44} & 0 & 0 & 0 & 0 & 0 & 0 & 0 \\
    0 & 0 & c_{33} & 0 & 0 & 0 & 0 & 0 & 0
    \end{pmatrix}.
\]
\end{small}

\subsubsection{Трансверсально-изотропное тело}

Другой вид анизотропного материала предполагает, что в~каждой точке пространства существует плоскость, в~которой все
направления эквивалентны, но отличаются от~свойств материала в~направлении нормали. При этом плоскости для каждой точки
тела являются параллельными. Такой тип анизотропии называется трансверсальной изотропией.

Допустив, что плоскости изотропии параллельны плоскости $XY$, а выделенное направление совпадает с~направление оси $Z$.
В таком случае матрица упругости содержит всего 5 независимых компонент:
\[
    c_{ij} = \begin{pmatrix}
    c_{11} & c_{12} & c_{13} & 0 & 0 & 0 \\
    c_{12} & c_{11} & c_{13} & 0 & 0 & 0 \\
    c_{13} & c_{13} & c_{33} & 0 & 0 & 0 \\
    0 & 0 & 0 & c_{44} & 0 & 0 \\
    0 & 0 & 0 & 0 & c_{44} & 0 \\
    0 & 0 & 0 & 0 & 0 & c_{66}
    \end{pmatrix},
\]
где компоненты $c_{11}$, $c_{12}$, $c_{66}$ связаны соотношением
\[
    c_{66} = \frac{c_{11} - c_{12}}{2}.
\]

Как видно, матрица упругости для трансверсально-изотропного тела является частным случаем матрицы упругости для тела с
орторомбической анизотропией. Соответственно, матрицы $A_i$ в~\eqref{матричные_уравнения_мдтт} имеют вид:
\begin{small}
\[
    \mathbf{A}_x = -\begin{pmatrix}
    0 & 0 & 0 & \frac 1 \rho & 0 & 0 & 0 & 0 & 0 \\
    0 & 0 & 0 & 0 & \frac 1 \rho & 0 & 0 & 0 & 0 \\
    0 & 0 & 0 & 0 & 0 & \frac 1 \rho & 0 & 0 & 0 \\
    c_{11} & 0 & 0 & 0 & 0 & 0 & 0 & 0 & 0 \\
    0 & c_{66} & 0 & 0 & 0 & 0 & 0 & 0 & 0 \\
    0 & 0 & c_{44} & 0 & 0 & 0 & 0 & 0 & 0 \\
    c_{12} & 0 & 0 & 0 & 0 & 0 & 0 & 0 & 0 \\
    0 & 0 & 0 & 0 & 0 & 0 & 0 & 0 & 0 \\
    c_{13} & 0 & 0 & 0 & 0 & 0 & 0 & 0 & 0
    \end{pmatrix},
\]
\[
    \mathbf{A}_y = -\begin{pmatrix}
    0 & 0 & 0 & 0 & \frac 1 \rho & 0 & 0 & 0 & 0 \\
    0 & 0 & 0 & 0 & 0 & 0 & \frac 1 \rho & 0 & 0 \\
    0 & 0 & 0 & 0 & 0 & 0 & 0 & \frac 1 \rho & 0 \\
    0 & c_{12} & 0 & 0 & 0 & 0 & 0 & 0 & 0 \\
    c_{66} & 0 & 0 & 0 & 0 & 0 & 0 & 0 & 0 \\
    0 & 0 & 0 & 0 & 0 & 0 & 0 & 0 & 0 \\
    0 & c_{11} & 0 & 0 & 0 & 0 & 0 & 0 & 0 \\
    0 & 0 & c_{44} & 0 & 0 & 0 & 0 & 0 & 0 \\
    0 & c_{13} & 0 & 0 & 0 & 0 & 0 & 0 & 0
    \end{pmatrix},
\]
\[
    \mathbf{A}_z = -\begin{pmatrix}
    0 & 0 & 0 & 0 & 0 & \frac 1 \rho & 0 & 0 & 0 \\
    0 & 0 & 0 & 0 & 0 & 0 & 0 & \frac 1 \rho & 0 \\
    0 & 0 & 0 & 0 & 0 & 0 & 0 & 0 & \frac 1 \rho \\
    0 & 0 & c_{13} & 0 & 0 & 0 & 0 & 0 & 0 \\
    0 & 0 & 0 & 0 & 0 & 0 & 0 & 0 & 0 \\
    c_{44} & 0 & 0 & 0 & 0 & 0 & 0 & 0 & 0 \\
    0 & 0 & c_{13} & 0 & 0 & 0 & 0 & 0 & 0 \\
    0 & c_{44} & 0 & 0 & 0 & 0 & 0 & 0 & 0 \\
    0 & 0 & c_{33} & 0 & 0 & 0 & 0 & 0 & 0
    \end{pmatrix}.
\]
\end{small}

Такой вид анизотропии наиболее часто встречается у~материалов с~гексагональным типом кристаллической решётки, а также в
случае полимерных композиционных материалов, состоящий из~матрицы, армированной однонаправленными волокнами.

\subsubsection{Изотропный случай}

Очевидно, что общие формулы для анизотропного случая могут быть легко сведены к~изотропному. Так, если свойства
материала одинаковы по~всем направлениям, то матрица упругости принимает \cite{реслер2011механическое} следующий вид:
\begin{small}
\begin{equation}
    \label{изотропная_матрица_упругости}
    c_{ij} = \begin{pmatrix}
    c_{11} & c_{12} & c_{12} & 0 & 0 & 0 \\
    c_{12} & c_{11} & c_{12} & 0 & 0 & 0 \\
    c_{12} & c_{12} & c_{11} & 0 & 0 & 0 \\
    0 & 0 & 0 & c_{44} & 0 & 0 \\
    0 & 0 & 0 & 0 & c_{44} & 0 \\
    0 & 0 & 0 & 0 & 0 & c_{44}
    \end{pmatrix},
\end{equation}
\end{small}
где компоненты $c_{11}$, $c_{12}$, $c_{44}$ связаны дополнительным соотношением
\[
    c_{44} = \frac{c_{11} - c_{12}}{2}.
\]
Эти коэффициенты могут быть выражены через рассмотренные ранее параметры:
\begin{eqnarray}
    \label{упругие_постоянные}
    c_{11} &=&  \frac{E(1-\nu)}{(1+\nu)(1-2\nu)}, \nonumber \\
    c_{12} &=&  \lambda =  \frac{E\nu}{(1+\nu)(1-2\nu)}, \\
    c_{44} &=&  \mu = \frac{E}{2(1+\nu)}. \nonumber
\end{eqnarray}

Легко заметить, что при использовании матрицы упругости вида \eqref{изотропная_матрица_упругости} и коэффициентов
\eqref{упругие_постоянные}, матрицы для анизотропного случая \eqref{анизотропная_матрица_x}--\eqref{анизотропная_матрица_z}
в~точности совпадают с~матрицами для изотропного случая \eqref{изотропная_матрица_x}--\eqref{изотропная_матрица_z}.

\subsection{Модели разрушения}
При разработке новых композиционных материалов для высокотехнологичных отраслей промышленности крайне важно надёжное
предсказание прочностных свойств изготовленных из~них деталей. В~целом, композиты обладают высокой удельной прочностью,
в том числе усталостной, и устойчивы к~погодным условиям. Тем не менее, несмотря на~обширный накопленный опыт
проектирования и эксплуатации композитных деталей, единой теории разрушения композитов, достаточно надёжной для
практического применения, на~данный момент не существует. Проектирование конструкций зачастую идёт с~избыточным запасом
прочности, что повышает вес и объем детали, существенно снижая эффективность использования композиционных материалов.
Математическое моделирование с~использованием распространённых коммерческих пакетов широко применяется при работе с
традиционными конструкционными материалами, но в~случае композитов не всегда даёт достаточно надёжные результаты для
применения на~практике.

В реальных экспериментах можно видеть, что разрушение полимерного композитного материала проявляется в~несколько этапов.
Первичным этапом разрушения является разрушение первого слоя (First Ply Failure, FPF), после него слоистый материал ещё
может нести нагрузку. Второй этап, общего разрушения слоев, называется критическим разрушением слоистого материала
(Ultimate Laminate Failure, ULF). Разрушение композиционного материала проявляется на~микромеханическом уровне вследствие
разрушения волокон, раскола матрицы, а также разрушения поверхности сопряжения или межфазной границы. Некоторые критерии
позволяют уловить FPF, но, чаще всего, регистрируют только критическое разрушение.

Как показывают результаты международного проекта WWFE (World Wide Failure Exercise), целиком посвящённого
математическому моделированию разрушения полимерных композиционных материалов (ПКМ) с~армированием длинными волокнами,
теория разрушения композитов ещё далека от~завершения \cite{hinton2004failure,kaddour2013maturity}. В~рамках проекта
рассматривались и сравнивались различные критерии, применяемые в~мировой практике для моделирования объёмного разрушения
при сложном трёхмерном статическом нагружении ПКМ. После сравнения критериев друг с~другом и с~результатами натурного
эксперимента был сделан вывод, что существующие критерии не обладают достаточно надёжной предсказательной способностью
для непосредственного инженерного применения.

В случае динамического нагружения информация о~критериях разрушения носит ещё более разрозненный характер. Как правило,
при математическом моделировании применяются те же критерии разрушения, что и в~статике. Соответственно, значения
порогов разрушения также определяются при экспериментах по~статическому нагружению, что является одной из~причин
расхождения численных экспериментов с~натурными.

Ещё одной из~проблем, появляющихся при моделировании разрушения композиционных материалов, является то, что большая
часть критериев разрушения формулируются для двухмерного случая. В~трёхмерном случае многие критерии зачастую не только
имеют сложную формулировку, но и требуют констант, которые крайне сложно получить при экспериментах.

Разработкой критериев разрушения занимается большое количество независимых друг от~друга научных коллективов
\cite{hinton2004failure}. Можно сказать, что теория разрушения композиционного материала в~наше время представляет
собой активно развивающуюся и востребованную для практического применения область науки.

В данной главе приводится описание использованных в~дальнейшей работе критериев разрушения композитов, расчёт
модельных задач, сравнение критериев друг с~другом и анализ физических механизмов, на~которых эти критерии основаны.

\subsubsection{Базовые критерии разрушения}
В данном разделе приводится краткое описание ряда классических критериев разрушения. В~исходной формулировке их
применение к~композиционным материалам не даёт совпадающего с~экспериментальными данными результата, однако на
физических механизмах, описываемых ими, были построены более сложные критерии, применимые как для осреднённо-анизотропной
модели среды, так и учитывающие специфичную волокнистую структуру ПКМ.

\paragraph{Критерий наибольших нормальных напряжений}
Согласно критерию наибольших нормальных (главных) напряжений разрушение хрупкого материала или пластическое
течение в~случае использования соответствующей модели начинается при превышении одним из~главных напряжений порогового
значения $\sigma*$\cite{надаи1954пластичность}.

Этот критерий неплохо подходит для описания разрушения изотропного материала при одноосных нагрузках, но при этом весьма
неудовлетворительно описывает явления пластических деформаций.

Для материала, имеющего различные пороги прочности на~растяжение и на~сжатие, данный критерий адаптируется легко.
Основной его недостаток в~применении к~композитам состоит в~том, что с~его помощью невозможно различать пороги прочности
в различных направлениях~--- особенно в~случае композитов, армированных длинными волокнами, для которых эти пороги
отличаются на~несколько порядков.

\paragraph{Критерий Мизеса}
Критерий Мизеса основывается на~максимальной удельной энергии изменения формы \cite{ильюшин1948пластичность} и,
согласно этому критерию, пластическое деформирование или хрупкое разрушение  наступает при достижении
удельной энергией изменения формы определённого значения порогового значения \cite{иванов1990расчёт}:
\begin{equation}
    J_2 \geq k^2,
\end{equation}
где $J_2$~--- второй инвариант девиатора, а $k$~--- предел текучести на~сдвиг. Критерий Мизеса может быть записан через
компоненты тензора напряжений следующим образом:
\begin{align}
    k^2 \leq J_2 &= \frac{1}{2}S_{ij}S_{ji} = -S_1 S_2 - S_2 S_3 - S_1 S_3 \nonumber \\
                 &= \frac{1}{6}\left( (\Sxx - \Syy)^2 + (\Syy - \Szz)^2 + (\Szz - \Sxx)^2 \right) +
                   \Sxy^2 + \Syz^2 + \Sxz^2,
\end{align}
а также через главные напряжения:
\begin{equation}
    k^2 \leq \frac{1}{6}\left( (\sigma_1 - \sigma_2)^2 + (\sigma_2-\sigma_3)^2 + (\sigma_3-\sigma_1)^2 \right).
\end{equation}

Значение эквивалентного напряжения Мизеса может быть определено следующим образом:
\begin{equation}
    \sigma_e = \sqrt{3J_2} = \sqrt \frac{(\sigma_1 - \sigma_2)^2 + (\sigma_2-\sigma_3)^2 + (\sigma_3-\sigma_1)^2}{2}.
\end{equation}

Критерий Мизеса удовлетворительно описывает поведение изотропных материалов при сдвиге: разрушение хрупких материалов
и переход пластичных материалов в~пластическую фазу. Аналогично критерию главных напряжений, критерий Мизеса в~исходной
формулировке для анизотропного материала не применяется. В~данной главе он приведён, так как по~аналогии с~ним строятся
более сложные критерии, способные учесть анизотропное поведение материала.

\paragraph{Критерий Мора-Кулона}
Особенностью критерия Мора-Кулона является тот факт, что этот критерий изначально разработан для применения к
материалам, обладающим разными пределами прочности при сжатии и при растяжении \cite{надаи1954пластичность,coulomb1776essai}.
Критерий учитывает как нормальные, так и касательные напряжения.

Критерий Мора-Кулона основан на~теории сухого трения Кулона, а именно на~предположении о~том, что
материал разрушается вследствие возникновения внутреннего скольжения в~среде. Критерий предполагает, что скольжение
возникает на
площадках, проходящих через ось $\sigma_2$ в~пространстве главных напряжений, но при этом предполагается, что напряжение
$\sigma_2$ не влияет на~возникновение скольжения.

Критерий Мора-Кулона вводит следующую связь между нормальным и касательным напряжениями:
\begin{equation}
    \tau_n = \sigma_n \tan \phi + c,
    \label{критерия_мора}
\end{equation}
где $\sigma_n$~--- нормально напряжение, $\tau_n$~--- касательное напряжение, $\phi$~--- угол внутреннего трения, $c$
--- коэффициент когезии.

Для кругов Мора верны равенства:
\begin{align}
    \sigma_n &= \sigma_m - \tau_m \sin \phi, \nonumber \\
    \tau_n  &= \tau_m \cos \phi,
\end{align}
где
\begin{align}
    \tau_m = \frac{\sigma_1-\sigma_3}{2}, \nonumber \\
    \sigma_m = \frac{\sigma_1+\sigma_3}{2}.
\end{align}

Таким образом, критерий может быть альтернативно записан в~форме:
\begin{equation}
    \tau_m = \sigma_m\sin \phi + c\cos \phi.
\end{equation}

В трёхмерном случае критерий имеет более сложный вид:
\begin{align}
    \pm\frac{\sigma_1-\sigma_2}{2} &= \frac{\sigma_1+\sigma_2}{2}\sin\phi + c\cos \phi, \nonumber \\
    \pm\frac{\sigma_2-\sigma_3}{2} &= \frac{\sigma_2+\sigma_3}{2}\sin\phi + c\cos \phi, \nonumber \\
    \pm\frac{\sigma_3-\sigma_1}{2} &= \frac{\sigma_3+\sigma_1}{2}\sin\phi + c\cos \phi.
    \label{трёхмерный_критерий_кулона}
\end{align}

Данный критерий применим к~анизотропным материалам в~формулировке \eqref{трёхмерный_критерий_кулона}, если для каждого из
направлений взять свои константы. Основным препятствием к~использованию критерия Мора-Кулона в~данной работе было
отсутствие требуемых специфичных констант для моделируемого материала, восстановление по~имеющимся данным привело бы к
возникновению погрешности, сравнимой с~полученными значениями. Измерение этих констант требует отдельных
экспериментальных исследований.

\paragraph{Критерий Друкера-Прагера}
Критерий Друкера-Прагера описывает поведение материалов, прочность которых зависит от~гидростатического давления
\cite{drucker1952soil} и задаётся следующей формулой:
\begin{equation}
    \sqrt J_2 = A + B _1,
\end{equation}
где $I_1$~--- первый инвариант тензора напряжений, а $J_2$~--- второй инвариант девиатора тензора напряжений. Константы
$A$, $B$ определяются экспериментально.

Критерий может быть переписан с~использованием главных напряжений:
\[
    \sqrt{\frac{1}{6}\left( (\sigma_1-\sigma_2)^2+(\sigma_2-\sigma_3)^2+(\sigma_3-\sigma_1)^2 \right)}=
    A+B(\sigma_1+\sigma_2+\sigma_3).
\]

Для случая одноосного растяжения критерий принимает вид
\[
    \sigma_t = \sqrt{3}(A+B\sigma_t),
\]
а для одноосного сжатия вид
\[
    \sigma_c = \sqrt{3}(A-B\sigma_t).
\]

Из этих уравнений можно получить связь параметров $A$, $B$ с~пределами прочности при одноосном сжатии и растяжении:
\begin{align}
    A &= \frac{2}{\sqrt 3}\frac{\sigma_c\sigma_t}{\sigma_c+\sigma_t}, \nonumber \\
    B &= \frac{1}{\sqrt 3}\frac{\sigma_t-\sigma_c}{\sigma_c+\sigma_t}.
\end{align}

Критерий Друкера-Прагера был разработан для описания пластических деформаций глинистых грунтов, также он может
применяться для описания разрушения скальных грунтов, бетона, полимеров, пены и других, зависящих от~давления,
материалов. Аналогично вышеописанным критериям, он может быть расширен на~применение к~анизотропным средам.

\subsubsection{Критерии разрушения для композитов}
\label{раздел:критерии-для-композитов}

\paragraph{Критерий Цая-Хилла}
Данный критерий является обобщением критерия Мизеса на~случай анизотропного материала и при равенстве характеристик
прочности материала во~всех направлениях обращается в~критерий Мизеса.

Оригинальный критерий был предложен Р. Хиллом в~1948 г. как критерий перехода анизотропных металлов в~пластическое
состояние \cite{hill1948theory}. Аналогично критерию Мизеса, переход в~пластическое состояние зависит только от~компонент
девиатора тензора напряжений и не зависит от~шаровой части. Для трёхмерного случая он формулируется следующим образом:
\begin{small}
\begin{gather}
    F(\sigma_{22}-\sigma_{33})^2+G(\sigma_{11}-\sigma_{33})^2+H(\sigma_{22}-\sigma_{11})^2+ \nonumber \\
    +2L\sigma_{23}^2+2M\sigma_{31}^2+2N\sigma_{12}^2 = 1, \nonumber \\
    F=\frac{1}{2}\left( \frac{1}{\left( \sigma_2^y \right)^2}+
                        \frac{1}{\left( \sigma_3^y \right)^2}-
                        \frac{1}{\left( \sigma_1^y \right)^2}\right), \nonumber \\
    G=\frac{1}{2}\left( \frac{1}{\left( \sigma_1^y \right)^2}+
                        \frac{1}{\left( \sigma_3^y \right)^2}-
                        \frac{1}{\left( \sigma_2^y \right)^2}\right), \\
    H=\frac{1}{2}\left( \frac{1}{\left( \sigma_2^y \right)^2}+
                        \frac{1}{\left( \sigma_1^y \right)^2}-
                        \frac{1}{\left( \sigma_3^y \right)^2}\right), \nonumber \\
    L=\frac{1}{2\left( \tau_{23}^y \right)^2},
    M=\frac{1}{2\left( \tau_{31}^y \right)^2},
    N=\frac{1}{2\left( \tau_{12}^y \right)^2}. \nonumber
\end{gather}
\end{small}
Здесь $\sigma_1^y$,$\sigma_2^y$,$\sigma_3^y$~--- пороги разрушения при нормальном нагружении, $\tau_{12}^y$,$\tau_{23}^y$,
$\tau_{31}^y$~--- при сдвиговом.

Критерий Хилла дополнялся многими авторами, в~том числе и самим Р. Хиллом, главным образом для двухмерного случая.
С. Цай предложил использовать данный критерий не только для перехода металла в~пластическое состояние, но и для хрупкого
разрушения анизотропного материала.

Основным недостатком данного критерия является то, что он не различает механизмы разрушения на~сжатие и на~растяжения.
Изначально данный критерий разрабатывался для металлов, в~которых значения порога прочности для сжатия и растяжения
совпадают. В~композиционных материалах пороги прочности могут отличаться на~несколько порядков~--- соответственно,
достоверность расчётов с~использованием данного критерия вызывает сомнения.

\paragraph{Критерий Цая-Ву}
Критерий Цая-Ву учитывает полную энергию деформации: энергию формоизменения и энергию расширения. Он является более
полным, чем критерий Цая-Хилла, и способен различать пределы прочности на~сжатие и растяжение. Если материал обладает
трансверсальной изотропией (одна выделенная ось Х), то критерий записывается следующим образом \cite{tsai1971general}:

\begin{small}
\begin{gather}
    F_1\Sxx + F_2\Syy + F_3\Szz + F_{11}{\Sxx}^2 + \nonumber \\
    + F_{22}{\Syy}^2 + F_{33}{\Szz}^2 + F_{44}{\Syz}^2 + F_{55}{\Sxz}^2 + F_{66}{\Sxy}^2 + \nonumber \\
    + 2F_{12}\Sxx\Syy+2F_{13}\Sxx\Szz+2F_{23}\Syy\Szz = 1, \\
    F_1 \ \left( \frac{1}{X_1^T} - \frac{1}{X_1^C} \right),
    F_2=F_3=\left( \frac{1}{X_2^T} - \frac{1}{X_2^C} \right), \nonumber \\
    F_{11}=\frac{1}{X_1^T X_1^C},
    F_{22}=F_{33}=\frac{1}{X_2^T X_2^C}, \nonumber \\
    F_{44}=\frac{1}{(s_{23})^2},
    F_{55}=F_{66}=\frac{1}{(s_1)^2}, \nonumber \\
    F_{12}=-\frac{1}{2}\frac{1}{X_2^T X_2^C},
    F_{23}=F_{13}=-\frac{1}{2}\sqrt{\frac{1}{X_1^T X_1^C X_2^T X_2^C}} \nonumber.
\end{gather}
\end{small}
Здесь $X_1^T$~--- предел прочности слоистого материала при растяжении вдоль направления волокон, $X_1^C$ -- предел
прочности слоистого материала при сжатии вдоль направления волокон, $X_2^T$~--- предел прочности слоистого материала при
растяжении поперёк направления волокон, $X_2^C$~--- предел прочности слоистого материала при сжатии поперёк направления
волокон, $S_{23}$~--- предел прочности слоистого материала при сдвиге в~плоскости, перпендикулярной направлению волокон,
$S_1$~--- предел прочности слоистого материала при сдвиге в~плоскости, параллельной направлению волокон.

Критерий Цая-Ву является частным случаем обобщённого критерия Хилла \cite{hill1979theoretical,abrate2008criteria},
специально приспособленным для моделирования разрушения композитов.

\paragraph{Критерий Друкера-Прагера}
Обобщение критерия Друкера-Прагера на анизотропный случай было предложено в~статье О. Казаку и Ф. Барлата
\cite{cazacu2001generalization}. В~той же статье приводятся значения параметров критерия для некоторых металлов и сплавов.
Препятствием для использования критерия Казаку-Барлата в~данной работе стало отсутствие значения данных параметров для
каких-либо полимерных композитов.

Другая формулировка обобщения данного критерия основана на~обобщённом  критерии Хилла и приводится в~статье \cite{liu1997asymmetric}.
Именно эта формулировка используется в~данной работе, так как использует те же пределы прочности, что и вышеуказанный
критерий Цая-Ву:
\newpage
\begin{small}
\begin{gather}
    \sqrt{F(\sigma_{22}-\sigma_{33})^2 + G(\sigma_{11}-\sigma_{33})^2 + H(\sigma_{22}-\sigma_{11})^2+
    2L\sigma_{23}^2+2M\sigma_{31}^2+2N\sigma_{12}^2} + \nonumber \\
    + I\sigma_{11}+J\sigma_{22}+K\sigma_{33} = 1,\\
    F=\frac{1}{2}(\Sigma_2^2+\Sigma_3^2-\Sigma_1^2),
    G=\frac{1}{2}(\Sigma_1^2+\Sigma_3^2-\Sigma_2^2),
    H=\frac{1}{2}(\Sigma_1^2+\Sigma_2^2-\Sigma_3^2), \nonumber \\
    L=\frac{1}{2\left( \sigma_{23}^y \right)^2},
    M=\frac{1}{2\left( \sigma_{31}^y \right)^2},
    N=\frac{1}{2\left( \sigma_{12}^y \right)^2}, \nonumber \\
    I=\frac{\sigma_{1c}-\sigma_{1t}}{2\sigma_{1c}\sigma_{1t}},
    J=\frac{\sigma_{2c}-\sigma_{2t}}{2\sigma_{2c}\sigma_{2t}},
    K=\frac{\sigma_{3c}-\sigma_{3t}}{2\sigma_{3c}\sigma_{3t}}, \nonumber \\
    \Sigma_1=\frac{\sigma_{1c}+\sigma_{1t}}{2\sigma_{1c}\sigma_{1t}},
    \Sigma_2=\frac{\sigma_{2c}+\sigma_{2t}}{2\sigma_{2c}\sigma_{2t}},
    \Sigma_3=\frac{\sigma_{3c}+\sigma_{3t}}{2\sigma_{3c}\sigma_{3t}}, \nonumber \\
\end{gather}
\end{small}
где $\sigma_{23}^y$,$\sigma_{31}^y$,$\sigma_{12}^y$~--- пределы прочности при чистом сдвиге,
$\sigma_{1c}$,$\sigma_{2c}$,$\sigma_{3c}$~--- пределы прочности при одноосном сжатии по~трём главным направлениям
анизотропии, $\sigma_{1t}$,$\sigma_{2t}$,$\sigma_{3t}$~--- пределы прочности при одноосном растяжении.

\paragraph{Критерий Хашина}
Хашин предположил \cite{hashin1973fatigue,batra2012damage}, что композит с~эпоксидной матрицей и наполнителем из~волокон может быть промоделирован
как однородное тело с~ортотропной анизотропией --- одним выделенным направлением, вдоль которого свойства материала
отличаются. Выделенное направление совпадает с~направлением укладки волокон. Данный критерий учитывает различные
механизмы разрушения композиционного материала. Разрушение появляется, когда срабатывает одно из~пяти условий, каждое
из которых характеризует определённый механизм разрушения.

Растяжение и сжатие волокон, расслоение:
\begin{small}
\begin{gather}
    f_1 \equiv \sqrt{\left( \frac{\sigma_{11}}{X_T} \right)^2 + \frac{\sigma_{12}^2+\sigma_{13}^2}{S^2}}, \sigma_{11}>0, \nonumber \\
    f_2 \equiv \frac{\abs{\sigma_{11}}}{X_C}, \sigma_{11} < 0, \nonumber \\
    f_3 \equiv \frac{\abs{\sigma_{33}}}{Z_C}, \sigma_{33} < 0.
\end{gather}
\end{small}

Растяжение и сжатие матрицы:
\begin{small}
\begin{gather}
    f_4 \equiv \sqrt{\left( \frac{\sigma_{22}+\sigma_{33}}{Y_T} \right)^2+
                     \frac{\sigma_{23}^2-\sigma_{22}\sigma_{33}}{S_T^2}+
                     \frac{\sigma_{12}^2-\sigma_{13}^2}{S^2}}, \sigma_{22}+\sigma_{33} > 0, \\
    f_5 \equiv \sqrt{\left( \frac{Y_C}{2S_T}-1 \right)^2\left( \frac{\sigma_{22}+\sigma_{33}}{Y_C} \right)+
                     \left( \frac{\sigma_{22}+\sigma_{33}}{4S_T} \right)^2 +
                     \frac{\sigma_{23}^2-\sigma_{22}\sigma_{33}}{S_T^2}+
                     \frac{\sigma_{12}^2-\sigma_{13}^2}{S^2}}, \sigma_{22}+\sigma_{33} < 0.
\end{gather}
\end{small}

Здесь ось 1 направлена вдоль укладки волокон, $X_T$ и $X_C$~--- прочность на~растяжение и сжатие вдоль оси 1,
$Z_C$~--- прочность соединения слоев, $Y_T$ и $Y_C$~--- прочность на~растяжение и сжатие поперёк оси 1,
$S_T$~--- сдвиговая прочность в~плоскости, перпендикулярной оси 1, остальные сдвиговые прочности~--- $S$.
Равенство какой-либо из~функций $f_1$ -- $f_5$ единице означает срабатывание условия.

\paragraph{Критерий Пака}
Теория Пака \cite{puck1969failure,puck2002failure} основана на~теории Мора-Кулона и по~заложенным в~ней механическим принципам наиболее близка
к реальной картине разрушения композиционного материала. Основной механизм разрушения~--- возникновение трещин.
Предполагается, что существует шесть основных типов разрушения композиционного материала (см. рис.~\ref{рис:критерий-пака})
и для модели требуется, соответственно, шесть констант для порогового напряжения: $R_{\|}^C$, $R_{\|}^T$,
$R_\bot^C$, $R_\bot^T$, $R_{\bot\bot}$, $R_{\bot\|}$ . Эти шесть типов разбиваются на~два класса: разрушение волокон и
разрушение матрицы.

\figfull{рис:критерий-пака}{png/destruction/puck-types.png}
    {Основные типы разрушения: FF – разрушение волокон (fiber failure), IFF – разрушение матрицы (interfiber failure)}

Разрушение волокон (FF) происходит при выполнении одного из~следующих условий:
\begin{align}
    \frac{\sigma_1}{R_{\|}^T} = 1, \sigma_1 > 0, \nonumber \\
    \frac{\sigma_1}{-R_{\|}^C} = 1, \sigma_1 < 0.
\end{align}

Разрушение матрицы происходит по~более cложному механизму. Опираясь на~экспериментальный опыт, Пак разделяет трещины,
образующиеся в~матрице, на~три типа: $A$, $B$ и $C$.

Трещины типа $А$ появляются при растяжении материала поперёк волокон при нагружении в~направлении, перпендикулярном
толщине слоя. Математическая формулировка выглядит следующим образом:
\begin{equation}
    \sqrt{\left( \frac{\tau_{21}}{R_{\bot\|}^A} \right)^2+
          \left( 1-\frac{p_{\bot\|}^{(+)}}{R_{\bot\|}^A}R_\bot^{(+)A} \right)^2
          \left( \frac{\sigma_2}{R_\bot^{(+)A}} \right)^2}+\frac{p_{\bot\|}^{(+)}}{R_{\bot\|}^A}\sigma_2=1,
    \sigma_n \geq 0,
\end{equation}
где $\sigma_2$ и $\tau_{21}$~--- наибольшее главное и сдвиговое напряжения в~плоскости, перпендикулярной направлению
волокон, $R_{\bot\|}^A$~--- порог прочности в~плоскости трещины при трансверсально-параллельном нагружении,
$R_\bot^{(+)A}$~--- порог прочности в~плоскости трещины при трансверсальном растягивающем нагружении,
$p_{\bot\|}^{(+)}$~--- внутренний параметр модели, наклон огибающей. Образующиеся по~данному механизму трещины
наклонены к~направлению толщины под углом 0\degree .

При сжатии материала появляются трещины типа $В$, также наклонённые под углом 0\degree к~направлению толщины:
\begin{equation}
    \sqrt{\left( \frac{\tau_{21}}{R_{\bot\|}^A} \right)^2+
          \left( \frac{p_{\bot\|}^{(-)}}{R_{\bot\|}^A} \right)^2\sigma_2^2}+
    \frac{p_{\bot\|}^{(-)}}{R_{\bot\|}^A} = 1, \sigma_n < 0,
\end{equation}
где $p_{\bot\|}^{(-)}$~--- внутренний параметр модели, наклон огибающей.

Третий механизм, объясняющийся сдвиговыми напряжениями, вызывает трещины, расположенные под углом $\approx45\degree$
(тип $С$). Точное значение угла определяется по~формуле
\begin{equation}
    \theta_{fp} = \arccos \sqrt{\frac{1}{2\left( 1+p_{\bot\bot}^{(-)} \right)}
                                \left( 1+\left( \frac{\tau_{21}}{\sigma_2} \right)^2
                                         \left( \frac{R_{\bot\bot}^A}{S_{21}} \right)^2 \right)},
\end{equation}
где $p_{\bot\bot}^{(-)}$~--- внутренний параметр модели, наклон огибающей, $S_{21}$~--- сдвиговая прочность при
трансверсально-параллельном нагружении.

После определения значения угла наклона трещины можно найти напряжения в~плоскости трещины (см. рис.
\ref{рис:направления-в-критерии-пака}):
\begin{gather}
    \sigma_n(\theta) = \sigma_2\cos^2\theta + \sigma_3\sin^2\theta + 2\tau_{23}\sin\theta\cos\theta, \nonumber \\
    \tau_{nt}(\theta) = (\sigma_3-\sigma_2)\sin\theta\cos\theta + \tau_{23}(\cos^2\theta-\sin^2\theta), \nonumber \\
    \tau_{n1}(\theta) = \tau_{31}\sin\theta + \tau_{21}\cos\theta.
\end{gather}

\figfull{рис:направления-в-критерии-пака}{png/destruction/puck-directions.png}
    {Обозначения направлений и напряжений в~теории Пака}

Далее вводятся следующие обозначения:
\begin{gather}
    \sin^2\psi=\frac{\tau_{n1}^2}{\tau_{nt}^2+\tau_{n1}^2}, \nonumber \\
    \cos^2\psi=\frac{\tau_{nt}^2}{\tau_{nt}^2+\tau_{n1}^2}, \nonumber \\
    R_{\bot\bot}^A=\frac{Y_C}{2(1+p_{\bot\bot}^{(-)})}, \nonumber \\
    \frac{p_{\bot\psi}^{(+)}}{R_{\bot\psi}^A}=\frac{p_{\bot\bot}^{(+)}}{R_{\bot\bot}^A}\cos^2\psi+
                                              \frac{p_{\bot\|}^{(+)}}{R_{\bot\|}^A}\sin^2\psi \nonumber, \\
    \frac{p_{\bot\psi}^{(-)}}{R_{\bot\psi}^A}=\frac{p_{\bot\bot}^{(-)}}{R_{\bot\bot}^A}\cos^2\psi+
                                              \frac{p_{\bot\|}^{(-)}}{R_{\bot\|}^A}\sin^2\psi.
\end{gather}

После введения всех обозначений можно сформулировать критерий возникновения трещины типа $С$:
\begin{small}
\begin{align}
    \sqrt{\left( \left( \frac{1}{R_\bot^{(+)A}}-\frac{p_{\bot\psi}^{(+)}}{R_{\bot\psi}^A} \right) \sigma_n(\theta) \right)^2+
          \left( \frac{\tau_{nt}(\theta)}{R_{\bot\bot}^A} \right)^2+\left( \frac{\tau_{n1}(\theta)}{R_{\bot\|}^A} \right)^2}+
    \frac{p_{\bot\psi}^{(+)}}{R_{\bot\psi}^A}=1, \sigma_n \geq 0, \\
    \sqrt{\left( \frac{\tau_{nt}(\theta)}{R_{\bot\bot}^A} \right)^2+\left( \frac{\tau_{n1}(\theta)}{R_{\bot\|}^A} \right)^2+
          \left( \frac{p_{\bot\psi}^{(-)}}{R_{\bot\psi}^A}\sigma_n(\theta) \right)^2 }+
    \frac{p_{\bot\psi}^{(-)}}{R_{\bot\psi}^A}=1, \sigma_n < 0.
\end{align}
\end{small}

Особым преимуществом критерия Пака в~сравнении с~остальными существующими критериями является то, что он предсказывает
направление расположения микротрещин в~композиционном материале, что существенно повышает его предсказательную способность.
Однако существенным минусом является наличие четырех внутренних параметров модели, наклонов огибающей
$p_{\bot\bot}^{(-)},p_{\bot\bot}^{(+)},p_{\bot\|}^{(-)},p_{\bot\|}^{(+)}$. Значения этих параметров невозможно
получить напрямую из~эксперимента, а их физический смысл не является очевидным. От эксперимента к~эксперименту значения
данных параметров могут отличаться, что снижает эффективность данного критерия.

\subsubsection{Адгезионная прочность}
Многие современные исследования динамической нагрузки ПКМ используют различные осреднённые модели композиционного
материала. Вместо представления о~субпакете композита как о~слоистой анизотропной структуре вводятся эффективные
осреднённые прочностные характеристики изучаемого материала. Этот подход во~многом оправдан: с~одной стороны, строгие
математические построения осреднённых моделей ПКМ позволили более глубоко понять процессы, происходящие в~них, с~другой~---
исследование осреднённых структур в~вычислительном плане более дешёвая задача по~сравнению с~расчётом, явно учитывающим
всю структуру субпакета композитной панели. Тем не менее, осреднение слоистых структур не позволяет наблюдать эффекты
расслоения, которые в~свою очередь могут серьезным образом сказаться на~прочностных характеристиках ПКМ.

Эффекты расслоения слоистого материала, в~смысле разрушения контакта между слоями, а также разрушения контакта между
матрицей и армирующими волокнами ПКМ относятся к~понятию адгезионной прочности материала.

\subsubsection{Механика разрушенной области}
Критерии разрушения, описанные в~предыдущих разделах, позволяют дать ответ на~вопрос, разрушен ли материал в~данной
точке. При этом, очевидно, разрушенный материал должен обладать отличными от~неразрушенного материала свойствами. В~данной
работе используются две меры разрушенности материала, по~разному влияющие на~его свойства: дискретная скалярная и
дискретная векторная мера.

Дискретная скалярная мера предполагает, что после разрушения материал теряет свойство сопротивляться нагрузкам по~всем
направлениям. Такой материал не способен держать любые растягивающие напряжения, а также любые сдвиговые нагрузки.
Соответственно, для моделирования такого поведения у~разрушенного материала обнуляется девиатор, а также отрицательная
часть гидростатического давления в~шаровой составляющей тензора деформаций.

Дискретная векторная мера предполагает возникновение микротрещин в~материале (модель Майнчена-Сака
\cite{олдер1967вычислительные}), которые имеют определённую ориентацию в~пространстве, поэтому разрушенный материал
теряет свойства сопротивляться нагрузке по-разному для разных направлений. Так, материал с~трещиной не может
сопротивляться растягивающим напряжениям в~направлении нормали к~плоскости возникновения трещины, а также сдвиговым
нагрузкам в~этой плоскости. Для моделирования этого эффекта используется обнуление соответствующих компонент тензора
деформаций в~системе координат, у~которой ось $X$ совпадает с~направлением нормали к~плоскости трещины.

\subsection{Моделирование больших деформаций}
Описанная ранее математическая модель вещества удовлетворительно описывает процессы, протекающие в~деформируемом твёрдом
теле, только в~приближении малых деформаций. При высокоскоростных соударениях, сопровождающихся большими деформациями
области интегрирования, требуется использование более сложных моделей. Для расчётов задач такого типа (например, методом
сглаженных частиц) в~этой работе используется следующая упруго-пластическая модель изотропного вещества \cite{потапов2009диссертация}.
Законы сохранения массы, импульса и энергии записываются в~следующем виде:
\begin{align}
    \label{уравнения_движения_в_частицах}
    \frac{d\rho}{dt} + \rho \PD{u_\alpha}{x_\alpha} = 0, \nonumber \\
    \frac{du_\alpha}{dt}-\frac{1}{\rho}\PD{\sigma_{\alpha\beta}}{x_\beta} = 0, \\
    \frac{de}{dt} = \frac{1}{\rho}\sigma_{\alpha\beta}\varepsilon_{\alpha\beta}, \nonumber
\end{align}
где $\rho$~--- плотность вещества, $u_\alpha$~--- компоненты вектора скорости, $\sigma_{\alpha\beta}$~--- тензор напряжений,
$\varepsilon$~--- внутренняя энергия, $\varepsilon_{\alpha\beta}=\frac{1}{2}\left( \PD{u_\alpha}{x_\beta} + \PD{u_\beta}{x_\alpha}\right)$,
$\frac{d}{dt}$~--- субстанциональная производная.

При этом реологические соотношения записываются в~гипоупругой форме с~учетом Яумановских членов в~производной по~времени:
\begin{equation}
    \label{реологические_соотношения_в_частицах}
    \frac{dS_{\alpha\beta}}{dt} = 2\mu \left( \varepsilon_{\alpha\beta}-\frac{1}{3}\delta_{\alpha\beta}\varepsilon_{\alpha\beta} \right) +
                                  S_{\alpha\gamma}R_{\beta\gamma}+S_{\gamma\beta}R_{\alpha\gamma},
\end{equation}
где $S_{\alpha\beta}$~--- девиатор тензора напряжений, $\mu$~--- упругая константа Ламе, а $R_{\alpha\beta}$ выражается следующим образом:
\[
    R_{\alpha\beta} = \frac{1}{2}\left( \PD{u_\alpha}{x_\beta} - \PD{u_\beta}{x_\alpha}\right).
\]

Для описания пластических течений используется теория Прандтля-Рейса совместно с~критерием Мизеса, который был
рассмотрен ранее. Для учёта пластических эффектов в~правую часть реологических соотношений
\eqref{реологические_соотношения_в_частицах} добавляется дополнительный член
$-\theta(s)(S_{\alpha\beta}\varepsilon_{\alpha\beta})S_ {\alpha\beta}$, где
\[
    \theta(s)=\left\{\begin{aligned}
        0, & при s<2K^2, \nonumber \\
        0, & при s=2K^2, S_{\alpha\beta}\varepsilon_{\alpha\beta} \leq 0, \nonumber \\
        \frac{\mu}{K^2}, & при s=2K^2, S_{\alpha\beta}\varepsilon_{\alpha\beta} > 0, \nonumber
    \end{aligned}\right.
\]
где $s=S_{\alpha\beta}S_{\alpha\beta}$, $K$~--- предел текучести на~сдвиг.

Также система \eqref{уравнения_движения_в_частицах} дополняется следующим уравнением состояния вещества \cite{monaghan1988introduction}
\[
    P = P(\rho,e) = \frac{K}{n}\left( \left( \frac{\rho}{\rho_0} \right)^n -1  \right),
\]

где константы $K$ и $n$ определяются экспериментальным путём.

\end{document}
