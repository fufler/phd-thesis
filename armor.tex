\documentclass[thesis.tex]{subfiles}

\begin{document}

\subsection[Поведение слоистой прозрачной конструкции\\при~динамическом нагружении]{Поведение слоистой прозрачной конструкции при~динамическом нагружении}

\subsubsection{Постановка}

Рассматривается задача о~высокоскоростном столкновении металлического ударника в~форме шара и многослойной стеклянной
конструкции. Решение этой задачи имеет практический смысл: такая постановка возникает как в~гражданском авиастроении
(прохождение самолёта через град), так и в~военном (взрыв шариковой бомбы).

Численное решение аналогичной двумерной постановки рассматривается в~статье \cite{петров2003численное}. В~данной работе
постановка является трёхмерной, а также рассматривается другой способ укладки слоёв: слои стекла и клея чередуются,
в~качестве подложки используется слой пластика. Также рассматривается влияние адгезионной прочности на характер
объёмного разрушения и размеров области деламинации.

\begin{figure}[h]
    \tikzset{every picture/.style={scale=1.0}}
    \begin{center}
        \subfile{tikz/armor}
    \end{center}
    \caption{Постановка задачи о~столкновении шарика с~многослойной стеклянной конструкцией.}
    \label{рис:броня}
\end{figure}

Конструкция состоит из~7~слоёв различной толщины~(см. рис.~\ref{рис:броня}): стекло (10~мм), клей (2~мм),
стекло (5~мм), клей (2~мм), стекло (5~мм), клей (2~мм), пластик (10~мм).

Параметры материалов приведены в~табл.~\ref{таб:свойства-брони}, адгезионная прочность (т.е. предельное давление,
вызывающее расслоение) равна 10 МПа. В~таблице отсутствуют пределы прочности для клея и стали, так
как хрупкое разрушение для этих материалов не моделировалось в~силу того, что оно не наблюдается на~практике в~подобных
постановках.

\begin{table}[ht]
    \centering
    \begin{tabular}{|c|c|c|c|c|}
        \hline
        & $\lambda$, ГПа & $\mu$, ГПа & $\rho$, $\textnormal{кг/м}^3$ & Предел прочности, ГПа \\
        \hline
        Стекло  & 22.5 & 28.7 & 2500 & 1 \\
        \hline
        Клей    & 8.29 & 2.07 & 1200 & --- \\
        \hline
        Пластик & 2.12 & 1.42 & 1200 & 1.5 \\
        \hline
        Сталь   & 121  & 81   & 7800 & --- \\
        \hline
    \end{tabular}
    \caption{Параметры материалов слоистой конструкции}
    \label{таб:свойства-брони}
\end{table}

Скорость столкновения шарика и стеклянной конструкции составляет 1~км/с. При моделировании процесса учитывается только
хрупкое разрушение, возникающее в~ходе взаимодействия волновых фронтов внутри материала. Деформация конструкции
вследствие столкновения не учитывается. Такое предположение вполне корректно, поскольку скорость звука в~слоях
конструкции в~несколько раз превосходит скорость соударения.

Для моделирования хрупкого разрушения в~объёме конструкции используется критерий наибольшего главного напряжения и
однобереговая модель трещин Майнчена-Сака, для расчёта разрушаемого контакта используется критерий разрушения по
превышению адгезионной прочности.

Задача решается численно при помощи сеточно-характеристического метода на~неструктурированной
тетраэдральной сетке ($\sim{200}$ тысяч узлов).

\subsubsection{Результаты расчётов}

Результаты расчётов представлены на~рис \ref{рис:броня-10-0}--\ref{рис:броня-10-200}. Каждый рисунок состоит из~трёх
частей: изображение разрушенной области (a), изображение расслоения в~разрезе конструкции (b) и область деламинации
между первым слоем стекла и первым слоем клея (c).

Один шаг по~времени равен 50 нс.

По представленным результатам видно, что столкновение стального шарика и слоистой стеклянной конструкции вызывает
серьёзные повреждения последней. Так, область основного разрушения сконцентрирована в~первом слое, который принимает на
себя изначальный удар~--- этот слой практически полностью разрушается в~результате соударения. Более того, происходит
расслоение всех контактирующих (склеенных) поверхностей, что на~практике приведёт или к~отрыву частей конструкции, или,
как минимум, серьёзно скажется на~её прочности и оптической прозрачности.

\threefigsH{рис:броня-10-0}{Разрушение в~стеклянной конструкции, 0-й шаг}
           {png/armor/10/image001-crack.png}{Разрушение}
           {png/armor/10/image001-failed-contact.png}{Расслоение}
           {png/armor/10/image001-failed-contact-delamination.png}{Расслоение, вид сверху}

\threefigsH{рис:броня-10-25}{Разрушение в~стеклянной конструкции, 25-й шаг}
           {png/armor/10/image026-crack.png}{Разрушение}
           {png/armor/10/image026-failed-contact.png}{Расслоение}
           {png/armor/10/image026-failed-contact-delamination.png}{Расслоение, вид сверху}

\threefigsH{рис:броня-10-50}{Разрушение в~стеклянной конструкции, 50-й шаг}
           {png/armor/10/image051-crack.png}{Разрушение}
           {png/armor/10/image051-failed-contact.png}{Расслоение}
           {png/armor/10/image051-failed-contact-delamination.png}{Расслоение, вид сверху}

\threefigsH{рис:броня-10-75}{Разрушение в~стеклянной конструкции, 75-й шаг}
           {png/armor/10/image076-crack.png}{Разрушение}
           {png/armor/10/image076-failed-contact.png}{Расслоение}
           {png/armor/10/image076-failed-contact-delamination.png}{Расслоение, вид сверху}

\threefigsH{рис:броня-10-100}{Разрушение в~стеклянной конструкции, 100-й шаг}
           {png/armor/10/image101-crack.png}{Разрушение}
           {png/armor/10/image101-failed-contact.png}{Расслоение}
           {png/armor/10/image101-failed-contact-delamination.png}{Расслоение, вид сверху}

\threefigsH{рис:броня-10-125}{Разрушение в~стеклянной конструкции, 125-й шаг}
           {png/armor/10/image126-crack.png}{Разрушение}
           {png/armor/10/image126-failed-contact.png}{Расслоение}
           {png/armor/10/image126-failed-contact-delamination.png}{Расслоение, вид сверху}

\threefigsH{рис:броня-10-150}{Разрушение в~стеклянной конструкции, 150-й шаг}
           {png/armor/10/image151-crack.png}{Разрушение}
           {png/armor/10/image151-failed-contact.png}{Расслоение}
           {png/armor/10/image151-failed-contact-delamination.png}{Расслоение, вид сверху}

\threefigsH{рис:броня-10-175}{Разрушение в~стеклянной конструкции, 175-й шаг}
           {png/armor/10/image176-crack.png}{Разрушение}
           {png/armor/10/image176-failed-contact.png}{Расслоение}
           {png/armor/10/image176-failed-contact-delamination.png}{Расслоение, вид сверху}

\threefigsH{рис:броня-10-200}{Разрушение в~стеклянной конструкции, 200-й шаг}
           {png/armor/10/image201-crack.png}{Разрушение}
           {png/armor/10/image201-failed-contact.png}{Расслоение}
           {png/armor/10/image201-failed-contact-delamination.png}{Расслоение, вид сверху}

\subsubsection[Зависимость размера области расслоения\\от~адгезионной прочности]{Зависимость размера области расслоения от~адгезионной прочности}

Для того, чтобы определить, как сильно адгезионная прочность влияет на~итоговый размер областей разрушения и расслоения,
было выполнено несколько расчётов, в~которых все параметры материалов были оставлены прежними, а адгезионная прочность
увеличена. Результаты расчётов для различных значений адгезионной прочности представлены
на~рис.~\ref{рис:броня-сравнение-50}--\ref{рис:броня-сравнение-200}.

Как видно из~результатов расчёта, величина адгезионной прочности не только влияет на~итоговый размер области
деламинации, но и оказывает качественное влияние на~размеры разрушенной области. Это можно объяснить тем, что при
уменьшении области деламинации уменьшается количество отражённых от~свободной границы волн, что в~свою очередь ведёт к
снижению нагрузки на~верхние слои и вызывает заметно меньшее их повреждение.

\threefigsH{рис:броня-сравнение-50}{Области разрушения и расслоения, адгезионная прочность 50 МПа}
          {png/armor/50/image250-crack.png}{Разрушение}
          {png/armor/50/image250-failed-contact.png}{Расслоение}
          {png/armor/50/image250-failed-contact-delamination.png}{Расслоение, вид сверху}

\threefigsH{рис:броня-сравнение-100}{Области разрушения и расслоения, адгезионная прочность 100 МПа}
          {png/armor/100/image250-crack.png}{Разрушение}
          {png/armor/100/image250-failed-contact.png}{Расслоение}
          {png/armor/100/image250-failed-contact-delamination.png}{Расслоение, вид сверху}

\threefigsH{рис:броня-сравнение-150}{Области разрушения и расслоения, адгезионная прочность 150 МПа}
          {png/armor/150/image250-crack.png}{Разрушение}
          {png/armor/150/image250-failed-contact.png}{Расслоение}
          {png/armor/150/image250-failed-contact-delamination.png}{Расслоение, вид сверху}

\threefigsH{рис:броня-сравнение-200}{Области разрушения и расслоения, адгезионная прочность 200 МПа}
          {png/armor/200/image250-crack.png}{Разрушение}
          {png/armor/200/image250-failed-contact.png}{Расслоение}
          {png/armor/200/image250-failed-contact-delamination.png}{Расслоение, вид сверху}

\end{document}
