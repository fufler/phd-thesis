\documentclass[thesis.tex]{subfiles}

\begin{document}

\section{Численные методы}

Для решения различных классов задач механики деформируемого тела зачастую требуется использование различных численных
метод, так как все они обладают определёнными преимуществами и недостатками. В~этой главе рассматриваются четыре
численных метода решения: се\-точ\-но-характеристический, метод сглаженных частиц, комбинированный метод и метод маркеров.

Первым рассматривается сеточно-характеристический метод \cite{петров1984численное}, позволяющий моделировать сложные
волновые процессы, которые происходят в~телах со~сложной реологией. Метод позволяет корректно учитывать отклик от~границ
различного типа, что оказывается весьма полезным при моделировании слоистых конструкций, например, композиционных
материалов. Обычно этот метод реализуется с~использование неструктурированных тетраэдральных сеток, что негативно
сказывается на~времени расчёта, а также предполагает работу только в~режиме малых деформаций.

Следующий метод~--- метод сглаженных частиц~--- изначально разработан \cite{monaghan1988introduction} для расчёта задач
гидродинамики. Тем не менее, этот метод оказывается весьма удобным при решении задач механики деформируемого твёрдого
тела, в~которых присутствуют большие деформации области интегрирования, а также разлёт вещества.

Завершает раздел описание двух методов, являющихся логическим развитием метода сглаженных частиц и
сеточно-характеристического метода. Так, комбинированный метод \cite{петров2014схм} представляет из~себя
гибридный численный метод, который использует одновременно метод сглаженных частиц и сеточно-характеристический метод
для расчёта различных частей области интегрирования. Рассмотренный далее метод маркеров предполагает одновременное
использование неподвижной расчётной эйлеровой сетки и подвижной лагранжевой сетки для отслеживания положения границы.

\subfile{gcm}
\subfile{sph}
\subfile{combined}
\subfile{markers}

\end{document}
