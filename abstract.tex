\documentclass[a4paper,14pt]{extarticle}
\usepackage[top=2.0cm,bottom=2.0cm,left=2.0cm,right=2.0cm]{geometry}
\usepackage[T2A,T1]{fontenc}
\usepackage[english,russian]{babel}

\usepackage{fontspec}
\defaultfontfeatures{Ligatures={TeX}}
\setmainfont{CMU Serif}
\setsansfont{CMU Sans Serif}
\setmonofont{CMU Typewriter Text}

\usepackage{subfiles}
\usepackage{graphicx}
\usepackage{gnuplottex}
\usepackage{subcaption}


\usepackage{hyperref}

\usepackage{tabularx}
\usepackage{array}
\usepackage{multirow}

\usepackage{gensymb}
\usepackage{amsmath}
\usepackage[shortcuts]{extdash}


\usepackage{indentfirst}
\usepackage[nodisplayskipstretch]{setspace}

\usepackage[format=hang, labelsep=none, margin=10pt, figurename=Рисунок]{caption}
\DeclareCaptionLabelSeparator{gost}{~---~}
\captionsetup{labelsep=gost}
\numberwithin{equation}{section}
\setcounter{secnumdepth}{3}
\setcounter{tocdepth}{3}
%\onehalfspacing
\linespread{1.2}
\frenchspacing

\usepackage{fancyhdr}
\fancyhf{}
\renewcommand{\headrulewidth}{0pt}
\fancyhead[C]{\thepage}
\pagestyle{fancy}

\usepackage{cite}

\hyphenation{ЦАГИ}

\newcommand{\disser}{Численное решение трёхмерных задач\\
                     разрушения инженерных конструкций\\
                     при разных режимах нагружения}

\newcommand{\twofigscommon}[7]{
    \begin{figure}[#7]
        \centering
        \subcaptionbox{#4}{\includegraphics[width=0.45\linewidth]{#3}}
        \subcaptionbox{#6}{\includegraphics[width=0.45\linewidth]{#5}}
        \caption{#2}
        \label{#1}
    \end{figure}
}

\newcommand{\twofigs}[6]{ \twofigscommon{#1}{#2}{#3}{#4}{#5}{#6}{ht} }

\newcommand{\twofigsH}[6]{ \twofigscommon{#1}{#2}{#3}{#4}{#5}{#6}{h!} }
\newcommand{\twofigsT}[6]{ \twofigscommon{#1}{#2}{#3}{#4}{#5}{#6}{t!} }

\newcommand{\threefigscommon}[9]{
    \begin{figure}[#9]
        \centering
        \subcaptionbox{#4}{\includegraphics[width=0.30\linewidth]{#3}}
        \subcaptionbox{#6}{\includegraphics[width=0.30\linewidth]{#5}}
        \subcaptionbox{#8}{\includegraphics[width=0.30\linewidth]{#7}}
        \caption{#2}
        \label{#1}
    \end{figure}
}

\newcommand{\threefigs}[8]{ \threefigscommon{#1}{#2}{#3}{#4}{#5}{#6}{#7}{#8}{ht} }
\newcommand{\threefigsH}[8]{ \threefigscommon{#1}{#2}{#3}{#4}{#5}{#6}{#7}{#8}{h!} }

\newcommand{\figsub}[4]{
    \begin{figure}[ht]
        \centering
        \begin{subfigure}{0.45\linewidth}
            \centering
            \includegraphics[width=\textwidth]{#3}
            \caption{#4}
        \end{subfigure}
        \caption{#2}
        \label{#1}
    \end{figure}
}

\newcommand{\twofigsminipage}[6]{
    \begin{figure}[ht]
        \begin{minipage}[b]{0.45\linewidth}
            \centering
            \includegraphics[width=\textwidth]{#1}
            \caption{#3}
            \label{#2}
        \end{minipage}
        \begin{minipage}[b]{0.45\linewidth}
            \centering
            \includegraphics[width=\textwidth]{#4}
            \caption{#6}
            \label{#5}
        \end{minipage}
    \end{figure}
}

\newcommand{\figminipage}[3]{
    \begin{figure}[ht]
        \begin{minipage}[b]{0.45\linewidth}
            \centering
            \includegraphics[width=\textwidth]{#1}
            \caption{#3}
            \label{#2}
        \end{minipage}
    \end{figure}
}

\newcommand{\figcommon}[5]{
    \begin{figure}[#5]
        \centering
        \includegraphics[width=#4]{#2}
        \caption{#3}
        \label{#1}
    \end{figure}
}

\newcommand{\fig}[3]{
    \figcommon{#1}{#2}{#3}{0.45\linewidth}{ht}
}

\newcommand{\figsmall}[3]{
    \figcommon{#1}{#2}{#3}{0.35\linewidth}{ht}
}

\newcommand{\figsmallH}[3]{
    \figcommon{#1}{#2}{#3}{0.35\linewidth}{h!}
}

\newcommand{\figH}[3]{
    \figcommon{#1}{#2}{#3}{0.45\linewidth}{h!}
}

\newcommand{\figfull}[3]{
    \figcommon{#1}{#2}{#3}{0.80\linewidth}{ht}
}

\newcommand{\figfullH}[3]{
    \figcommon{#1}{#2}{#3}{0.80\linewidth}{h!}
}

\newcommand{\ctodo}{\todo[color=red]}
\newcommand{\ntodo}{\todo[color=blue!40]}
\newcommand{\rtodo}{\todo}

\newcolumntype{L}[1]{>{\raggedright\let\newline\\\arraybackslash\hspace{0pt}}m{#1}}
\newcolumntype{C}[1]{>{\centering\let\newline\\\arraybackslash\hspace{0pt}}m{#1}}
\newcolumntype{R}[1]{>{\raggedleft\let\newline\\\arraybackslash\hspace{0pt}}m{#1}}

\newcommand\abs[1]{\left|#1\right|}

\newcommand{\PD}[2]{\frac{\partial{#1}}{\partial{#2}}}
\newcommand{\TD}[2]{\frac{d#1}{d#2}}

\newcommand{\PDx}[1]{\PD{#1}{x}}
\newcommand{\PDy}[1]{\PD{#1}{y}}
\newcommand{\PDz}[1]{\PD{#1}{z}}
\newcommand{\PDt}[1]{\PD{#1}{t}}

\newcommand{\TDt}[1]{\TD{#1}{t}}

\newcommand{\Sxx}{\sigma_{xx}}
\newcommand{\Sxy}{\sigma_{xy}}
\newcommand{\Sxz}{\sigma_{xz}}
\newcommand{\Syy}{\sigma_{yy}}
\newcommand{\Syz}{\sigma_{yz}}
\newcommand{\Szz}{\sigma_{zz}}
\bibliographystyle{ugost2008ll}


\begin{document}
    \pagenumbering{gobble}

    \begin{titlepage}
        \begin{flushright}
            На правах рукописи
        \end{flushright}

        \vspace{6em}

        \begin{center}
            Ермаков Алексей Сергеевич
        \end{center}

        \vspace{4em}

        \begin{center}
            \textsc{\textbf{\disser}}
        \end{center}

        \vspace{4em}

        \begin{center}
            Специальность 05.13.18\\
            Математическое моделирование,\\
            численные методы и комплексы программ
        \end{center}

        \vspace{1.5em}

        \begin{center}
        \textsc{
            \textbf{Автореферат}}\\
            диссертации на~соискание учёной степени\\
            кандидата физико-математических наук
        \end{center}

        \vspace{\fill}

        \begin{center}
            Москва --- 2015
        \end{center}
    \end{titlepage}

    \newpage

    \noindent
    Работа выполнена на~кафедре информатики\\
    Московского физико-технического института\\
    (государственного университета)\\

    \vspace{1em}

    \begin{hyphenrules}{nohyphenation}
    \noindent
    \begin{minipage}[t]{0.26\textwidth}
        Научный\\руководитель:
    \end{minipage}
    \begin{minipage}[t]{0.72\textwidth}
        \textbf{Петров Игорь Борисович},
        член-корреспондент РАН, доктор физико-математических наук, профессор
    \end{minipage}

    \vspace{1em}

    \noindent
    \begin{minipage}[t]{0.26\textwidth}
        Официальные\\оппоненты:
    \end{minipage}
    \begin{minipage}[t]{0.72\textwidth}
        \textbf{Перепёлкин Евгений Евгеньевич},
        доктор физико-математических наук, Московский государственный университет
        им. М.В. Ломоносова, физический факультет, отделение экспериментальной и
        теоретической физики, кафедра квантовой статистики и теории поля,
        ведущий научный сотрудник
        \\
        \\
        \textbf{Шувалов Павел Вадимович},
        кандидат физико-математических наук, Общество с~ограниченной
        ответственностью <<ДАБЛ>>, технический директор.
    \end{minipage}

    \vspace{1em}

    \noindent
    \begin{minipage}[t]{0.26\textwidth}
    Ведущая\\организация:
    \end{minipage}
    \begin{minipage}[t]{0.72\textwidth}
    \leavevmode \newline
    Институт автоматизации проектирования РАН
    \end{minipage}

    \end{hyphenrules}

    \vspace{2em}

    \noindent
    Защита состоится <<19>> ноября 2015 г. в~ 9:15 часов на~заседании
    диссертационного совета Д 212.156.05 на базе Московского физико-технического
    института (государственного университета) по~адресу:
    141700, Московская обл., г. Долгопрудный, Институтский пер., д. 9, ауд.
    903 КПМ.

    \vspace{1em}

    \noindent
    С~диссертацией можно ознакомиться в~библиотеке МФТИ и на сайте института http://mipt.ru.

    \vspace{1em}

    \noindent
    Автореферат разослан <<\underline{\hspace{0.75cm}}>> \underline{\hspace{2cm}} 2015 г.

    \vspace{\fill}

    \noindent
    Учёный секретарь\\диссертационного совета \hspace{5cm}Федько Ольга Сергеевна

    \newpage
    \pagenumbering{arabic}
    \setcounter{page}{3}

    \section*{Общая характеристика работы}
    \subsection*{Актуальность темы работы}

    Высокие темпы развития современной науки и техники приводят к~появлению новых классов задач, от~решения которых
    зависит возможность использования передовых разработок. Так, например, всё более широкое применение находят
    композиционные материалы, которые обладают очень высоким отношением прочности к~весу, заметно опережая по~этому
    показателю <<классические>> материалы, такие как металлы. Тем не менее, массовое использование этого класса
    материалов заметно осложнено из-за отсутствия экспериментальных методов определения характеристик
    композитов. Композиты, будучи по~своей природе принципиально анизотропными, совершенно иначе ведут себя при
    динамическом нагружении, нежели металлы. Примером может служить расслоение, возникающее в~ряде случаев между
    матрицей и армирующими элементами, которое внешне никак не заметно, при этом практически всегда ведёт к
    значительному снижению несущей способности и, как следствие, невозможности использования конструкции в~дальнейшем.

    В~связи с~этим в~последнее время значительное развитие получили методы, позволяющие проводить численные
    эксперименты там, где выполнение натурного эксперимента затруднено или невозможно. Наиболее часто для решения этого
    класса задач используется метод конечных элементов, который всё же имеет некоторые ограничения, сужающие
    область возможного применения. Этот метод хорошо зарекомендовал себя при решении задач статического нагружения, а
    также динамических задач, в~которых не требуется высокое временное разрешение. В~случае же моделирования поведения
    композиционных материалов сложной структуры под действием динамических нагрузок высокие требования выдвигаются как к
    пространственному разрешению метода, так и к~временному. Это в~первую очередь связано с~тем, что разрушения,
    характерные для полимерных материалов, происходят на~фоне формирования сложной волновой картины, возникающей в
    результате многократных отражений волн как от~внешних границ тела, так и от~внутренних границ раздела сред.

    В~связи с~этим становится очевидной необходимость разработки методов численного моделирования, позволяющих решать
    задачи динамического нагружения конструкций с~учётом особенностей анизотропных материалов. Более того, использование
    этих методов позволит решать не только задачи по~моделированию последствий физического воздействия на~образец, но
    также даст возможность приблизиться к~решению задач дефектоскопии и неразрушающего контроля. Решение задач неразрушающего
    контроля необходимо для использования композиционных материалов при проектировании сложных
    конструкций в~авиастроении, так как актуальные методы диагностики дефектов были разработаны для изотропных
    материалов. Использование современных методов диагностики позволит в~дальнейшем заметно снизить риск неожиданного
    выхода из~строя важных инженерных конструкций, что весьма актуально, например, для авиастроения.

    В~диссертации рассматривается применение сеточно-характеристичес\-ко\-го метода для решения системы уравнений
    механики деформируемого твёрдого тела. Этот метод обладает хорошим разрешением как по времени, так и по
    пространству, а также позволяет корректно учитывать процессы взаимодействия волн как в~объёме тела, так и
    на~границе. Особенностью этого метода является достаточно высокая сложность алгоритмов, позволяющих производить
    численное решение задач, подразумевающих большие деформации исходной геометрии. Для преодоления этого
    ограничения в~работе предложен метод маркеров, позволяющий рассчитывать при помощи сеточно-характеристического
    метода задачи высокоскоростного соударения, не приводящие к~сквозному пробою. Также в~работе рассмотрено применение метода
    сглаженных частиц к~задачам механики деформируемого твёрдого тела и комбинированного метода, сочетающего
    в~себе метод сглаженных частиц и сеточно-характеристический метод.


    \subsection*{Цели работы}

    \begin{enumerate}
        \item Разработка и программная реализация вычислительного алгоритма для численного метода, позволяющего
              моделировать сложные волновые процессы, происходящие в анизотропных композитных конструкциях.
        \item Анализ и сопоставление существующих критериев объёмного разрушения материала с~целью формулирования рекомендаций по их
              применению при моделировании композиционных материалов.
        \item Реализация и верификация программного комплекса, использующего сеточно-характеристический метод для
              численного моделирования процессов в~анизотропных средах.
        \item Реализация и верификация программного комплекса, использующего комбинацию сеточно-характеристического
              метода и метода сглаженных частиц для решения задач с~конечными деформациями.
        \item Адаптация трёхмерного метода маркеров для решения задачи динамического деформирования твёрдого
              тела с~конечными деформациями при помощи сеточно-характеристического метода.
        \item Разработка и верификация программного комплекса, использующего предложенную комбинацию метода маркеров
              и сеточно-ха\-рак\-те\-риc\-ти\-чес\-ко\-го метода, для моделирования волновых процессов в~деформируемом
              твёрдом теле.
        \item Применение описанных выше методов для решения ряда практически значимых задач.
    \end{enumerate}

    \subsection*{Научная новизна}

    \begin{enumerate}
        \item Адаптирован для использования совместно с~сеточно-характерис\-ти\-чес\-ким методом и проверен на~модельных
              задачах трёхмерный метод маркеров, позволяющий проводить численное моделирование процессов в~задачах механики
              деформируемого твёрдого тела в~условиях конечных деформаций.
        \item Проведен анализ наиболее используемых на~данный момент критериев разрушения композиционного
              материала, по~результатам которого сделаны выводы о~целесообразности применения этих критериев для решения практически
              значимых задач.
        \item Программно реализован численный метод, позволяющий моделировать сложные волновые процессы, происходящие в конструкциях,
              изготовленных из~анизотропных композиционных материалов.
        \item Реализован и проверен на~модельных задачах программный комплекс, использующий численный метод,
              являющийся комбинацией сеточно-характеристического метода и метода сглаженных частиц.
        \item С~использованием описанных методов решён ряд практически значимых задач, среди которых:
            \begin{itemize}
                \item задача о~низкоскоростном ударе по~трёхстрингерной композиционной панели;
                \item задача о~низкоскоростном ударе по~композиционным панелям, выполненным из~различных материалов
                      (GFRP и CFRP);
                \item задача об~объёмном разрушении стекла под действием лазерного излучения;
                \item задача о~высокоскоростном ударе по~прозрачной слоистой конструкции;
                \item задача о~падении самолёта на~оболочку ядерного реактора;
                \item задача о~пробивании оболочки спутника микрометеоритом;
                \item задача о~столкновении стального осколка со~стальной преградой при различных соотношениях
                      характерных размеров  тел, участвующих в~столкновении.
            \end{itemize}
    \end{enumerate}

    \subsection*{Теоретическая и практическая значимость работы}
    Реализованный метод моделирования волновых процессов, протекающих в~анизотропных композиционных материалах, может быть
    в~дальнейшем использован  для численного моделирования поведения композиционных авиационных конструкций при действии
    различных динамических  нагрузок. Также этот метод может быть использован для решения задач неразрушающего контроля и
    дефектоскопии.

    Результаты моделирования процесса разрушения стеклянного образца под действием лазерного излучения могут быть
    использованы для исследования поведения подобных материалов при энергетических воздействиях и верификации моделей
    объёмного разрушения хрупких материалов.

    Моделирование волновых процессов в~многослойных прозрачных конструкциях проводится для улучшения
    их защищённости, что может быть использовано, например, в~военной и гражданской авиации.

    Численное моделирование процессов высокоскоростного взаимодейст\-вия используется для исследования несущей
    способности различных конструкций, подверженных ударному нагружению.

    Предложенная комбинация метод маркеров и сеточно-ха\-рак\-те\-рис\-ти\-чес\-ко\-го метода обеспечивает
    возможность выполнения расчётов, требующих одновременного изменения геометрии расчётной области и моделирования
    сложных волновых процессов, порождённых интенсивными динамическими нагрузками.

    Работа поддержана рядом государственных и коммерческих грантов:
    \begin{enumerate}
        \item Грант РФФИ 11-01-00723 А. Разработка численных методов моделирования динамических задач биомеханики на
              современных высокопроизводительных вычислительных системах, 2011-2013 гг.
        \item Грант РФФИ 13-07-00072 А. Разработка параллельных алгоритмов для решения систем уравнений гиперболического
              типа на~многопроцессорных вычислительных системах, 2013-2015 гг.
    \end{enumerate}

    \subsection*{Методология и методы исследования}

    Исследуются задачи механики деофрмируемого твёрдого тела методами численного моделирования. В работе используется
    сеточно-ха\-рак\-те\-рис\-ти\-чес\-кий численный метод на неструктурированных тетраэдральных сетках, комбинация метода маркеров
    и сеточно-характеристического метода  на регулярных гексаэдральных эйлеровых сетках, а также комбинация метода
    сглаженных частиц и сеточно-характеристического метода. Для верификации методов используется сравнение с
    аналитическими решениями ряда задач, а также сравнение с экспериментальными данными. При исследовании прикладных
    задач методами моделирования рассчитывается полная пространственно-временная волновая картина, позволяющая получить
    точную информация о полях скоростей и напряжений внутри тела, а также выявить области деформации и разрушения.

    \subsection*{Положения, выносимые на защиту}
    Положения, выносимые на защиту, соответствуют основным результатам, приведённым в заключении автореферата.

    \subsection*{Степень достоверности и апробация результатов}

    Результаты диссертации опубликованы в~одиннадцати работах, из~которых семь
    \cite{беклемышева2013численное,васюков2014комбинирование,петров2014схм,petrov2014combined,петров2014численный,петров2015комбинированный,беклемышева2014численное}~---
    в~изданиях, рекомендованных ВАК для публикации основных результатов диссертации.

    Результаты работы были доложены, обсуждены и получили одобрение специалистов на~следующих научных конференциях:

    \begin{enumerate}
        \item 55-я научная конференция МФТИ <<Проблемы фундаментальных и прикладных естественных и технических наук в
              современном информационном обществе>> (МФТИ, Долгопрудный, 2012).
        \item 56-я научная конференция МФТИ <<Проблемы фундаментальных и прикладных естественных и технических наук в
              современном информационном обществе>> (МФТИ, Долгопрудный, 2013).
        \item Семинары Центра компьютерного моделирования (ЦКМ) в рамках программы совместных фундаментальных исследований
              ЦАГИ и РАН (ЦАГИ, Жуковский, 2014--2015).
    \end{enumerate}

    Результаты работы были доложены, обсуждены и получили одобрение специалистов на~научных семинарах в~следующих
    организациях:

    \begin{enumerate}
        \item Национальный исследовательский центр <<Курчатовский институт>> (Москва, 2014).
        \item Центральный аэрогидродинамический институт имени профессора
              Н. Е. Жуковского (Жуковский, 2014--2015).
        \item Институт прикладной математики им. М.В. Келдыша Российской академии наук (Москва, 2015).
        \item Институт автоматизации проектирования Российской академии наук (Москва, 2015).
    \end{enumerate}


    \section*{Основное содержание работы}
    \subsection*{Структура и объём диссертации}

    Диссертация состоит из~введения, пяти глав, заключения и списка литературы. Объем диссертации
    составляет 175~страниц, включая 156~рисунков, 8~таблиц и список литературы со ссылками на~91
    публикацию.

        \subsubsection*{Введение}

        Во~введении показывается необходимость численного моделирования при решении актуальных задач динамического
        нагружения материалов и обосновывается выбор методов, использованных в~ходе данной работы. Также в~этом разделе
        описываются основные цели и задачи, которые были поставлены перед автором, проводится обоснование научной
        новизны и практической значимости работы.

        \subsubsection*{Глава 1}

        В~первой главе рассматривается математическая модель, используемая для решения задач механики деформируемого
        твёрдого тела, а именно система уравнений:
        \begin{eqnarray*}
            \rho\dot{v}_i=\nabla_j\sigma_{ij}+f_i             & \textrm{(уравнения движения),} \\
            \dot\sigma_{ij}=q_{ijkl}\dot{\varepsilon}_{kl}+F_{ij} & \textrm{(реологические соотношения).}
        \end{eqnarray*}
        Здесь $\rho$~--- плотность среды, $v_i$~--- компоненты скорости смещения,
        $\sigma_{ij}$, $\varepsilon_{ij}$~--- компоненты тензоров напряжений и деформаций,
        $\nabla_j$ – ковариантная производная по~$j$-й координате, $f_i$ – массовые
        силы, действующие на~единицу объёма, $F_{ij}$~--- добавочная правая часть.


        Вид компонент тензора 4-го порядка $q_{ijkl}$ определяется реологией среды. Для общего случая анизотропного
        материала тензор $q_{ijkl}$ имеет существенно более сложный вид, поэтому здесь не приводится. Матричная форма
        исходных уравнений, получаемая после подстановки тензора $q_{ijkl}$, представлена в~тексте диссертации как для
        общего случая, так и для частных случаев орторомбической анизотропии и трансверсальной изотропии материала.

        Дальнейшая часть первой главы посвящена обзору наиболее распространённых критериев объёмного разрушения
        материала и математических моделей линейно-упругого тела. В~качестве критериев разрушения рассмотрены: критерий
        максимального нормального напряжения, критерии Мизеса, Мора-Кулона, Друкера-Прагера, Хашина,  Пака,  Цая-Хилла,
        Цая-Ву. Все перечисленные критерии изучались с~точки зрения применимости их к~расчёту композиционных материалов,
        ввиду наличия особых механизмов формирования трещин и областей разрушения, специфичных сугубо для полимеров.
        Отдельно рассмотрено понятие адгезионной прочности, которое неразрывно связано с~вопросами численного
        моделирования деламинации в~сложных композиционных материалах.


        \subsubsection*{Глава 2}

        Во второй главе диссертации описываются численные методы, использованные при решении задач. Первым представлен
        сеточно-ха\-рак\-те\-рис\-ти\-чес\-кий метод, суть которого описана ниже.

        Для расчёта при помощи сеточно-характеристического метода используется неструктурированная тетраэдральная сетка,
        в~узлах которой ищется решение. Поиск решения осуществляется в~несколько этапов: расщепление по~направлениям
        исходной системы уравнений, записанной в~матричной форме, последующий переход к~инвариантам Римана и
        интерполяция значений на~новом временном слое с~использованием ранее полученных значений. Отдельно в~тексте
        диссертации рассмотрены вопросы постановки контактных условий различного типа при
        использовании сеточно-характеристического метода.

        Далее приводится описание метода сглаженных частиц, основная идея которого заключается в~использовании
        дискретных расчётных элементов, которые имеют <<область влияния>>. Таким образом, этот несеточный численный метод
        основывается на~аппроксимации значений искомой функции при помощи суммирования значений в~определённых точках
        пространства (частицах) с использованием весовой функции, называемой ядром сглаживания.

        Следующий описанный в~главе метод является комбинацией сеточно-характеристического метода и метода частиц. Этот метод
        предназначен для решения <<гибридных>> задач, в~которых итоговое разрушение формируется как непосредственно от
        столкновения с~ударником, так и под действием различных волн, формирующихся в~материале после удара. Суть метода
        заключается в~разбиении исходной области интегрирования на~две: одна считается при помощи
        сеточно-характеристического метода, другая~--- методом сглаженных частиц. Также в~главе приводятся результаты
        верификации реализованного метода на~наборе тестовых задач, в~том числе на~качественном уровне получено
        совпадение результатов решения задачи о~пробивании преграды при помощи коммерческого пакета LS-DYNA (см. рис.
        \ref{fig:dyna_vs_spgcm}).

        \begin{figure}[ht]
            \centering
            \begin{subfigure}{0.32\linewidth}
                \centering
                \tiny
                \begin{gnuplot}
                    set datafile separator ","
                    set terminal epslatex monochrome size 5.5cm, 4.5cm 8
                    set grid ytics xtics
                    set grid
                    set xlabel "t, мкс"
                    set ylabel "$v_x$, м/с"
                    set xrange [0:40]
                    set yrange [0:1200]
                    plot "csv/angled-sphere-spgcm.csv" title "SPGCM", \
                        "csv/angled-sphere-dyna.csv" title "LS-DYNA" with points pointtype 7
                \end{gnuplot}
                \caption{Шар, 30\degree}
            \end{subfigure}
            \begin{subfigure}{0.32\linewidth}
                \centering
                \tiny
                \begin{gnuplot}
                    set datafile separator ","
                    set terminal epslatex monochrome size 5.5cm, 4.5cm 8
                    set grid ytics xtics
                    set grid
                    set xlabel "t, мкс"
                    set ylabel "$v_x$, м/с"
                    set xrange [0:40]
                    set yrange [0:1200]
                    plot "csv/normal-sphere-spgcm.csv" title "SPGCM", \
                        "csv/normal-sphere-dyna.csv" title "LS-DYNA" with points pointtype 7
                \end{gnuplot}
                \caption{Шар, 0\degree}
            \end{subfigure}
            \begin{subfigure}{0.32\linewidth}
                \centering
                \tiny
                \begin{gnuplot}
                    set datafile separator ","
                    set terminal epslatex monochrome size 5.5cm, 4.5cm 8
                    set grid ytics xtics
                    set grid
                    set xlabel "t, мкс"
                    set ylabel "$v_x$, м/с"
                    set xrange [0:40]
                    set yrange [0:1200]
                    plot "csv/normal-cube-spgcm.csv" title "SPGCM", \
                        "csv/normal-cube-dyna.csv" title "LS-DYNA" with points pointtype 7
                \end{gnuplot}
                \caption{Куб, 0\degree}
            \end{subfigure}
            \caption{Сравнение результатов SPGCM и LS-DYNA при  моделировании пробивания преграды осколком разной формы при
                    разных углах столкновения. На всех рисунках изображена зависимость среднемассовой скорости (м/с) от
                    времени (мкс)}
            \label{fig:dyna_vs_spgcm}
        \end{figure}

        Завершает главу описание метода маркеров, разработанного специально, чтобы дать возможность проводить численные
        расчёты сеточно-характеристическим методом в~случае заметных деформаций. Этот метод представляет собой объединение
        двух подходов к~использованию расчётных сеток: эйлерова и лагранжева. Его главной особенностью является
        одновременное использование неподвижной структурированной сетки, состоящей из~прямоугольных параллелепипедов,
        для расчёта внутренней области тела и неструктурированной поверхностной сетки, состоящей из~треугольников, для
        отслеживания положения внешней границы тела. Также в~разделе рассмотрены несколько подходов к~адаптации
        сеточно-ха\-рак\-те\-рис\-ти\-чес\-ко\-го метода для решения задач с~существенными изменениями области
        интегрирования.

        % \newpage
        \subsubsection*{Глава 3}

        Третья глава посвящена вопросам моделирования хрупкого разрушения.

        Первой рассматривается задача о внутреннем разрушении стеклянной преграды под действием лазерного излучения (см.~рис.~\ref{fig:glass}).
        Эта задача является весьма актуальной в~современных реалиях в~связи с~достаточно частым использованием подобных
        материалов, например, в~военной промышленности, при проектировании важных инженерных конструкций,
        которые должны выдерживать высокие нагрузки. Численное моделирование этой задачи наглядно демонстрирует,
        каким именно образом происходит формирование внутреннего разрушения материала под действием волн, возникающих в
        образце под действием ударного нагружения.

       \twofigsH
            {fig:glass}{Разрушение в~стекле под действием лазерного излучения}
            {png/glass/glass-laser-crack-3d-gray.png}{Численное моделирование}
            {png/glass/experiment-gray.png}{Натурный эксперимент}

        Стоит отметить, что численное решение этой задачи на~качественном уровне совпадает с~экспериментом: получены те
        же характерные области разрушения, достаточно хорошо совпадающие как по~размерам, так и по~своему положению.
        Более детальное сравнение приведено в~полном тексте диссертации.

        Другой задачей, представляющей интерес с~практической точки зрения, является задача о~столкновении стального
        шарика с~многослойным бронированным стеклом (см.~рис.~\ref{fig:armor}). По сути, эта задача возникает как
        в~гражданском авиастроении (прохождение самолёта через град), так и в~военном (взрыв шариковой бомбы). Процессы,
        приводящие к~разрушению многослойного бронированного стекла, весьма близки к~тем, что происходят
        в~композиционных материалах, поэтому при решении задачи внимание было уделено в~том числе процессу расслоения,
        который в~жизни может привести к~отрыву части поверхности, что в суровых условиях эксплуатации авиационной
        техники с~практической точки зрения равносильно пробою брони.

        \begin{figure}[h]
            \begin{center}
                \includegraphics[width=0.6\textwidth]{png/armor/100/image401-failed-contact-gray.png}
            \end{center}
            \caption{Расслоение многослойной стеклянной брони}
            \label{fig:armor}
        \end{figure}

        Отдельно рассмотрен вопрос о~зависимости области расслоения от~адгезионной прочности контакта между клеем и
        стеклом, в~ходе численного эксперимента получено подтверждение того, что увеличение адгезионной прочности в~
        несколько раз ведёт к~существенному уменьшению области расслоения
        (см.~рис. \ref{fig:armor_delamination}).

        \twofigs
            {fig:armor_delamination}{Зависимость области расслоения от~адгезионной прочности}
            {png/armor/10/image250-failed-contact-delamination-gray.png}{Реальные параметры}
            {png/armor/150/image250-failed-contact-delamination-gray.png}{Прочность увеличена в~три раза}

        Для проверки применимости рассмотренных критериев разрушения была решена модельная задача о~столкновении
        стального шарика и преграды, изготовленной из~стеклопластика. Как было сказано ранее, материал, из~которого
        изготовлена преграда, является существенно анизотропным, поэтому при моделировании столкновения используются
        различные критерии объёмного разрушения, изначально разработанные для корректного учёта особенностей в~поведении
        композиционных материалов при нагрузках. Рассматривалось две постановки задачи: в~одной материал, из~которого
        изготовлена преграда, был армирован вдоль направления удара, в~другом~--- перпендикулярно направлению удара.

        Было проведено изучение результатов (см.~рис.~\ref{fig:destruction_1}--\ref{fig:destruction_2}) численного
        моделирования низкоскоростного столкновения с~использованием различных критериев разрушения, после чего были
        сформулированы соответствующие выводы об~области применения критериев (см.~табл.~\ref{table:failure_criteria}).

        \twofigsH
            {fig:destruction_1}{Моделирование разрушения при помощи различных критериев}
            {png/destruction/x-drpr-x--gray.png}{Критерий Друкера-Прагера}
            {png/destruction/x-tsaihill-x+-gray.png}{Критерий Цая-Хилла}

        \twofigsH
            {fig:destruction_2}{Моделирование разрушения при помощи различных критериев}
            {png/destruction/x-hashin-x--gray.png}{Критерий Хашина}
            {png/destruction/x-puck-x--gray.png}{Критерий Пака}

        \begin{table}[t!]
            \centering
            \begin{tabular}{| L{9cm} | C{1.0cm} | C{1.0cm} | C{1.0cm} | C{1.0cm} | C{1.0cm} |}
                \hline
                & \textbf{ДП} & \textbf{Х} & \textbf{П} & \textbf{ЦХ} & \textbf{ЦВ} \\
                \hline
                \textbf{Учёт наличия матрицы и волокон} & Нет & Да & Да & Нет & Нет \\
                \hline
                \textbf{Различие пределов на~сжатие и разряжение} & Да & Да & Да & Нет & Да \\
                \hline
                \textbf{Различие механизмов разрушения} & Нет & Да & Да & Нет & Нет \\
                \hline
                \textbf{Скалярная мера разрушения} & Да & Да & Нет & Да & Да \\
                \hline
                \textbf{Векторная мера разрушения} & Нет & Нет & Да & Нет & Нет \\
                \hline
                \textbf{Отсутствие внутренних параметров модели} & Да & Да & Нет & Да & Да \\
                \hline
                \textbf{Применимость к~однородному материалу} & Да & Нет & Нет & Да & Да \\
                \hline
                \textbf{Применимость к~армированному монослою} & Нет & Да & Да & Нет & Нет \\
                \hline
            \end{tabular}
            \caption{Сравнение физического смысла, допущений и области применения критериев разрушения
                     \textbf{Д}рукера-\textbf{П}рагера, \textbf{Х}ашина, \textbf{П}ака,
                     \textbf{Ц}ая-\textbf{Х}илла и \textbf{Ц}ая-\textbf{В}у}
            \label{table:failure_criteria}
        \end{table}

        По результатам расчётов был сделан вывод: несмотря на~то, что некоторые критерии изначально разработаны для
        моделирования анизотропных материалов, ни один из~них не позволяет проводить расчёты армированного
        композиционного материала в~приближении субпакета. Критерии  Друкера-Прагера, Цая-Хилла и Цая-Ву разработаны
        для применения с~однородным, хоть и анизотропным материалом, а поэтому принципиально не могут учитывать различные
        механизмы внутреннего разрушения, вызванные сложной структурой композиционного материала. В~свою очередь,
        критерии Хашина и Пака учитывают особенности строения композита, но могут быть применены только для
        моделирования монослоя с~армированием вдоль одного направления, что делает невозможным их использование для
        моделирования поведения субпакета, состоящего из~нескольких ориентированных монослоёв.

        После рассмотрения различных критериев разрушения в~работе приведены результаты
        (см.~рис.~\ref{fig:stringer_1}--\ref{fig:stringer_2}) численного решения задачи о~низкоскоростном столкномении
        металлического ударника и трёхстрингерной композиционной панели, применяемой при создании обшивки и силового
        кессона крыла самолёта. Используются ранее описанные критерии разрушения применительно к~реальной геометрии, а
        также проводится сравнение с~натурным экспериментом.

        \begin{figure}[h!]
            \begin{center}
                \includegraphics[width=0.6\textwidth]{png/3-stringer-panel/panel-3d-cover-gray.png}
            \end{center}
            \caption{Вид расчётной области композиционной панели при ударе в~обшивку}
            \label{fig:stringer_1}
        \end{figure}

        \twofigsT
            {fig:stringer_2}{Форма разрушенной области при ударе в~обшивку. Вид со~стороны ударника}
            {png/3-stringer-panel/cover-hashin-gray.png}{Критерий Хашина}
            {png/3-stringer-panel/cover-puck-gray.png}{Критерий Пака}

        По результатам расчётов сделан вывод о~том, что только критерий Пака даёт удовлетворительные результаты при
        моделировании армированного композиционного материала. Сравнение результатов, получаемых при использовании
        различных критериев разрушения, с~натурным экспериментом приведено в~табл.
        \ref{table:failure_criteria_vs_experiment}.

        Далее рассматривается задача о~деламинации композиционного материла, при этом в~численном эксперименте также
        учитывается объёмное разрушение материала, которое может влиять на~форму и размеры расслоения
        (см.~рис.~\ref{fig:delamination}).

        \twofigsT
            {fig:delamination}{Влияние используемого критерия объёмного разрушения на~итоговую форму и размер области
                               деламинации}
            {png/destruction/2lc-hashin-delam-gray.png}{Критерий Хашина}
            {png/destruction/2lc-puck-delam-gray.png}{Критерий Пака}

         \begin{table}[t!]
            \centering
            \begin{tabular}{| L{5cm} | C{1.3cm} | C{1.3cm} | C{1.7cm} | C{1.3cm} | C{1.3cm} | C{1.7cm} |}
                \hline
                \multirow{2}{5cm}{\textbf{Постановка задачи}} & \multicolumn{6}{ c| } {\textbf{Размер разрушенной области, мм}} \\
                \cline{2-7}
                & \textbf{ДП} & \textbf{Х} & \textbf{П} & \textbf{ЦХ} & \textbf{ЦВ} & \textbf{Э} \\
                \hline
                \textbf{Удар в~обшивку, 90 Дж} & 40x40 & 60x45 & 100x52 & 40x40 & 52x44 & 120x65 \\
                \hline
                \textbf{Удар в~стрингер, 135 Дж} & 40x40 & 52x35 & 65x32 & 40x40 & 45x30 & 80x30 \\
                \hline
            \end{tabular}
            \caption{Сравнение результатов численного моделирования с~использованием критериев
                     \textbf{Д}рукера-\textbf{П}рагера, \textbf{Х}ашина, \textbf{П}ака,
                     \textbf{Ц}ая-\textbf{Х}илла и \textbf{Ц}ая-\textbf{В}у с~результатами
                     натурного эксперимента (Э)}
            \label{table:failure_criteria_vs_experiment}
        \end{table}

        По результатам моделирования сделан вывод, что применение различных критериев объёмного разрушения оказывает
        качественное влияние на~размеры и форму области расслоения.

        \subsection*{Глава 4}

        Четвёртая глава диссертации посвящена рассмотрению прикладных задач, требующих для своего численного расчёта
        методов, позволяющих корректно учитывать большие деформации исходной геометрии.

        Первой рассматривается задача о~столкновении стального осколка и стальной преграды: рассматриваются постановки с
        различной скоростью подлёта, а также разными соотношениями характерных размеров сталкивающихся тел
        (см.~рис.~\ref{fig:obstacle}). Эта задача имеет достаточно широкое практическое применение: от~моделирования
        поведения различных инженерных конструкций с целью определения их несущей способности до~проведения экспертных
        оценок о~характере воздействия на~конструкцию, исходя из~итоговой картины повреждения.

        \twofigs
            {fig:obstacle}{Столкновение осколка с~преградой}
            {png/obstacle/thin/straight/2D-XZ-VELOCITY-ABS-10-gray.png}{Тонкая преграда}
            {png/obstacle/eq/straight/2D-XZ-VELOCITY-ABS-10-gray.png}{Преграда средней толщины}


        Далее рассматривается задача о~падении массивного быстро движущегося объекта на~сферическую оболочку
        (см.~рис.~\ref{fig:air_nuke-1}--\ref{fig:air_nuke-2}). Описаны две постановки: падении фюзеляжа самолёта
        на~сферическую оболочку и столкновение фюзеляжа и двигателей с~плоской преградой.

        \twofigs
            {fig:air_nuke-1}{Столкновение фюзеляжа с~оболочкой}
            {png/air-nuke/2D-VELOCITY-ABS-0-gray.png}{Начало столкновения}
            {png/air-nuke/2D-VELOCITY-ABS-69-gray.png}{Полное пробивание}

        \twofigs
            {fig:air_nuke-2}{Столкновение фюзеляжа и двигателей с~преградой}
            {png/air-nuke-engines/2D-VELOCITY-ABS-0-gray.png}{Начало столкновения}
            {png/air-nuke-engines/2D-VELOCITY-ABS-399-gray.png}{Полное пробивание}

        Как видно из~результатов расчёта, падение самолёта на~такой скорости вызывает полное разрушение поверхности
        оболочки, что может привести к~повреждению, например, важных объектов промышленности, находящихся внутри
        оболочки.

        В конце раздела рассматривается моделирование процесса столкновения космического спутника с~микрометеоритом~---
        другая важная задача, требующая для своего расчёта методов, учитывающих большие деформации.

        \twofigs
            {fig:satellite-2}{Столкновение микрометеорита со~спутником}
            {png/satellite/2D-VELOCITY-ABS-0-gray.png}{Начало расчёта}
            {png/satellite/2D-VELOCITY-ABS-3999-gray.png}{Сквозное пробивание внешней оболочки}

        По представленным результатам (см.~рис.~\ref{fig:satellite-2}) видно, что столкновение подобного рода ведёт
        к~серьёзным повреждениями спутника, что в~дальнейшем, скорее всего, делает невозможным его использование
        по~прямому назначению.

        % \newpage
        \subsubsection*{Заключение}

        В~заключении перечислены основные результаты, полученные в~ходе выполнения работы, а также сделаны выводы о
        применимости описанных методов для решения практически значимых задач.

    \subsection*{Основные результаты и выводы диссертации}

    \begin{enumerate}
        \item Реализован сеточно-характеристический численный метод с~использованием неcтруктурированных расчётных
              сеток для моделирования сложных волновых процессов, происходящих в~анизотропных композитных конструкциях.
              Метод обладает возможностью определения области разрушения в~результате воздействия различных динамических
              нагрузок, а также расчёта откликов разрушенных областей для решения задач неразрушающего контроля.
        \item Проведено исследование наиболее используемых на~данный момент критериев разрушения композиционного
              материала, по~результатам которого сделаны выводы о~применимости этих критериев для решения практически
              значимых задач динамической прочности композитных конструкций.
        \item Реализован и проверен на~модельных задачах программный комплекс, использующий комбинированный численный
              метод, основанный на~сеточно-характеристическом методе и методе сглаженных частиц. Получено качественное
              совпадение результатов решения модельных задач с~результатами расчёта при помощи коммерческого
              программного комплекса.
        \item Адаптирован для использования совместно с~сеточно-характеристи\-чес\-ким методом и проверен на~модельных
              задачах трёхмерный метод маркеров, позволяющий проводить численное моделирование в~задачах механики
              деформируемого твёрдого тела в~условиях конечных деформаций. Этот подход обеспечивает возможность решать
              задачи с~большими деформациями, а также позволяет в~несколько раз ускорить расчёт по~сравнению с
              использованием неструктурированных сеток для всего тела благодаря фиксированному шагу по~времени.
        \item Выполнено моделирование волновых процессов в~стекле, возникающих под действием лазерного излучения.
              На~качественном уровне получено совпадение с~результатами эксперимента.
        \item Проведено моделирование волновых процессов, сопровождающихся хрупким разрушением, в~многослойной
              стеклянной конструкции.
        \item Выполнено моделирование низкоскоростного удара по~композиционным панелям из~разных материалов. Получены
              результаты, свидетельствующие о~зависимости размеров и формы области расслоения от~используемого критерия
              объёмного разрушения.
        \item Проведено моделирование низкоскоростного удара по~трёхстрингерной композиционной панели с~использованием
              различных критериев разрушения. Получено совпадение с~результатами эксперимента на~качественном уровне.
        \item Выполнено моделирование процесса пробивания осколком преград различной толщины.
        \item Выполнены расчёты ряда задач о~высокоскоростном соударении тел: падение самолёта на
              здание и столкновение микрометеорита со~спутником.

    \end{enumerate}

    \subsection*{Список публикаций соискателя по~теме диссертации}
        \nocite{беклемышева2013численное,васюков2014комбинирование,петров2014схм,petrov2014combined,петров2014численный,петров2015комбинированный,беклемышева2014численное,беклемышева2012численное,ермаков2013построение,беклемышева2013численное2,беклемышева2013численное3}
        \begingroup
            \renewcommand{\section}[2]{}%
            \bibliography{biblio}
        \endgroup

    \subsection*{Личный вклад соискателя в~работах с~соавторами}

    Сеточно-характеристический метод реализован для численного решения задач о~динамическом поведении анизотропных
    композитных конструкций.

    В~части численных методов соискателем реализован метод, представляющий из~себя комбинацию метода сглаженных частиц и
    сеточно-ха\-рак\-те\-рис\-ти\-чес\-ко\-го метода. Проведена верификация реализованного метода на~ряде модельных задач
    распада разрыва. Также реализован метод маркеров, предполагающий использование структурированных эйлеровых сеток и
    сеточно-характеристического метода для моделирования процессов с~существенными изменениями геометрии расчётной
    области.

    В~части программной реализации соискателем проделана работа по~реализации описанных ранее методов, а также
    интеграции полученного программного комплекса с~библиотеками, использующимися для визуализации (paraview, pvbatch)
    в~автоматическом режиме.

    В~части математического моделирования соискателем выполнено исследование волновых процессов, возникающих в
    композиционных материалах при ударном нагружении и приводящих к~последующему разрушению материала и деламинации.
    Проведено численное моделирование последствий воздействия лазерного излучения на~стеклянный образец, а также
    проведено сравнение полученных данных с~экспериментом. Выполнено моделирование поведения многослойной прозрачной
    конструкции при ударном нагружении. Получена зависимость размеров области расслоения от~адгезионной прочности. Также
    выполнены расчёты нескольких высокоскоростных столкновений, в~частности, падения быстро движущегося объекта на~сферическую
    оболочку, столкновения микрометеорита с космическим спутником, а также пробивания стальным осколком стальных преград различной толщины.

    % Последняя страница
    \newpage
    \pagenumbering{gobble}
    \begin{center}\end{center}
    \vspace{24em}

    \begin{center}
        \textbf{Ермаков Алексей Сергеевич}
    \end{center}

    \vspace{1em}

    \begin{center}
        \textsc{\textbf{\disser}}
    \end{center}

    \vspace{0.5em}

    \begin{center}
        Автореферат
    \end{center}

    \vspace{0.5em}

    \begin{center}
        \begin{singlespace}
            Подписано в~печать 01.10.2015. Формат 60 x 84 $^1/_{16}$. Усл. печ. л. 1,0.\\
            Тираж 100 экз. Заказ 391.\\
            Федеральное государственное автономное образовательное учреждение высшего профессионального образования «Московский физико-технический
            институт (государственный университет)»\\
            Отдел оперативной полиграфии «Физтех-полиграф»\\
            141700, Московская обл., г. Долгопрудный, Институтский пер., д. 9
        \end{singlespace}
    \end{center}
\end{document}
