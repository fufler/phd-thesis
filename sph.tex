\documentclass[thesis.tex]{subfiles}

\begin{document}

\subsection{Метод сглаженных частиц}
\label{раздел:метод-сглаженных-частиц}

\subsubsection{Оригинальный метод}

Метод сглаженных частиц, изначально разработан \cite{monaghan1988introduction} для численного моделирования задач
гидродинамики, что отражено в~его названии: \textbf{S}moothed-\textbf{P}article \textbf{H}ydrodynamics. Несмотря на~это,
данный метод нашёл широкое применение при численном решении задач механики деформируемого твёрдого тела, особенно в~тех
случаях, когда имеют место большие деформации области интегрирования, сопровождающиеся разлётом вещества. Метод
сглаженных частиц является бессеточным численным методом: вместо расчётной сетки используется набор расчётных точек
(частиц), которые движутся вместе со~средой. Здесь стоит заметить, что эти точки следует воспринимать не как часть тела,
а как своеобразную замену расчётной сетки.

Существует достаточное количество реализаций метода сглаженных частиц, обзор которых можно найти в~работе
Паршикова~А.Н.~\cite{паршиков2013диссертация}. В~данной работе используется модифицированная версия оригинального метода, которая
описана в~работе Потапова А.П.~\cite{потапов2009диссертация}.

Главная идея метода сглаженных частиц состоит в~аппроксимации выражения
\begin{equation}
    a(x)=\int_R a(x')\delta(x'-x)dx'
    \label{полевая_функция}
\end{equation}
суммой аналитических функций, вычисленных в~определённых точках. Для этого вводится функция $\omega(x,h)$, называемая
ядром сглаживания, удовлетворяющая условиям
\begin{align}
    \int_R\omega(x,h)dx=1 \label{ядро_сглаживания} \\
    \omega(x,h) \xrightarrow[h \to 0]{} \delta(x). \nonumber
\end{align}

После подстановки аппроксимации \ref{ядро_сглаживания} в~выражение \ref{полевая_функция} получим
\[
    a(x) = \int_R a(x')\omega(x'-x,h)dx'=\int_R \left(  \frac{a(x')}{\rho(x')} \right) \omega(x'-x,h)\rho(x')dx',
\]
где $\rho(x)$~--- плотность рассматриваемой среды.

Как было сказано ранее, вместо расчётной сетки используется набор частиц, в~каждой из~которых имеются определённые
значения полевой функции $a_i = a(x_i)$. Если предположить помимо этого, что для каждой частицы известны плотность
материала $\rho_i$ и масса $m_i$, то можно записать следующую аппроксимацию полевой функции
\[
    a(x) = \sum_i \frac{m_i a_i}{\rho_i}\omega(x_i-x,h),
\]
таким образом сведя интегрирование к~операции суммирования по~точкам.

Для такой аппроксимации операция вычисления производной полевой функции по~сути сводится к~дифференцированию
аналитической функции~--- ядра сглаживания
\[
    \PD{a(x)}{x_{\alpha}} = \sum_i \frac{m_i a_i}{\rho_i}\PD{\omega(x_i-x,h)}{x_{\alpha}}.
\]

Как было сказано ранее, в~общем случае на~ядро сглаживания не накладывается особых ограничений, кроме \ref{ядро_сглаживания}.
Однако раз ядром сглаживания определяется характерное расстояние взаимодействия в~среде (см.рис.~\ref{рис:ядро_сглаживания}),
то, очевидно, что носитель этой функции должен быть конечным.

\begin{figure}[h]
    \begin{center}
        \tikzset{every picture/.style={scale=0.75}}
        \subfile{tikz/sph}
    \end{center}
    \caption{Область взаимодействия частиц}
    \label{рис:ядро_сглаживания}
\end{figure}

В данной работе используется следующее выражения для ядра сглаживания:
\begin{equation}
    \omega(x,h) = \left\{\begin{aligned}
        \frac{1-\frac{3}{2}\phi^2+\frac{3}{4}\phi^3}{\pi h^3}, & \phi \in [0,1), \nonumber \\
        \frac{(2-\phi)^3}{4\pi h^3}, & \phi \in [1,2], \nonumber \\
        0, & \phi \in (2, \infty),
    \end{aligned} \right.
\end{equation}
где $\phi = \frac{\abs{x-x'}}{h}$.

После подстановки выражения для полевой функции в~исходные уравнения можно получить их аппроксимации в~следующем виде:
\begin{small}
\begin{gather}
    \TDt{\rho_i} = -\sum_k m_k \left( u_k^\alpha - u_i^\alpha \right) \PD{\omega_{ik}}{x_i^\alpha}, \nonumber \\
    \TDt{u_i^\alpha} = \sum_k m_k \left( \frac{\sigma_i^{\alpha\beta}}{\rho_i^2}+\frac{\sigma_k^{\alpha\beta}}{\rho_k^2} \right) \PD{\omega_{ik}}{x_i^\beta},  \nonumber \\
    \TDt{e_i} = \sum_k m_k \left( u_i^\alpha - u_k^\alpha \right) \left( \frac{\sigma_i^{\alpha\beta}}{\rho_i^2}+\frac{\sigma_k^{\alpha\beta}}{\rho_k^2} \right) \PD{\omega_{ik}}{x_i^\beta},  \nonumber \\
    \TDt{S_i^{\alpha\beta}} = 2\mu \left( \varepsilon_i^{\alpha\beta} - \frac{1}{3}\delta^{\alpha\beta}\varepsilon_i^{\alpha\beta} \right)  +
                              S_i^{\alpha\gamma}R_i^{\beta\gamma} + S_i^{\gamma\beta}R_i^{\alpha\gamma}, \nonumber \\
    \varepsilon_i^{\alpha\beta} = \frac{1}{2}\sum_k \frac{m_k}{\rho_k} \left(
                                  \left( u_k^\alpha - u_i^\alpha \right)\PD{\omega_{ik}}{x_i^\beta} +
                                  \left( u_k^\beta - u_i^\beta \right)\PD{\omega_{ik}}{x_i^\alpha} \right), \nonumber \\
    R_i^{\alpha\beta} = \frac{1}{2}\sum_k \frac{m_k}{\rho_k} \left(
                        \left( u_k^\alpha - u_i^\alpha \right)\PD{\omega_{ik}}{x_i^\beta} -
                        \left( u_k^\beta - u_i^\beta \right)\PD{\omega_{ik}}{x_i^\alpha} \right).
\end{gather}
\end{small}

В таком случае интегрирование исходных уравнений для $i$-й частицы может быть выполнено следующим образом:
\begin{small}
\begin{align}
    \vec x_i^{n+1} &= \vec x_i^{n} + \Delta t \left( \vec u_i^n + \frac{1}{2}\Delta t \TDt{\vec u_i^n} \right), \nonumber \\
    e_i^{n+1} &= e_i^n + \Delta t \TDt{e_i^n}, \nonumber \\
    \rho_i^{n+1} &= \rho_i^n + \Delta t \TDt{\rho_i^n}, \nonumber \\
    \vec u_i^{n+1} &= \vec u_i^n + \Delta t \TDt{\vec u_i^n}, \nonumber \\
    S_i^{n+1} &= \hat S_i^n + \Delta t \TDt{S_i^n}, \nonumber \\
    \sigma_i^{n+1} &= -\delta P(\rho_i^{n+1}, e_i^{n+1}) + S_i^{n+1}.
\end{align}
\end{small}

Шаг интегрирования определяется по~формуле
\begin{equation}
    \small
    \Delta t = \min_i \frac{a h_i}{\sqrt{(\max(c_i, u_i))^2 + \left( \frac{h_i \dot \rho_i}{\rho_i} \right)^2 }},
\end{equation}

где $u_i$~--- скорость $i$-й частицы, $c_i$~--- скорость звука в~$i-й$ частице, $h_i$~--- радиус сглаживания $i$-й частицы,
$\rho_i$~--- плотность $i$-й частицы, $\alpha$~--- параметр, зависящий от~используемого метода.

\subsubsection{Модификация метода}

Рассмотренный выше метод обладает рядом существенных недостатков, среди которых немонотонность и наличие нефизичных
осцилляций. Полное описание возникающих проблем и способов решения приводится в~работе \cite{потапов2009диссертация},
здесь же даётся лишь краткое описание отличий используемого метода от~оригинального.

Для борьбы с~нефизичным осцилляциями можно \cite{monaghan1989problem} использовать искусственную вязкость. Суть этой идеи
заключается в~добавлении дополнительного члена $\frac{a\xi_{ij}\bar c_{ik}+b\xi_{ik}^2}{\bar \rho_{ik}}$ в~множители
вида
$\left( \frac{\sigma_i^{\alpha\beta}}{\rho_i^2} + \frac{\sigma_k^{\alpha\beta}}{\rho_k^2}\right) $, где
\[
        \xi_{ik} = \frac{(u_i^\alpha-u_k^\alpha)(x_i^\alpha-x_k^\alpha)h}{(x_i^\alpha-x_k^\alpha)^2+0.01h^2}
\]
и $\bar c_{ik}$~--- средняя скорость звука, $\bar \rho_{ik}$~--- средняя плотность, $a$ и $b$~--- коэффициенты
искусственной вязкости.

Следующий метод \cite{monaghan1997sph} заключается в~использовании аналитического решения задачи Римана, а именно
замене всех выражений вида $a_i+a_j$ и $a_i-a_j$ на~$2a_{ij}^*$ и $2(a_{ij}^*-a_j)$ соответственно, где $a_i$ и $a_j$ ---
значения полевой функции в~$i$-й и $j$-й частицах соответственно, а $a_{ij}^*$~--- решения задачи распада разрыва. Такой
способ позволяет избавиться от~немонотонности, но при этом размывает фронт ударной волны, что, в~принципе, является весьма
характерным особенностью подобных методов.

Дальнейшие улучшения \cite{потапов2009диссертация} метода основываются на~идеях гибридизации, т.е. переключения между
численными алгоритмами решения в~зависимости от~гладкости полевой функции.  Построенный таким образом алгоритм
при правильно подобранных коэффициентах гибридизации позволяет получить более устойчивый и монотонный численный метод,
нежели оригинальный.

\end{document}
