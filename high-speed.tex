\documentclass[thesis.tex]{subfiles}

\begin{document}

\clearpage
\subsection{Высокоскоростное столкновение массивных тел}

В данном разделе рассматриваются задачи о~высокоскоростном соударении массивных объектов. Решение задач такого
типа представляет достаточно большой интерес с~практической точки зрения, при этом проведение натурных экспериментов
практически всегда невозможно ввиду высокой стоимости или большой опасности.

\subsubsection[Моделирование последствий падения самолёта\\на~крышу здания]{Моделирование последствий падения самолёта на~крышу здания}

Рассматривается задача о~падении лёгкого самолёта на~крышу здания, имеющего форму полусферы. Самолёт моделируется полой
оболочкой, выполненной из~дюралюминия, длиной 8~м, диаметром 3~м и толщиной 0.25~м. Крыша здания изготовлена из~бетона
и имеет диаметр 40~м и толщину стенки 2~м. Скорость столкновения составляет 250~м/с.

Результаты расчётов представлены на~рис.  \ref{рис:падение-самолёта-1}--\ref{рис:падение-самолёта-6}.

\twofigs{рис:падение-самолёта-1}{Падение самолёта на~здание}
        {png/air-nuke/2D-VELOCITY-ABS-0.png}{Начало расчёта}
        {png/air-nuke/2D-VELOCITY-ABS-6.png}{Время 3~мс}

\twofigs{рис:падение-самолёта-2}{Падение самолёта на~здание}
        {png/air-nuke/2D-VELOCITY-ABS-12.png}{Время 5~мс}
        {png/air-nuke/2D-VELOCITY-ABS-18.png}{Время 7~мс}

\twofigsT{рис:падение-самолёта-3}{Падение самолёта на~здание}
        {png/air-nuke/2D-VELOCITY-ABS-24.png}{Время 10~мс}
        {png/air-nuke/2D-VELOCITY-ABS-30.png}{Время 12~мс}

\twofigsT{рис:падение-самолёта-4}{Падение самолёта на~здание}
        {png/air-nuke/2D-VELOCITY-ABS-36.png}{Время 15~мс}
        {png/air-nuke/2D-VELOCITY-ABS-42.png}{Время 17~мс}

\twofigsT{рис:падение-самолёта-5}{Падение самолёта на~здание}
        {png/air-nuke/2D-VELOCITY-ABS-48.png}{Время 20~мс}
        {png/air-nuke/2D-VELOCITY-ABS-54.png}{Время 22~мс}

\twofigsT{рис:падение-самолёта-6}{Падение самолёта на~здание}
        {png/air-nuke/2D-VELOCITY-ABS-60.png}{Время 25~мс}
        {png/air-nuke/2D-VELOCITY-ABS-69.png}{Время 28~мс}


Как видно из~результатов, такое столкновение ведёт к~полному разрушению половины крыши, что впоследствии с~высокой
вероятностью вызовет обрушение всего купола здания.

Отдельно был рассмотрен вопрос об~изменении размера разрушенной области при учёте наличия двигателей самолёта.
Для этого была была рассчитана другая постановка, в~которой в~качестве преграды выступает бетонная плита толщиной 2~м,
фюзеляж самолёта моделируется полой железной оболочкой длиной 20~м, диаметром 6~м и толщиной 0.5~м. Двигатели
моделируется железными шарами радиусом 1~м. Скорость столкновения составляет 320~м/с.

Результаты расчёта представлены на~рис.~\ref{рис:самолёт-с-двигателями-1}--\ref{рис:самолёт-с-двигателями-5}.

Как видно из~результатов расчёта, при моделировании самолёта с~учётом наличия двигателей область разрушения оказывается
значительно больше: та часть преграды, что не была пробита фюзеляжем самолёта,  но находилась в~зоне непосредственного
контакта тел, разрушается, испытывая повторное ударное нагружение вследствие столкновения с~двигателями самолёта. Стоит
отметить, что, судя по~результатам моделирования, сами двигатели в~отличие от~фюзеляжа при этом не испытывают
существенных деформаций и продолжают и двигаться на~высокой скорости, что в~дальнейшем может привести к~заметным
повреждениям конструкций, находящихся за~преградой.

\twofigs{рис:самолёт-с-двигателями-1}{Столкновение самолёта с~бетонной стеной}
        {png/air-nuke-engines/2D-VELOCITY-ABS-0.png}{Начало расчёта}
        {png/air-nuke-engines/2D-VELOCITY-ABS-45.png}{Время 5~мс}

\twofigs{рис:самолёт-с-двигателями-2}{Столкновение самолёта с~бетонной стеной}
        {png/air-nuke-engines/2D-VELOCITY-ABS-90.png}{Время 10~мс}
        {png/air-nuke-engines/2D-VELOCITY-ABS-135.png}{Время 15~мс}

\twofigs{рис:самолёт-с-двигателями-3}{Столкновение самолёта с~бетонной стеной}
        {png/air-nuke-engines/2D-VELOCITY-ABS-180.png}{Время 20~мс}
        {png/air-nuke-engines/2D-VELOCITY-ABS-225.png}{Время 25~мс}

\twofigs{рис:самолёт-с-двигателями-4}{Столкновение самолёта с~бетонной стеной}
        {png/air-nuke-engines/2D-VELOCITY-ABS-270.png}{Время 30~мс}
        {png/air-nuke-engines/2D-VELOCITY-ABS-315.png}{Время 35~мс}

\twofigs{рис:самолёт-с-двигателями-5}{Столкновение самолёта с~бетонной стеной}
        {png/air-nuke-engines/2D-VELOCITY-ABS-360.png}{Время 40~мс}
        {png/air-nuke-engines/2D-VELOCITY-ABS-399.png}{Время 45~мс}

\FloatBarrier

\subsubsection{Столкновение микрометеорита со~спутником}

Рассматривается задача о~высокоскоростном столкновении спутника и микрометеорита. Спутник моделируется в~виде двух
концентрических сферических слоёв: внешнего радиусом 2.5~м и толщиной 0.4~м и внутреннего радиусом 1.7~м и толщиной 0.2~м.
Микрометеорит моделируется шаром радиусом 0.3~м. Обе оболочки спутника изготовлены из~пластика, микрометеорит
предполагается полностью состоящим из~железа. Скорость столкновения составляет 2~км/с.

Результаты расчётов представлены на~рис.~\ref{рис:спутник-1}--\ref{рис:спутник-5}.

По результатам расчётов можно сделать вывод, что такое столкновение приводит к~полному пробиванию обеих оболочек
спутника, что гарантированно приводит к~выходу из~строя находящегося внутри оборудования.


\twofigs{рис:спутник-1}{Столкновение микрометеорита со~спутником}
        {png/satellite/2D-VELOCITY-ABS-0.png}{Начало расчёта}
        {png/satellite/2D-VELOCITY-ABS-450.png}{Время 1~мс}

\twofigs{рис:спутник-2}{Столкновение микрометеорита со~спутником}
        {png/satellite/2D-VELOCITY-ABS-900.png}{Время 2~мс}
        {png/satellite/2D-VELOCITY-ABS-1350.png}{Время 3~мс}

\twofigs{рис:спутник-3}{Столкновение микрометеорита со~спутником}
        {png/satellite/2D-VELOCITY-ABS-1800.png}{Время 4~мс}
        {png/satellite/2D-VELOCITY-ABS-2250.png}{Время 5~мс}

\twofigs{рис:спутник-4}{Столкновение микрометеорита со~спутником}
        {png/satellite/2D-VELOCITY-ABS-2700.png}{Время 6~мс}
        {png/satellite/2D-VELOCITY-ABS-3150.png}{Время 7~мс}

\twofigs{рис:спутник-5}{Столкновение микрометеорита со~спутником}
        {png/satellite/2D-VELOCITY-ABS-3600.png}{Время 8~мс}
        {png/satellite/2D-VELOCITY-ABS-3999.png}{Время 9~мс}

\end{document}
